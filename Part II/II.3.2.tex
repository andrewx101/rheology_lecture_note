\documentclass[main.tex]{subfiles}
% 欧几里得空间的几何对象
\begin{document}
本节我们依次在欧几里得空间中引入角、直线、位置向量和坐标系。

为了引入角,我们考虑欧几里得空间$\mathcal{E}$中的给定三个不同的点$X,O,Y\in\mathcal{E}$,由于$\mathcal{E}$的平移空间$\mathcal{V}$是一个赋范内积空间,故有极化恒等式,
\begin{align*}
    \left\|X-O\right\|^2+\left\|Y-O\right\|^2 & =\left\|\left(X-O\right)-\left(Y-O\right)\right\|^2+2\left(X-O|Y-O\right) \\
                                              & =\left\|X-Y\right\|^2+2\left(X-O|Y-O\right)
\end{align*}
再应用柯西--施瓦茨不等式,有
\[\left\|X-O\right\|^2\left\|Y-O\right\|^2\geq\left|\left(X-O|Y-O\right)\right|^2\Leftrightarrow-1\leq\frac{\left(X-O|Y-O\right)}{\left\|X-O\right\|\left\|Y-O\right\|}\leq1\]
我们就做好了引入角的准备。

\begin{definition}[角]\label{def:II.3.6}
    设$\mathcal{E}$是欧几里得空间,$\mathcal{E}$中的角是一个映射$\angle:\mathcal{E}^3\rightarrow\mathbb{R}$满足
    \[\angle XOY\equiv\cos^{-1}\frac{\left(X-O|Y-O\right)}{\left\|X-O\right\|\left\|Y-O\right\|},\quad\forall X,O,Y\in\mathcal{E},X\neq O\neq Y\]
    称\emph{点XOY所夹的角},或简称\emph{角XOY}。其中余弦函数$\cos:\left[0,\pi\right]\rightarrow\mathbb{R}$定义为
    \[\cos\left(x\right)=\frac{1}{2}\left(e^{ix}+e^{-ix}\right),\quad\forall x\in\left[0,\pi\right)\]
\end{definition}

注意到,上述定义中的余弦函数是一个双射,故其逆映射$\cos^{-1}$也是双射,所以上述定义的角的取值范围是$\mathrm{ran}\angle=\left[0,\pi\right]$。同时$\angle XOY$的顺序是重要的,$\angle YOX=-\angle XOY$。

%在欧几里得空间中,把3个不同的点平移相同的向量,它们的夹角不变。这一结论在上述的定义下是可验证的。设$\mathcal{V}$是欧几里得空间$\mathcal{E}的平移空间,给定任意$\mathbf{a}\in\mathcal{V}$,以及任意三个两两不同的点$X,O,Y\in\mathcal{E}$,有$\angle XOY=\angle\left(X+\mathbf{a}\right)\left(O+\mathbf{a}\right)\left(Y+\mathbF{a}\right)$。虽然这个结论直观上很容易接受,但是要严格用定义\label{def:II.3.6}证明还是比较繁琐的。

设$i:\mathcal{E}\rightarrow\mathcal{E}$是一个等距变换,可验证$\angle i\left(X\right)i\left(O\right)i\left(Y\right)=\angle XOY,\forall X,O,Y\in\mathcal{E},X\neq O\neq Y$,即等距变换前后角不变。注意到,等距变换属于$\mathcal{E}$的等距群,

\begin{definition}[过两点的直线]\label{def:II.3.7}
    设$\left(\mathcal{E},d\right)$是欧几里得空间,给定两点$X,Y\in\mathcal{E},X\neq Y$,则$\mathcal{E}$的子集$L_{XY}=\left\{C|C=X+\alpha\left(Y-X\right),\alpha\in\mathbb{R}\right\}$是\emph{过}$X,Y$\emph{两点的一条直线}。如果$\angle XOY=\frac{\pi}{2}$,则直线$L_{OX}$与$L_{OY}$\emph{垂直},记为$L_{OX}\perp L_{OY}$。
\end{definition}

由角的定义,如果$L_{OX}\perp L_{OY}$,则$\left(X-O|Y-O\right)=0$。再由内积空间的格拉姆--施密特正交化过程可知,过$\mathcal{E}$中任一点$O$的两两垂直的直线最大条数都相等且等于$\mathrm{dim}\mathcal{V}$,故欧几里得空间的维数就可被自然地定义为其平移空间的维数。

$L_{XY}$又可记为$L_{XY}=\left\{C|C-X=\alpha\left(Y-X\right),\alpha\in\mathbb{R}\right\}$,它对应着平移向量空间$\mathcal{V}$的子集$L^{\mathcal{V}}_{XY}=\left\{\mathbf{u}|\mathbf{u}=\alpha\left(X-Y\right),\alpha\in\mathbb{R}\right\}$,易知该子集是$\mathcal{V}$的子空间,维数是1\footnote{这里需要实数集的完备性概念。}。

我们将一个选定的原点$O\in\mathcal{E}$和$\mathcal{V}$的一组规范正交基的组合$\left(O,\left\{\mathbf{\hat{e}}_i\right\}\right)$称为欧几里得空间$\mathcal{E}$的一个\emph{直角坐标系(rectangular coordinates)},又称\emph{笛卡尔坐标系(Cartesian coordinates)}。我们常常默认一个$n$维欧几里得空间必然已经自带一个直角坐标系,称为\emph{基本坐标系(common coordinates)},从而直接采用$\mathbb{R}^n$来表示任意一点点的坐标。在基本坐标系下,原点坐标为$\left(0,\cdots,0\right)$,第$i$个基向量为$\left(0,\cdots,1,\cdots,0\right)^\intercal$,也就是除第$i$个分量为1外其他分量均为零的有序实数$n$元组。选定了原点$O$后,对任一点$X\in\mathcal{E}$可定义映射$\mathbf{r}_O:\mathcal{E}\rightarrow\mathcal{V},\mathbf{r}\left(X\right)\equiv\mathbf{r}_X=X-O,\forall X\in\mathcal{E}$,我们称这个向量值函数$\mathbf{r}_X$就是选定原点$O$下点$X$的\emph{位置向量(position vector)}。注意,当且仅当选定了原点后,欧几里得空间$\mathcal{E}$中的点才与其平移空间$\mathcal{V}$的向量通过位置向量这个映射一一对应。

总而言之,一个(有限维)欧几里得空间$\left(\mathcal{E},d,O,\mathcal{V},\left\{\mathbf{\hat{e}}_i\right\}\right)$包括:
\begin{enumerate}
    \item 一个度量空间$\left(\mathcal{E},d\right)$
    \item 一个实数域上的$n$维内积空间$\mathcal{V}$。它是该欧几里得空间的平移空间,意味着:
          \begin{align*}
              \forall\mathbf{a}\neq\mathbf{0}\wedge\mathbf{a}\in\mathcal{V}\exists!\left(X,Y\right)\in\mathcal{E}\times\mathcal{E},\mathbf{a}=Y-X \\
              X-X=\mathbf{0}\forall X\in\mathcal{E}
          \end{align*}
          且$n$同时定义了该欧几里得空间的维数。
    \item 选定了原点$O\in\mathcal{E}$
    \item 选定了一组规范正交基$\left\{\mathbf{\hat{e}}_i\right\}\subset\mathcal{V}$

          最后两项同时也使欧几里得空间默认带有一个直角坐标系,以及位置向量的概念。
\end{enumerate}

明确了欧几里得空间的完整概念之后,为了简便我们仍只用$\mathcal{E}$表示一个欧几里得空间。

本节所定义的欧几里得空间是欧几里得几何的现代基础。基于《几何原本》的公设得到的大量欧氏几何定理仍然成立,因为这些公设的要求已蕴含在了实数域的、向量内积的性质以及度量的性质中了\cite{Audin2002}。更重要的是,明确了这一线性结构后,我们能够用统一的数学语言推导出更多几何结论\cite{Berger1987}。
\end{document}