\documentclass[main.tex]{subfiles}
\begin{document}
本讲义不是一个完整、深入的流变学教材,仅为了帮助化工类本科专业的人士建立必要的数学基础,为理解其他正规的流变学教材和论文作准备。

讲义主要内容分两部分。

第一部分是必要的数学基础。第二部分是流变学的物理理论基础。

关于数学基础部分的内容深度和广度,我作了如下考虑:
\begin{enumerate}
    \item 从集合论开始介绍。集合论是近世数学的基础,然而近几年工科大学的高等数学教学以为集合论知识已下沉到中学,就普遍不再作正面介绍。笔者不完全调查发现也不是所有中学都学习了集合论;就算学习了,中学的集合论内容也非常浅。因此本讲义从集合论开始介绍。
    \item 以化学类工科专业大一学习的微积分和线性代数为基础。具体地,我假定读者已完整地、熟练地掌握华南理工大学数学系编著的《高等数学(上、下册)》和《线性代数与解析几何》,故只介绍上述课本中没有涉及到的数学知识。我在讲义中尽可能多地提示了所介绍的新内容跟上述课本中已经介绍过哪些内容相关(精确到章、节、页码)。
    \item 采用不依赖坐标的表达风格(不介绍也不使用求和约定)。我希望这种做法本身能彰显流变学理论的创立者所希望保证的\emph{物理客观性};即材料的响应规律不依赖包括坐标系的选择在内的任何来自观察者的主观选择而变化。向量空间、线性变换和曲线坐标系等知识的纳入和介绍方式,如果说与其他类似的教材有所不同,那么主要也是为了在数学语言上保证正确描述这一物理观念。
    \item 仅为了证明某定理所需的数学概念和引理,要么不正式介绍、要么放在附录中介绍。定理的证明过程使用浅灰色,仅供有兴趣的读者参考,故未必都提供。不提供时,我尽可能提出其他提供了证明过程的参考资料。
    \item 本讲义的物理部分用不到的数学知识,尽量不作介绍。例如,在集合论的章节中提到了部分公理集合论的公理,但又假定关于自然数、算术运算、偏序、全序、数学归纳法等知识已为为读者所熟知,故包括皮亚诺公设在内的相关的公设就不作介绍了。这使得第一章深度上似乎要介绍公理集合论,但广度上又远非一套完整的公理集合论资料,似乎是一篇“四不像”。类似的情况在所有数学部分章节都有,只因本讲义的目标是要向化学专业的读者补充看懂流变学理论的最少必要的数学知识。本讲义每章都引用了推荐的参考书,供想要更完整地学习该章知识读者选读。
\end{enumerate}

一般情况下,概念的定义仅通过字体的改变来暗示。例如,\emph{集合(set)}是具有某种特性的事物的整体。仅在需要时,定义才以带编号的方式引入。而定理、引理和例子则均带编号。定义是极其重要的。它在文中只出现一次,因此难免要经常反复回顾。不采用带编号式的引入,只是因为需要定义的概念很多,如果每个定义都带编号定义将会严重打断行文的流畅性,让本来就抽象的内容更难以阅读,而代价则是使定义失去了引用链接的便利,故在此敬请读者在学习时贴上标签,不厌其烦地翻阅回顾定义。

关于物理理论基础部分的内容深度和广度,我作了如下考虑:(待补充。)

更多说明将在讲义完成之后再在此解释。
\begin{flushright}
    孙尉翔\\
    2020年10月
\end{flushright}
\end{document}

