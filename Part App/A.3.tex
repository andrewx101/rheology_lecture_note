\documentclass[main.tex]{subfiles}
\begin{document}
%====================================================
\subsection{伴算算符的唯一存在性}\label{sec:A.3.1}
\begin{theorem}
    设$\mathcal{V}$是数域$\mathbb{F}$上的有限维内积空间,对$\mathcal{V}$上的任一线性算符$\mathbf{T}\in\mathcal{L}\left(\mathcal{V}\right)$有且只有一个伴随算符$\mathbf{T}^*\in\mathcal{L}\left(\mathcal{V}\right)$。
\end{theorem}
\begin{proof}
    给定任意向量$\mathbf{b}\in\mathcal{V}$,均可定义一个线性泛函$f\in\mathcal{V}^*,f\left(\mathbf{a}\right)\equiv\left(\mathbf{Ta}|\mathbf{b}\right),\forall\mathbf{a}\in\mathcal{V}$。由有限维Riesz表示定理(\ref{thm:II.2.19}),每一个这样的线性泛函都唯一对应一个$\mathbf{b}^\prime\in\mathcal{V}$满足$f\left(\mathbf{a}\right)=\left(\mathbf{a}|\mathbf{b}^\prime\right)$。因此,对每一个线性变换$\mathbf{T}\in\mathcal{L}\left(\mathcal{V}\right)$,都可以定义一个映射$\mathbf{T}^*:\mathcal{V}\rightarrow\mathcal{V}$来将$\mathcal{V}$中的每一个向量$\mathbf{b}$如上所述地对应到$\mathcal{V}$中的另一个向量$\mathbf{b}^\prime$,且这个映射$\mathbf{T}^*$是一个线性算符,因为对任意$\gamma\in\mathbb{F},\mathbf{b},\mathbf{c}\in\mathcal{V}$,
    \begin{align*}
        \left(\mathbf{a}|\mathbf{T}^*\left(\gamma\mathbf{b}+\mathbf{c}\right)\right) & =\left(\mathbf{Ta}|\gamma\mathbf{b}+\mathbf{c}\right)                                                           \\
                                                                                     & =\overline{\gamma}\left(\mathbf{Ta}|\mathbf{b}\right)+\left(\mathbf{Ta}|\mathbf{c}\right)                       \\
                                                                                     & =\overline{\gamma}\left(\mathbf{a}|\mathbf{T}^*\mathbf{b}\right)+\left(\mathbf{a}|\mathbf{T}^*\mathbf{c}\right) \\
                                                                                     & =\left(\mathbf{a}|\gamma\mathbf{T}^*\mathbf{b}+\mathbf{T}^*\mathbf{c}\right),\forall\mathbf{a}\in\mathcal{V}    \\
        \Leftrightarrow\mathbf{T}^*\left(\gamma\mathbf{b}+\mathbf{c}\right)          & =\gamma\mathbf{T}^*\mathbf{b}+\mathbf{T}^*\mathbf{c}
    \end{align*}

    现证明$\mathbf{T}^*$是唯一的。设另一线性算符$\mathbf{U}\in\mathcal{L}\left(\mathcal{V}\right)$满足命题条件即$\left(\mathbf{Ta}|\mathbf{b}\right)=\left(\mathbf{a}|\mathbf{T}^*\mathbf{b}\right)=\left(\mathbf{a}|\mathbf{Ub}\right)$,则$0=\left(\mathbf{a}|\mathbf{T}^*\mathbf{b}\right)-\left(\mathbf{a}|\mathbf{Ub}\right)=\left(\mathbf{a}|\left(\mathbf{T}^*-\mathbf{U}\right)\mathbf{b}\right),\forall\mathbf{a},\mathbf{b}\in\mathcal{V}\Leftrightarrow\mathbf{T}^*=\mathbf{U}$,故$\mathbf{T}^*$是唯一的。
\end{proof}

% ===================================================
\subsection{定理\ref{thm:II.2.32}“4”的证明}
若向量空间$\mathcal{V}$上定义了线性算符$\mathbf{T}$,则子空间$\mathcal{W}$上的向量被$\mathbf{T}$作用后可能不再属于$\mathcal{W}$,于是我们定义以下的特殊情况。

\begin{definition}[子空间在线性算符的作用下不变]
    设$\mathbf{T}$是向量空间$\mathcal{V}$上的一个线性算符,$\mathcal{W}$是$\mathcal{V}$的一个子空间。如果$\mathbf{Ta}\in\mathcal{W},\forall\mathbf{a}\in\mathcal{W}$,则称\emph{子空间$\mathcal{W}$在线性算符$\mathbf{T}$的作用下不变(invariant under $\mathbf{T}$)}。
\end{definition}

显然,任一向一空间$\mathcal{V}$在其上的任一线性算符的作用下不变。

\begin{theorem}[定理\ref{thm:II.2.32}“4”]
    给定有限维内积空间$\mathcal{V}$上的自伴随线性算符$\mathcal{T}$,$\mathcal{V}$中总存在一组规范正交基是$\mathbf{T}$的特征向量。
\end{theorem}
\begin{proof}
    数学归纳法。记$n=\mathrm{dim}\mathcal{V}$。设$\mathbf{a}_1$是$\mathbf{T}$的一个(非零)特征向量(用到了定理\ref{thm:II.2.32}“3”),则易验单位向量$\mathbf{\hat{a}}_1=\mathbf{a}_1/\left\|\mathbf{a}\right\|$与$\mathbf{a}_1$同属于一个特征空间,故当$n=1$时命题成立。

    假定当维数大于1且小于$n$时命题成立,令$\mathcal{W}$为由$\mathbf{\hat{a}}_1$线性生成的1维子空间,则$\mathbf{\hat{a}}_1$是$\mathbf{T}$的一个特征向量$\Leftrightarrow\mathcal{W}$在$\mathbf{T}$的作用下不变。易验此时$\mathcal{W}^\perp$在$\mathbf{T}^*$的作用下不变。而$\mathcal{W}^\perp$是$n-1$维内积空间。设$\mathbf{U}$是$\mathbf{T}$限制在$\mathcal{W}^\perp$上的线性算符,易验$\mathbf{U}$是$\mathcal{W}^\perp$上的自伴随算符。由数学归纳法假设,$\mathcal{W}^\perp$中必有一组规范正效基$\left\{\mathbf{\hat{a}}_2,\cdots,\mathbf{a}_n\right\}$是$\mathbf{U}$的特征向量,且易验它们也都是$\mathbf{T}$的特征向量。由定理\ref{thm:A.4},$\mathcal{V}=\mathcal{W}\bigoplus\mathcal{W}^\perp$且$\mathbf{\hat{a}}_1\in\mathcal{W}$,故有$\mathbf{\hat{a}}_1$与$\left\{\mathbf{\hat{a}}_2,\cdots\mathbf{\hat{a}}_n\right\}$都正交,即$\left\{\mathbf{\hat{a}}_1,\cdots\mathbf{\hat{a}}_n\right\}$是一组规范正交基。
\end{proof}

\end{document}