\documentclass[main.tex]{subfiles}
% 正规算符
\begin{document}
\begin{definition}[正规算符]\label{def:II.2.24}
    设$\mathcal{V}$是数域$\mathbb{F}$上的内积空间,$\mathbf{T}\in\mathcal{L}\left(\mathcal{V}\right)$是一个线性算符,如果$\mathbf{TT}^*=\mathbf{T}^*\mathbf{T}$,则称$\mathbf{T}$是\emph{正规算符(normal operator)}。
\end{definition}

很容易验证,厄米算符和幺正算符都是正规算符。在复数域上,正规算符必可对角化,因为如下的定理

\begin{theorem}\label{thm:II.2.35}
    设$\mathcal{V}$是复数域$\mathbb{C}$上的内积空间,$\mathbf{T}\in\mathcal{L}\left(\mathcal{V}\right)$是一个正规算符,则$\mathcal{V}$中必有一组规范正交基是$\mathbf{T}$的特征向量。
\end{theorem}

该定理的证明路径较长,此略\cite[\S 8.5 Theorem 22]{Hoffman1971}。但值得注意的是定理\ref{thm:II.2.32}“4”已证明该结论至少对厄米算符是成立的。

结合此定理与定理\ref{thm:II.2.28}可知,在复数域上,正规算符必可幺正地对角化。而由定理\ref{thm:II.2.32}的推论,在实数域上,对称算符必可正交地对角化。

其逆命题也成立,即在复数域$\mathbb{C}$上,可幺正地对角化的算符必为正规算符。因为说一个线性算符$\mathbf{T}$可幺正地对角化,就是说存在一组规范正交基$B$满足$\left(T\right)_B=\mathrm{diag}\left(\lambda_1,\cdots,\lambda_n\right)$,其中$\left\{\lambda_i\right\}$是$\mathbf{T}$的特征值。若$Q$是从$B$到另一规范正交基$B^\prime$的过渡矩阵,则$\left(\mathbf{T}\right)_{B^\prime}=Q\left(\mathbf{T}\right)_{B}Q^{-1}$。易验$\left(\mathbf{T}\right)_{B^\prime}\overline{\left(\mathbf{T}\right)^\intercal_{B^\prime}}=\overline{\left(\mathbf{T}\right)^\intercal_{B^\prime}}\left(\mathbf{T}\right)_{B^\prime}$,亦即$\mathbf{TT}^*=\mathbf{T}^*\mathbf{T}$,故$\mathbf{T}$为正规算符。

还有两个定理不是关于正规算符的,但却需要引入正规算符及其谱定理才可证明,本讲义无法详细介绍。不过,这两个定理使得复数域上的线性算符很像复数。比如,厄米算符的所有特征值都是实数,因此类比于“实数”,我们可以给厄米算符定义“正”或“非负”的概念如下。

\begin{definition}\label{def:II.2.25}
    设$\mathcal{V}$是数域$\mathbb{F}$上的有限维内积空间,$\mathbf{T}$是$\mathcal{V}$上的自伴随算符。
    \begin{enumerate}
        \item 如果$\left(\mathbf{Ta}|\mathbf{a}\right)\geq 0,\quad\forall\mathbf{a}\in\mathcal{V}$,则称$\mathbf{T}$为\emph{非负(non-negative)}算符;
        \item 如果$\left(\mathbf{Ta}|\mathbf{a}\right)> 0,\quad\forall\mathbf{a}\in\mathcal{V}$,则称$\mathbf{T}$为\emph{正(positive)}算符。
    \end{enumerate}
\end{definition}

以下定理类似“非负实数的平方根是非负实数”。

\begin{theorem}\label{thm:II.2.36}
    设$\mathcal{V}$是数域$\mathbb{F}$上的有限维内积空间,$\mathbf{T}\in\mathcal{L}\left(\mathcal{V}\right)$是一个非负算符,则必存在唯一非负算符$\mathbf{N}\in\mathcal{L}\left(\mathcal{V}\right)$满足$\mathbf{T}=\mathbf{N}^2$。
\end{theorem}

以下定理则类似于:任一复数$z$可分解为$z=\rho e^{i\theta}$,其中$\rho$是非负实数,表示伸缩;$\left|e^{i\theta}\right|=1$,表示旋转。对于复数,这叫“极坐标分解”。对于线性算符,这叫极分解。

\begin{theorem}[极分解]\label{thm:II.2.37}
    设$\mathcal{V}$是数域$\mathbb{F}$上的有限维内积空间,$\mathbf{T}\in\mathcal{L}\left(\mathcal{V}\right)$是一个线性算符,则总存在幺正算符$\mathbf{Q}$和唯一一个非负算符$\mathbf{N}$满足$\mathbf{T}=\mathbf{QN}$,称为$\mathbf{T}$的一个\emph{极分解(polar decomposition)}。如果$\mathbf{T}$可逆,则连$\mathbf{Q}$也是唯一的。如果$\mathbf{T}$是正规算符,则$\mathbf{QN}=\mathbf{NQ}$。
\end{theorem}

值得注意的是,对于非正规算符$\mathbf{T}$,可极分解为$\mathbf{T}=\mathbf{N}_1\mathbf{Q}=\mathbf{QN}_2$,且一般地$\mathbf{N}_1\neq\mathbf{N}_2$。

在连续介质力学部分,讲到形变梯度张量的时候,就需要用到极分解定理。
\end{document}