\documentclass[main.tex]{subfiles}
% 正规算符及其谱分解
\begin{document}
\begin{definition}[正规算符]
设$\mathcal{V}$是数域$\mathbb{F}$上的内积空间,$\mathbf{T}\in\mathcal{L}\left(\mathcal{V}\right)$是一个线性算符,如果$\mathbf{TT}^*=\mathbf{T}^*\mathbf{T}$,则称$\mathbf{T}$是正规算符(normal operator)。
\end{definition}

自伴随算符和幺正算符都是正规算符。正规算符的特征值分解有十分方便的性质,见如下定理。

\begin{theorem}
设$\mathcal{V}$是数域$\mathbb{F}$上的内积空间,$\mathbf{T}\in\mathcal{L}\left(\mathcal{V}\right)$是一个正规算符,则$\mathcal{V}$中必有一组规范正交基是$\mathbf{T}$的特征向量。
\end{theorem}
\begin{proof}
见附录。
\end{proof}

由此定理可知,正规算符必可对角化。此外,对于幺正算符$\mathbf{U}$,其行列式的绝对值为1:由$\mathbf{U}^*\mathbf{U}=\mathbf{I}$,$\mathrm{det}\left(\mathbf{UU}^*\right)=1=\mathrm{det}\mathbf{U}\mathrm{det}\left(\mathbf{U}^*\right)=\left(\mathrm{det}\mathbf{U}\right)^2$。

在内积空间上,我们可以把正规算符展开成一系列正交投影算符的线性组合,由此我们可以定义这类线性算符的初等函数,扩大线性算符的运算性质。我们先引入正交投影算符的概念。

\begin{definition}[最好近似]
设$\mathcal{V}$是数域$\mathbb{F}$上的赋范内积空间,$\mathcal{W}$是$\mathcal{V}$的子空间,如果对$\mathcal{V}$中的向量$\mathbf{b}\in\mathcal{V}$,有$\mathcal{W}$中的一个向量$\mathbf{a}\in\mathcal{W}$满足$\left\|\mathbf{b}-\mathbf{a}\right\|\leq\left\|\mathbf{b}-\mathbf{c}\right\|\forall\mathbf{c}\mathcal{W}$,则称$\mathbf{a}$是$\mathbf{b}$在$\mathcal{W}$中的最好近似(best approximation)。
\end{definition}

\begin{theorem}
设$\mathcal{V}$是数域$\mathbb{F}$上的赋范内积空间,$\mathcal{W}$是$\mathcal{V}$的子空间,则
\begin{enumerate}
    \item $\mathbf{a}\in\mathcal{W}$是$\mathbf{b}\in\mathcal{V}$在$\mathcal{W}$中的最好近似$\Leftrightarrow\mathbf{b}-\mathbf{a}$与$\mathcal{W}$中所有向量都正交
    \item 若$\mathbf{a}\in\mathcal{W}$是$\mathbf{b}\in\mathcal{V}$的一个最好近似,则$\mathbf{a}$是唯一的
    \item 若$\mathcal{W}$是有限维的,且$\left\{\mathbf{e}_k\right\}$是$\mathcal{W}$的一组正交基,则向量
    \[\mathbf{a}=\sum_k\frac{\left(\mathbf{b}|\mathbf{e}_k\right)}{\left(\mathbf{e}_k|\mathbf{e}_k\right)}\mathbf{e}_k\]
    是$\mathbf{b}$在$\mathcal{W}$的(唯一)一个最好近似,$\forall\mathbf{b}\in\mathcal{V}$。
\end{enumerate}
\end{theorem}

上面的定理解决了存在唯一性问题,使得我们可以进一步把“最好近似”定义为“正交投影”。

\begin{definition}[正交投影算符]
设$\mathcal{V}$是数域$\mathbb{F}$上的赋范内积空间,$\mathcal{W}$是$\mathcal{V}$的子空间,若$\mathbf{b}\in\mathcal{V}$在$\mathcal{W}$中有最好近似$\mathbf{a}\in\mathcal{W}$,则称$\mathbf{a}$是$\mathbf{b}$在$\mathcal{W}$中的正交投影(orthogonal projection)。由$\mathcal{V}$到$\mathcal{W}$的映射$\mathbf{E}:\mathcal{V}\rightarrow\mathcal{W},\left(\mathbf{Eb}-\mathbf{b}|\mathbf{b}\right)=0\forall\mathbf{b}\in\mathcal{V}$称向量空间$\mathcal{V}$到其子空间$\mathcal{W}$的正交投影。
\end{definition}

我们可证明映射$\mathbf{E}$是一个线性算符,而且是幂等的,即$\mathbf{E}^n=\mathbf{E}$。

以下定理给出正规算符的谱分解。

\begin{theorem}
设$\mathcal{V}$是数域$\mathbb{F}$上的有限维内积空间,$\mathbf{T}\in\mathcal{L}\left(\mathcal{V}\right)$是一个正规算符,$\theta_1,\cdots,\theta_k$为$\mathbf{T}$的两两不同特征值,分别对应特征空间$\mathcal{W}_1,\cdots,\mathcal{W}_k$,记$\mathbf{E}_i$为$\mathcal{V}$到$\mathcal{W}_i$的正交投影,则$\mathbf{T}=\theta_1\mathbf{E}_1+\cdots+\theta_k\mathbf{E}_k$,称$\mathbf{T}$的一个谱分解(spectral decomposition)
\end{theorem}

\begin{corollary}
设表达式$e_j\left(x\right)\equiv\prod_{i}\left(\frac{x-\theta_i}{\theta_j-\theta_i}\right)$,则该表达式应用于正规算符就得到其谱分解的各个正交投影算符,$\mathbf{E}_j=e_j\left(\mathbf{T}\right)$。
\end{corollary}

上面的定理和推论分别定义了一个正规算符的谱分解形式,以及具体获得其中的正交投应算符的计算公式。值得注意的是,正规算符的谱分解一般没有唯一性。有了正规算符的一个谱分解,我们证明对任一初等表达式$f\left(x\right)$,$f\left(\mathbf{T}\right)=\sum_{i=1}^kf\left(\theta_i\right)\mathbf{E}_i$,可视为“正规算符的函数”的计算定义。

针对正规算符的特征值性质,我们还可以由如下定理进一步区分自伴随算符、幺正算符和非负算符。

\begin{theorem}
设$\mathcal{V}$是数域$\mathbb{F}$上的$n$维内积空间,$\mathbf{T}\in\mathcal{L}\left(\mathcal{V}\right)$是一个正规算符,$\left\{\lambda_i\right\}$是$\mathbf{T}$的特征值,则
\begin{enumerate}
    \item $\lambda_i\in\mathbb{R}\forall i=1,\cdots,n\Leftrightarrow\mathbf{T}$是厄米算符
    \item $\overline{\lambda_i}\in\mathbb{R}\forall i=1,\cdots,n\Leftrightarrow\mathbf{T}$是反厄米算符。
    \item $\lambda_i\geq 0\forall i=1,\cdots,n\Leftrightarrow\left(\mathbf{Ta}|\mathbf{a}\right)\geq0\forall\mathbf{a}\in\mathcal{V}$,称为非负算符(non-negative operator)
    \item $\left|\lambda_i\right|=1\forall i=1,\cdots,n\Leftrightarrow\mathbf{T}$是幺正算符
\end{enumerate}
\end{theorem}

以下性质类似“正实数的平方根是非负实数”。

\begin{theorem}
设$\mathcal{V}$是数域$\mathbb{F}$上的有限维内积空间,$\mathbf{T}\in\mathcal{L}\left(\mathcal{V}\right)$是一个正规算符,则必存在唯一非负算符$\mathbf{N}\in\mathcal{L}\left(\mathcal{V}\right)$满足$\mathbf{T}=\mathbf{N}^2$。
\end{theorem}

以下性质类似“复数$z$可分解为$z=\rho e^{i\theta}$”。

\begin{theorem}
设$\mathcal{V}$是数域$\mathbb{F}$上的有限维内积空间,$\mathbf{T}\in\mathcal{L}\left(\mathcal{V}\right)$是一个线性算符,则总存在幺正算符$\mathbf{U}$和唯一一个非负算符$\mathbf{N}$满足$\mathbf{T}=\mathbf{UN}$,称为$\mathbf{T}$的极分解(polar decomposition)。如果$\mathbf{T}$可逆,则连$\mathbf{U}$也是唯一的。如果$\mathbf{T}$是正规算符,则$\mathbf{UN}=\mathbf{NU}$。
\end{theorem}
\end{document}