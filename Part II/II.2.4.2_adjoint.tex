\documentclass[main.tex]{subfiles}
% 伴随算符
\begin{document}
按照内积的定义,我们有$\left(\mathbf{a}|\alpha\mathbf{b}\right)=\left(\overline{\alpha}\mathbf{a}|\mathbf{b}\right)$($\alpha\neq 0$)。如果把等号左边的标量$\alpha$换成一个非零线性算符$\mathbf{T}$,那么处于等号右边的“$\alpha$的复数共轭”这一角色的,又会是怎样的一个线性算符?我们先给这个线性算符一个名称,然后再分析它的特点。

\begin{definition}[伴随算符]\label{def:II.2.22}
    设$\mathcal{V}$是数域$\mathbb{F}$上的内积空间,$\mathbf{T}\in\mathcal{L}\left(\mathcal{V}\right)$是$\mathcal{V}$上的线性算符。若存在另一线性算符$\mathbf{T}^*\in\mathcal{L}\left(\mathcal{V}\right)$满足$\left(\mathbf{Ta}|\mathbf{b}\right)=\left(\mathbf{a}|\mathbf{T}^*\mathbf{b}\right),\forall\mathbf{a},\mathbf{b}\in\mathcal{V}$,则称$\mathbf{T}^*$是$\mathbf{T}$的\emph{伴随算符(adjoint operator)}。若$\mathbf{T}=\mathbf{T}^*$,则称$\mathbf{T}$是一个\emph{自伴随(self-adjoint)}或\emph{厄米(hermitian)}算符。若$\mathbf{T}=-\mathbf{T}^*$,则称$\mathbf{T}$是一个\emph{反厄米(skew hermitian)}算符。
\end{definition}

在有限维内积空间上,每一个线性算符有且只有一个伴随算符(证明见附录\S\ref{sec:VI.1})。

易验,若$\mathbf{T}$是厄米算符,则$i\mathbf{T}$就是反厄米算符。

以下定理列举了一些伴随算符的运算规律。

\begin{theorem}\label{thm:II.2.30}
    设$\mathcal{V}$是数域$\mathbb{F}$上的有限维内积空间,
    \begin{enumerate}
        \item $\left(\mathbf{T}+\mathbf{U}\right)^*=\mathbf{T}^*+\mathbf{U}^*,\forall\mathbf{T},\mathbf{U}\in\mathcal{L}\left(\mathcal{V}\right)$
        \item $\left(\alpha\mathbf{T}\right)^*=\overline{\alpha}\mathbf{T}^*,\forall\mathbf{T}\in\mathcal{L}\left(\mathcal{V}\right),\alpha\in\mathbb{F}$
        \item $\left(\mathbf{TU}\right)^*=\mathbf{U}^*\mathbf{T}^*,\forall\mathbf{T},\mathbf{U}\in\mathcal{L}\left(\mathcal{V}\right)$
        \item $\left(\mathbf{T}^*\right)^*=\mathbf{T},\forall\mathbf{T}\in\mathcal{L}\left(\mathcal{V}\right)$
    \end{enumerate}
\end{theorem}
\begin{proof}
    利用相关定义易证,留作练习。
\end{proof}

从上面的运算规律可以看出,线性算符的伴随与复数的共轭有些类似。例如在复数域$\mathbb{C}$上的内积空间$\mathcal{V}$上,任一线性算符$\mathbf{T}$都可以写成“虚部与实部”的形式,即$\mathbf{T}=\mathbf{U}_1+i\mathbf{U_2}$,其中$\mathbf{U}_1=\frac{1}{2}\left(\mathbf{T}+\mathbf{T}^*\right),\mathbf{U}_2=\frac{1}{2i}\left(\mathbf{T}-\mathbf{T}^*\right)$都是自伴随算符。注意到,$i\mathbf{U}_2$是反厄米算符,故我们也常说任一线性算符$\mathbf{T}$都可以分解成一个厄米算符和一个反厄米算符。

一对伴随算符在给定基下的坐标矩阵之间的关系是矩阵的共轭转置。以下我们给出证明。从证明的过程可以注意到,内积空间上的向量和线性算符如何通过内积来取得它们在给定有序基下的坐标。

\begin{theorem}\label{thm:II.2.31}
    设$\mathcal{V}$是数域$\mathbb{F}$上的$n$维内积空间,$B=\left\{\mathbf{\hat{e}}_i\right\}_{i=1}^n$是$\mathcal{V}$的一组规范正交基,$\mathcal{V}$上的线性算符$\mathbf{T}$及其伴随算符$\mathbf{T}^*$在有序基$B$下的坐标分别是$T_{ij},T^\prime_{ij},i,j=1,\cdots,n$,则$T_{ij}=\overline{T^\prime_{ji}}$。
\end{theorem}
\begin{proof}
    设$\mathbf{a}\in\mathcal{V}$是$\mathcal{V}$中的任一向量,则$\mathbf{a}=\sum_{i=1}^n\alpha_i\mathbf{\hat{e}}_i$,其中$\alpha_i,i=1,\cdots,n$是$\mathbf{a}$在有序基$B$下的坐标,则$\alpha_i=\left(\mathbf{a}|\mathbf{\hat{e}}_i\right)$,因为
    \begin{align*}
        \left(\mathbf{a}|\mathbf{\hat{e}}_i\right) & =\left(\sum_{j=1}^n\alpha_j\mathbf{\hat{e}}_j|\mathbf{\hat{e}}_i\right)=\sum_{j=1}^n\alpha_j\left(\mathbf{\hat{e}}_i|\mathbf{\hat{e}}_j\right)
        \\&=\sum_{j=1}^n\alpha_j\delta_{ij}=\alpha_i
    \end{align*}

    由\S\ref{sec:II.2.2.2}可知,线性算符$\mathbf{T}$在给定有序基$B$下的坐标$T_{ij}$满足$\mathbf{T\hat{e}}_i=\sum_{j=1}^nT_{ji}\mathbf{\hat{e}}_j,i=1,\cdots,n$。同时每个$\mathbf{T\hat{e}}_i$作为一个向量在有序基$B$下的坐标满足上面刚刚证明结论,故$\mathbf{T\hat{e}}_i=\sum_{j=1}^n\left(\mathbf{T\hat{e}}_i|\mathbf{\hat{e}}_j\right)\mathbf{\hat{e}}_j$。由于$\left\{\mathbf{\hat{e}}_j\right\}$线性无关,当$\mathbf{T}\neq\mathbf{0}$时比较上述两结果可得$T_{ij}=\left(\mathbf{T\hat{e}}_j|\mathbf{\hat{e}}_i\right)$。

    由伴随算符定义,
    \begin{align*}
        T_{ji}=\left(\mathbf{T\hat{e}}_i|\mathbf{\hat{e}}_j\right) & =\left(\mathbf{\hat{e}}_i|\mathbf{T}^*\mathbf{\hat{e}}_j\right)            \\
                                                                   & =\overline{\left(\mathbf{T}^*\mathbf{\hat{e}}_j|\mathbf{\hat{e}}_i\right)} \\
                                                                   & =\overline{T^\prime_{ij}}
    \end{align*}
\end{proof}

注意,这一定理讨论的是厄米算符仅限于在规范正交基下的坐标规律。在一般有序基下的坐标规律表达式将额外含有该组基的格拉姆矩阵分量。

回顾线性变换的转置(\S\ref{sec:II.2.2.3}),如果由$\mathcal{V}$的每个向量$\mathbf{a}\in\mathcal{V}$定义一个相应的线性泛函来实现内积,即$f_\mathbf{a}\in\mathcal{V}^*,f_\mathbf{a}\left(\mathbf{b}\right)\equiv\left(\mathbf{a}|\mathbf{b}\right)$,则$\mathcal{V}$上的任一线性算符$\mathbf{T}\in\mathcal{L}\left(\mathcal{V}\right)$与其转置$\mathbf{T}^\intercal\in\mathcal{L}\left(\mathcal{V}^*\right)$将满足
\[
    \left(\mathbf{T}^\intercal f_\mathbf{a}\right)\left(\mathbf{b}\right)=\left(\mathbf{a}|\mathbf{Tb}\right),\forall\mathbf{a},\mathbf{b}\in\mathcal{V}
\]

同时$\mathbf{T}$与$\mathbf{T}^\intercal$在给定一对基$B\subset\mathcal{V}$与对偶基$B^*\subset\mathcal{V}^*$下的坐标矩阵互为矩阵的转置。而线性算符$\mathbf{T}$与其伴随算符$\mathbf{T}^*$满足
\[
    \left(\mathbf{Ta}|\mathbf{b}\right)=\left(\mathbf{a}|\mathbf{T}^*\mathbf{b}\right),\forall\mathbf{a},\mathbf{b}\in\mathcal{V}
\]
且$\mathbf{T}$与$\mathbf{T}^*$在给定基$B$下的坐标矩阵之间互为矩阵的共轭转置。比较可发现,线性算符的转置与伴随似乎仅存在“是否需要共轭”的差别,这个差别来自于它们扮演的角色是放在在内积的第一个向量前面还是第二个向量前面。本讲义的内积定义规定对第二向量具有共轭线性,所以坐标矩阵需要共轭的是定义在内积的第二个向量的伴随算符。

但是,线性变换的转置和伴随概念根本不同。因为$\mathbf{T}^\intercal\in\mathcal{L}\left(\mathcal{V}^*\right)$而$\mathbf{T}^*\in\mathcal{L}\left(\mathcal{V}\right)$,即它们属于不同的空间,作用于不同空间中的对象。

在实数域中(经典力学语境下),线性算符的伴随的概念不如在复数域中(量子力学语境下)重要。在本讲义后续内容中凡涉及到实数域上的线性算符,都暂不区分其转置和伴随,而均写成$\mathbf{T}^\intercal$。

此外,在以往所学习的矩阵代数中,有“对称矩阵”的概念。如果数域$\mathbb{F}$上的$n\times n$矩阵$A\in\mathbb{F}^{n\times n}$满足$A=A^\intercal$,则称矩阵$A$是\emph{对称矩阵(symmetric matrix)};若$A=-A^\intercal$,则称矩阵$A$是\emph{斜称矩阵(skew-symmetric matrix)}。由厄米与反厄米算符的定义可知,只有在实数域$\mathbb{R}$上的内积空间上,厄米和反厄米算符在给定基于的坐标矩阵才是对称和斜称矩阵。因此,我们又把\emph{实数域上的}厄米和反厄米算符称为\emph{对称算符(symmetric operator)}和\emph{斜称算符(skew-symmetrix operator)}。

最后我们关心一下厄米和反厄米算符在特征值分解时的性质\footnote{在大一的线性代数课中只介绍了考虑实数域上的对称矩阵的对角化\cite[\S 5.3 定理3.4~3.6、例3.4]{周胜林2012线性代数},得到一系列性质。但是这些性质需要在复数域的更一般情况下得到确认。},归入以下定理——

\begin{theorem}\label{thm:II.2.32}
    设$\mathcal{V}$是数域$\mathbb{F}$上的$n$维内积空间,$\mathbf{T}$是$\mathcal{V}$上的一个线性算符,则
    \begin{enumerate}
        \item $\mathbf{T}$是厄米算符$\Rightarrow\mathbf{T}$的所有特征值都是实数
        \item $\mathbf{T}$是反厄米算符$\Rightarrow\mathbf{T}$的所有特征值都是纯虚数或0。
        \item $\mathbf{T}$是厄米算符$\Rightarrow$对应$\mathbf{T}$的不同特征值的特征向量间必正交
        \item $\mathbf{T}$是厄米算符$\Rightarrow\mathcal{V}$中存在一组规范正交基恰为$\mathbf{T}$的特征向量
    \end{enumerate}
\end{theorem}

注意这些命题的逆命题都不成立(反例可作为练习)。关于第3条,在特征值的章节中曾提到,在复数域上,对于一般向量空间上的线性算符,对应于不同特征值的特征向量\emph{线性无关}。如果这一线性算符是内积空间上的自伴随算符,那么进一步地,对应于不同特征值的特征向量是\emph{正交}的。第4条再加上定理\ref{thm:II.2.28}可知,\emph{可逆厄米算符必可对角化}。
\end{document}