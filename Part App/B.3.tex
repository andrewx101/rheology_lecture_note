\documentclass[main.tex]{subfiles}
% VI.2.3 复合函数求导的链式法则
\begin{document}
\begin{theorem*}
    定理\ref{thm:II.4.11}:如果函数$\mathbf{f}:\mathbb{R}^n\supseteq D\rightarrow\mathbb{R}^m$在$\mathbf{x}_0\in D$处可微分;函数$\mathbf{g}:\mathbb{R}^n\supset E\rightarrow\mathbb{R}^p$在$\mathbf{f}\left(\mathbf{x}_0\right)\in E\cap D$处可微分,则复合函数$\mathbf{g}\circ\mathbf{f}$在$\mathbf{x}_0$处可微分,且其导数
    %\[
    %\left.\frac{d\mathbf{g}\circ\mathbf{f}\left(\mathbf{x}\right)}{d\mathbf{x}}\right|_{\mathbf{x}=\mathbf{x}_0}=\left.\frac{d\mathbf{g}\left(\mathbf{y}\right)}{d\mathbf{y}}\right|_{\mathbf{y}=\mathbf{f}\left(\mathbf{x}_0\right)}\left.\frac{d\mathbf{f}\left(\mathbf{x}\right)}{d\mathbf{x}}\right|_{\mathbf{x}=\mathbf{x}_0}
    %\]
    \[\mathbf{d}_{\mathbf{x}=\mathbf{x}_0}\mathbf{g}\circ\mathbf{f}\left(\mathbf{x}\right)=\mathrm{d}_{\mathbf{y=\mathbf{f}\left(\mathbf{x}_0\right)}}\mathbf{g}\left(\mathbf{y}\right)\mathbf{d}_{\mathbf{x}=\mathbf{x}_0}\mathbf{f}\left(\mathbf{x}\right)\]
\end{theorem*}
\begin{proof}
    首先证明$\mathbf{x}_0$处于复合函数$\mathbf{g}\circ\mathbf{f}$的定义域内。由于$\mathbf{f}\left(\mathbf{x}_0\right)\in\mathrm{dom}\mathbf{g}$且$\mathbf{g}$在$\mathbf{f}\left(\mathbf{x}_0\right)$处可微分,故总存在正实数$\delta^\prime$使得只要$\left\|\mathbf{f}\left(\mathbf{x}\right)-\mathbf{f}\left(\mathbf{x}_0\right)\right\|<\delta^\prime$就有$\mathbf{f}\left(\mathbf{x}\right)\in\mathrm{dom}\mathbf{g}$。又因为$\mathbf{x}_0\in\mathrm{dom}\mathbf{f}$且$\mathbf{f}$在$\mathbf{x}_0$处可微分,故总存在正实数$\delta$使得只要$\left\|\mathbf{x}-\mathbf{x}_0\right\|<\delta$则$\mathbf{x}\in\mathrm{dom}\mathbf{f}$,同时还必存在$\delta^\prime>0$使得这一$\delta$选择下的$\mathbf{x}$满足$\left\|\mathbf{f}\left(\mathbf{x}\right)-\mathbf{f}\left(\mathbf{x}_0\right)\right\|<\delta^\prime$。所以任一满足$\left\|\mathbf{x}-\mathbf{x}_0\right\|<\delta$的$\mathbf{x}$均在复合函数$\mathbf{g}\circ\mathbf{f}$的定义域内。

    按照全微分和全导数的定义,由于函数$\mathbf{f}$和$\mathbf{g}$分别在$\mathbf{x}_0$和$\mathbf{f}\left(\mathbf{x}_0\right)$可导,故存在函数$\mathbf{z}_1$、$\mathbf{z}_2$满足$\lim_{\mathbf{x}\to\mathbf{x}_0}\mathbf{z}_1\left(\mathbf{x}-\mathbf{x}_0\right)=\mathbf{0}$、$\lim_{\mathbf{f}\left(\mathbf{x}\right)\to\mathbf{f}\left(\mathbf{x}_0\right)}\mathbf{z}_2\left(\mathbf{f}\left(\mathbf{x}\right)-\mathbf{f}\left(\mathbf{x}_0\right)\right)=\mathbf{0}$,且
    \begin{align*}
        \mathbf{f}\left(\mathbf{x}\right)-\mathbf{f}\left(\mathbf{x}_0\right)-\mathrm{d}_{\mathbf{x}=\mathbf{x}_0}\mathbf{f}\left(\mathbf{x}\right)\left(\mathbf{x}-\mathbf{x}_0\right)                                                                                                                    & =\left\|\mathbf{x}-\mathbf{x}_0\right\|\mathbf{z}_1\left(\mathbf{x}-\mathbf{x}_0\right)                                                                                             \\
        \mathbf{g}\left(\mathbf{f}\left(\mathbf{x}\right)\right)-\mathbf{g}\left(\mathbf{f}\left(\mathbf{x}_0\right)\right)-\mathrm{d}_{\mathbf{y}=\mathbf{f}\left(\mathbf{x}_0\right)}\mathbf{g}\left(\mathbf{y}\right)\left(\mathbf{f}\left(\mathbf{x}\right)-\mathbf{f}\left(\mathbf{x}_0\right)\right) & =\left\|\mathbf{f}\left(\mathbf{x}\right)-\mathbf{f}\left(\mathbf{x}_0\right)\right\|\mathbf{z}_2\left(\mathbf{f}\left(\mathbf{x}\right)-\mathbf{f}\left(\mathbf{x}_0\right)\right)
    \end{align*}
    把$\mathbf{g}\left(\mathbf{f}\left(\mathbf{x}\right)\right)$记为$\mathbf{g}\circ\mathbf{f}\left(\mathbf{x}\right)$,并把上面的第一条式子代入第二条,得
    \begin{align*}
                        & \mathbf{g}\circ\mathbf{f}\left(\mathbf{x}\right)-\mathbf{g}\circ\mathbf{f}\left(\mathbf{x}_0\right)-\mathrm{d}_{\mathbf{y}=\mathbf{f}\left(\mathbf{x}_0\right)}\mathbf{g}\left(\mathbf{y}\right)\left[\mathrm{d}_{\mathbf{x}=\mathbf{x}_0}\mathbf{f}\left(\mathbf{x}\right)\left(\mathbf{x}-\mathbf{x}_0\right)+\left\|\mathbf{x}-\mathbf{x}_0\right\|\mathbf{z}_1\left(\mathbf{x}-\mathbf{x}_0\right)\right] \\
        =               & \left\|\mathbf{f}\left(\mathbf{x}\right)-\mathbf{f}\left(\mathbf{x}_0\right)\right\|\mathbf{z}_2\left(\mathbf{f}\left(\mathbf{x}\right)-\mathbf{f}\left(\mathbf{x}_0\right)\right)                                                                                                                                                                                                                            \\
        \Leftrightarrow & \mathbf{g}\circ\mathbf{f}\left(\mathbf{x}\right)-\mathbf{g}\circ\mathbf{f}\left(\mathbf{x}_0\right)-\mathrm{d}_{\mathbf{y}=\mathbf{f}\left(\mathbf{x}_0\right)}\mathbf{g}\left(\mathbf{y}\right)\mathrm{d}_{\mathbf{x}=\mathbf{x}_0}\mathbf{f}\left(\mathbf{x}\right)\left(\mathbf{x}-\mathbf{x}_0\right)                                                                                                     \\
        =               & \left\|\mathbf{x}-\mathbf{x}_0\right\|\mathrm{d}_{\mathbf{y}=\mathbf{f}\left(\mathbf{x}_0\right)}\mathbf{g}\left(\mathbf{y}\right)\mathbf{z}_1\left(\mathbf{x}-\mathbf{x}_0\right)+\left\|\mathbf{f}\left(\mathbf{x}\right)-\mathbf{f}\left(\mathbf{x}_0\right)\right\|\mathbf{z}_2\left(\mathbf{f}\left(\mathbf{x}\right)-\mathbf{f}\left(\mathbf{x}_0\right)\right)
    \end{align*}
    由三角不等式,又有\footnote{
        此处用到定理\ref{thm:II.4.3}。
    }
    \begin{align*}
        \left\|\mathbf{f}\left(\mathbf{x}\right)-\mathbf{f}\left(\mathbf{x}_0\right)\right\| & =\left\|\mathrm{d}_{\mathbf{x}=\mathbf{x}_0}\mathbf{f}\left(\mathbf{x}\right)\left(\mathbf{x}-\mathbf{x}_0\right)+\left\|\mathbf{x}-\mathbf{x}_0\right\|\mathbf{z}_1\left(\mathbf{x}-\mathbf{x}_0\right)\right\|                    \\
                                                                                             & \leq \left\|\mathrm{d}_{\mathbf{x}=\mathbf{x}_0}\mathbf{f}\left(\mathbf{x}\right)\left(\mathbf{x}-\mathbf{x}_0\right)\right\|+\left\|\mathbf{x}-\mathbf{x}_0\right\|\left\|\mathbf{z}_1\left(\mathbf{x}-\mathbf{x}_0\right)\right\| \\
                                                                                             & \leq k\left\|\mathbf{x}-\mathbf{x}_0\right\|+\left\|\mathbf{x}-\mathbf{x}_0\right\|\left\|\mathbf{z}_1\left(\mathbf{x}-\mathbf{x}_0\right)\right\|
    \end{align*}
    故
    \begin{align*}
             & \mathbf{g}\circ\mathbf{f}\left(\mathbf{x}\right)-\mathbf{g}\circ\mathbf{f}\left(\mathbf{x}_0\right)-\mathrm{d}_{\mathbf{y}=\mathbf{f}\left(\mathbf{x}_0\right)}\mathbf{g}\left(\mathbf{y}\right)\mathrm{d}_{\mathbf{x}=\mathbf{x}_0}\mathbf{f}\left(\mathbf{x}\right)\left(\mathbf{x}-\mathbf{x}_0\right)                                                                                                                                   \\
        \leq & \left\|\mathbf{x}-\mathbf{x}_0\right\|\mathrm{d}_{\mathbf{y}=\mathbf{f}\left(\mathbf{x}_0\right)}\mathbf{g}\left(\mathbf{y}\right)\mathbf{z}_1\left(\mathbf{x}-\mathbf{x}_0\right)+\left(k\left\|\mathbf{x}-\mathbf{x}_0\right\|+\left\|\mathbf{x}-\mathbf{x}_0\right\|\left\|\mathbf{z}_1\left(\mathbf{x}-\mathbf{x}_0\right)\right\|\right)\mathbf{z}_2\left(\mathbf{f}\left(\mathbf{x}\right)-\mathbf{f}\left(\mathbf{x}_0\right)\right) \\
        \leq & \left\|\mathbf{x}-\mathbf{x}_0\right\|\left\{\left\|\mathrm{d}_{\mathbf{y}=\mathbf{f}\left(\mathbf{x}_0\right)}\mathbf{g}\left(\mathbf{y}\right)\mathbf{z}_1\left(\mathbf{x}-\mathbf{x}_0\right)\right\|+\left(k+\left\|\mathbf{z}_1\left(\mathbf{x}-\mathbf{x}_0\right)\right\|\right)\mathbf{z}_2\left(\mathbf{f}\left(\mathbf{x}\right)-\mathbf{f}\left(\mathbf{x}_0\right)\right)\right\}
    \end{align*}
    由于函数$\mathbf{f}$在$\mathbf{x}_0$处连续,即极限$\lim_{\mathbf{x}\to\mathbf{x}_0}\mathbf{f}\left(\mathbf{x}\right)=\mathbf{f}\left(\mathbf{x}_0\right)$,故上式最后的大括号在$\mathbf{x}\to\mathbf{x}_0$时趋于$\mathbf{0}$。按照全微分和全导数的定义,命题得证。
\end{proof}
\end{document}