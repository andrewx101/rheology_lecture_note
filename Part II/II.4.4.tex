\documentclass[../main.tex]{subfiles}
\begin{document}
我们在本科的高等数学课上已经学过二重积分和三重积分\cite[\S 8]{华工高数2009下}。在这里我们它的定义推广至$n$维。在正式定义之前,先要明确一系列概念和术语。

考虑$\mathbb{R}^n$上的“矩形”区域,它具体是$\mathbb{R}^n$的一个子集$R$,任一$\left(x_1,\cdots,x_n\right)^\intercal\in R$满足$a_i\leq x_i\leq b_i$,其中$a_i,b_i\in\mathbb{R}$,$i=1,\cdots,n$。我们称这个区域$R$是一个\emph{$n$维矩形闭区域}。如果所有的$\leq$号都换成$<$号,则称$R$是一个\emph{$n$维矩形开区域}。$R$的体积记作$V\left(R\right)$,
\[V\left(R\right)=\prod_{i=1}^n\left(b_i-a_i\right)\]
我们知道当$n=2$时$V\left(R\right)$称为面积,当$n=1$时$V\left(R\right)$称为长度。如果对于某一$i$,$a_i=b_i$,则称$V\left(R\right)=0$,我们称$R$是\emph{退化(degenerate)的}。

若$\mathbb{R}^n$的子集$B$满足:$\exists k\in\mathbb{R},k>0,\left\|\mathbf{x}\right\|<k,\forall\mathbf{x}\in B$,则称$B$是$\mathbb{R}^n$的\emph{有界(bounded)}子集。

$\mathbb{R}^n$中的若干个$\left(n-1\right)$维矩形区域可被称为一套\emph{网格(grid)}。任一套网格可把$\mathbb{R}^n$\emph{分割}成有限个有界闭矩形区域$R_1,\cdots,R_r$和有限个无界区域$R_{r+1},\cdots,R_s$。我们称这个分割是\emph{有限的}。若某区域$B\subset\mathbb{R}^n$在闭合区域$R_1,\cdots,R_r$的并集之内,由称$B$被这套网格\emph{覆盖(covered)}。显然$B$必须有界才可以被一个有限分割$\mathbb{R}^n$的网格所覆盖。我们称一套网格的所有闭合有界区域$R_1,\cdots,R_r$的边长的最大值$\lambda$称为这套网格的\emph{网眼(mesh)}。

现在我们已经做好定义积分准备。

\begin{definition}[$n$重积分的黎曼定义]\label{def:II.4.20}
    设函数$f:\mathbb{R}^n\supset D\rightarrow\mathbb{R}$,$B\subset D$且$B$是有界闭区域,$f$在$B$上有界。由函数$f$可构造一个判断某点$\mathbf{x}\in\mathbb{R}$是否在$B$内的特征函数$\chi_B\left(\mathbf{x}\right)$,即
    \[
        \chi_B\left(\mathbf{x}\right)=\left\{\begin{array}{ll}1,&\mathbf{x}\in B\\0,&\mathbf{x}\notin B\end{array}\right.
    \]
    令$G$是一套覆盖$B$的网格,$R_1,\cdots,R_r$是$G$的闭合有界区域,$\lambda$是$G$的网眼。在每个有界闭区域$R_i$上取一点$\mathbf{x}_i$,则以下求和
    \[\sum_{i=1}^r \chi_B\left(\mathbf{x}_i\right)V\left(R_i\right)
    \]
    称为函数$f$在区域$B$上的一个\emph{黎曼和(Riemann sum)}。它的值依赖网格$G$和点$\mathbf{x}_i$的选取。如果无论我们如何选取网格$G$和点$\mathbf{x}_i$,当$\lambda\rightarrow 0$时,黎曼和的极限
    \[\lim_{\lambda\to 0}\sum_{i=1}^r \chi_B\left(\mathbf{x}_i\right)V\left(R_i\right)\]
    都存在,且其值与网格$G$和点$\mathbf{x}_i$的选取无关而唯一确定,我们就把该极限值称为\emph{函数$f$在区域$B$上的积分(integral of $f$ on region $B$)},记作
    \[\int_B f\left(\mathbf{x}\right)\mathrm{d}\mathbf{x}\]
    并称函数$f$\emph{在区域$B$上是可积的(integrable on region $B$)}。
\end{definition}

定义只负责告诉我们,如果一个东西存在,那么如何称呼它。我们还需要证明这个东西的存在性。具体到函数积分,我们需要证明的是黎曼积分的存在定理。

\begin{theorem}[黎曼积分的存在定理]\label{thm:II.4.14}
    设函数$f:\mathbb{R}^n\supset D\rightarrow\mathbb{R}$在有界区域$B\subset D$上有界,且$B$被有限多个光滑曲面覆盖。如果在$B$上,函数$f$除在有限多个光滑集上之外,是连续的,则$f$在$B$上是可积的,且函数$f$在$B$上的积分值不随光滑曲面的选取而改变。
\end{theorem}
\begin{proof}
    见其他资料\cite[Appendix 8]{Williamson1972}。
\end{proof}

这个定理的证明是很长的。在本科的《高等数学》课本中,这个证明往往是从略的\footnote{例如定理8.1.1和定义8.3.1下面的“可以证明,……”\cite[\S 8]{华工高数2009下}。}。在这里我们只需理解定理的通俗意思。这个定理告诉我们,只要函数$f$在其定义域的某个子区域$B$上没有无穷大或无穷小的取值(有界),而这个区域$B$的边界是“分段光滑”的(被包含在有限多个光滑曲面上),那么$f$在$B$上是可积的。

拿定义\ref{def:II.4.20}与本科《高等数学》中的二重、三重积分的定义相对比,就会发现它们很相似。这是因为定义\ref{def:II.4.20}是对二重、三重积分的推广。《高等数学》中的重积分的所有性质,都能直接推广到$n$重积分上。这里不再赘述。在这里我们只需要简单说明什么是向量函数的积分。若$\mathbf{f}:\mathbb{R}^n\supset D\rightarrow\mathbb{R}^m$的坐标函数是$f_1,\cdots,f_m$,区域$B\subset D$,且对每个$i=1,\cdots,m$,$f_i$在$B$上是可积的,则$\mathbf{f}$在$B$上的积分定义为
\[\int_B\mathbf{f}\left(\mathbf{x}\right)\mathrm{d}\mathbf{x}=\left(\int_B f_1\left(\mathbf{x}\right)\mathrm{d}\mathbf{x},\cdots,\int_B f_m\left(\mathbf{x}\right)\mathrm{d}\mathbf{x}\right)^\intercal\]
\end{document}