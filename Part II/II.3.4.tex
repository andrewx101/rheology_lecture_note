\documentclass[main.tex]{subfiles}
\begin{document}
下面我们介绍等距变换的一个重要定理:等距变换的表示定理。这个定理大致说:任一等距变换都是一个平移变换加一个旋转变换。这个定理也是后面介绍物理定律的标架变换不变性时的理论基础。

\begin{theorem}[等距变换的表示定理]\label{thm:II.3.2}
    设$\mathcal{E}$是一个欧几里得空间,$\mathcal{V}$是其平移空间,选定任一点$X_0\in\mathcal{E}$,则$\mathcal{E}$上的任一等距变换$i:\mathcal{E}\rightarrow\mathcal{E},i\in\mathcal{I}$都可表示为
    \[
        i\left(X\right)=i\left(X_0\right)+\mathbf{Q}_i\left(X-X_0\right)
    \]
    其中$\mathbf{Q}_i$是一个正交算符,关于$i$唯一存在。
\end{theorem}
\begin{proof}
    见附录。
\end{proof}

\begin{corollary}
    欧几里得空间上的等距变换都是双射。
\end{corollary}
\begin{proof}
    定理\ref{thm:II.3.1}已经暗示欧几里得空间上的等距变换都是单射。故仅需再证明对任一$i:
        \mathcal{E}\rightarrow\mathcal{E}$和$Y\in\mathcal{E}$总存在一个$X\in\mathcal{E}$满足$i\left(X\right)=Y$。我们可直接找出这样的$X$:
    \[
        X=X_0+\mathbf{Q}^{-1}\left(Y-i\left(X_0\right)\right)
    \]
    验证这就是满足条件的$X$:
    \begin{align*}
        i\left(X\right) & =i\left(X_0\right)+\mathbf{Q}\left(\mathbf{X}_0+\mathbf{Q}^{-1}\left(Y-i\left(X_0\right)\right)-X_0\right) \\
                        & =i\left(X_0\right)+Y-i\left(X_0\right)                                                                     \\
                        & =Y
    \end{align*}
\end{proof}

定理\ref{thm:II.3.2}的通俗解释:给定任一等距变换$i$,仅需知道它对某一参考点$X_0$的像是哪个点,以及该变换的特征正交算符$\mathbf{Q}_i$,就可以知道它对任意点$X$的像。因此这一定理给出了等距变换的通用表达式。

这里的等距变换$i$不一定是欧几里得空间$\mathcal{E}$的平移向量空间$\mathcal{V}$中的元素(前面提到过$\mathcal{V}$至多是$\mathcal{I}$的子群)。例如旋转和镜向反转都是等距变换,却不满足向量空间对向量的要求。

一般$i\left(X_0\right)$是容易找到的,但是$\mathbf{Q}_i$不是直接易得的。我们可以举例认识$\mathbf{Q}_i$的一般意义。

\begin{example}
    考虑欧几里得空间$\left(\mathcal{E},d\right)$上的以下等距变换,其中$\mathbf{Q}$是一个正交算符,$X_0,C$是$\mathcal{E}$中固定的点:
    \begin{align*}
        i_1\left(X\right) & =X+\left(C-X_0\right)                \\
        i_2\left(X\right) & =X_0+\mathbf{Q}\left(X-X_0\right)    \\
        i_3\left(X\right) & =X+\mathbf{Q}^{-1}\left(C-X_0\right)
    \end{align*}

    $i_1$把任一点向固定的方向平移固定距离($i_1\left(X\right)=X+\mathbf{u},\mathbf{u}\equiv C-X_0$)。

    $i_1\circ i_2=i_2\circ i_3$(自行验证作为练习。)

    当$\mathbf{Q}=\mathbf{I}$时,$i_2$是恒等映射。当$\mathbf{Q}\neq\mathbf{I}$时,由正交算符性质$\mathrm{det}\mathbf{Q}=\pm 1$。当$\mathrm{det}\mathbf{Q}=1$时,$i_2$是一种旋转操作;当$\mathrm{det}\mathbf{Q}=-1$时,由$\mathbf{Q}=\left(-\mathbf{I}\right)\left(-\mathbf{Q}\right)$和$\mathrm{det}\left(-\mathbf{Q}\right)=1$可知$i_2$是先进行了一个旋转($-\mathbf{Q}$)再进行了反转($-\mathbf{I}$)的操作。
\end{example}

在连续介质力学中,我们只考虑$\mathrm{det}\mathbf{Q}=1$的情况,即等距变换中的正交算符仅表旋转。在此限定下,定理\ref{thm:II.3.2}说的就是,欧几里得空间的任一等距变换(镜像除外)都是平移加旋转。
\end{document}