\documentclass[zihao=-4,linespread=1.5,a4paper,heading=true,twoside]{ctexbook}
%\usepackage[utf8]{inputenc}
\pagestyle{empty}
% packages for page layout
\usepackage{geometry}
\geometry{
a4paper,
%total={171.8mm,246.2mm},
%left=19.1mm,
%top=25.4mm,
}
\setlength{\headheight}{14.5pt}
% package for colors
\usepackage{xcolor}

% packages for fonts and symbols
\usepackage{amsmath}
\usepackage{amssymb}
\usepackage{bm}
\usepackage{calrsfs}
\DeclareMathAlphabet{\pazocal}{OMS}{zplm}{m}{n}
\usepackage{stmaryrd}


% packages for graphics
\usepackage{graphicx}
\graphicspath{ {./images/}{./images/} }
\usepackage{wrapfig}
\usepackage{subcaption}
\usepackage{capt-of}
\usepackage{cutwin}
\usepackage[format=hang]{caption}

% packages for page layout
\usepackage{geometry}
\geometry{
a4paper,
total={171.8mm,246.2mm},
left=19.1mm,
top=25.4mm,
}
\usepackage{fancyhdr}
\pagestyle{fancy}
\fancyhead{}
\fancyhead[LE,RO]{\rightmark}
\fancyhead[LO,RE]{\leftmark}
\fancyfoot{}
\fancyfoot[LE,RO]{\thepage}
\fancyfoot[LO,RE]{更新至\today}

% package for quotes
\usepackage{csquotes}

% redefine itemize
\usepackage{enumitem} % http://ctan.org/pkg/enumitem
\setlist{nosep}

% package for hyperlinks
\usepackage{hyperref}
\hypersetup{colorlinks=true}

% packages for footnotes
% \renewcommand{\thefootnote}{\fnsymbol{footnote}}
% \usepackage{perpage}
% \MakePerPage{footnote} 
\usepackage[perpage,symbol*]{footmisc}

% packages for bibliography
\usepackage[
backend=biber,
style=gb7714-2015,
gbpub=false,
sorting=none
]{biblatex}
\addbibresource{./ref/ref.bib}

% packages for definitions, theorems, proofs.
\usepackage{amsthm}

% set styles for definitions, lemmata, theorems, proofs, examples
\theoremstyle{definition}
\newtheorem{definition}{定义}
\newtheorem*{definition*}{定义}
\newtheorem{theorem}{定理}
\newtheorem*{theorem*}{定理}
\newtheorem{lemma}{引理}
\newtheorem{corollary}{推论}[theorem]
\let\oldproof\proof
\renewcommand{\proof}{\color{gray}\oldproof}

\theoremstyle{plain}
\newtheorem{example}{例}




% counter controlling
\usepackage{chngcntr}
\counterwithin{figure}{chapter}
\counterwithin{equation}{chapter}
\counterwithin{definition}{chapter}
\counterwithin{theorem}{chapter}
\counterwithin{lemma}{chapter}
\counterwithin{example}{chapter}
%\counterwithin{chapter}{part}
\setcounter{tocdepth}{2}

% package for subfiles
\usepackage{subfiles}

\usepackage{datetime2}

%\title{流变学}
%\author{孙尉翔\\mswxsun@scut.edu.cn}



%=====================================================================================================================
\begin{document}

\frontmatter
\pagenumbering{roman}
\section*{临时前言}\label{sec:preface}
\subfile{preface.tex}

\tableofcontents
\mainmatter
\pagenumbering{arabic}

\part{数学部分}\label{sec:II}
\begin{figure}[p]
    \centering
    \includegraphics[width=0.75\textwidth]{images/math.jpg}
\end{figure}
\chapter{集合与映射}\label{sec:II.1}
\subfile{Part II/II.1.1.tex}\label{sec:II.1.1}
\subfile{Part II/II.1.2.tex}\label{sec:II.1.2}
\subfile{Part II/II.1.3.tex}\label{sec:II.1.3}

\chapter{线性代数}\label{sec:II.2}
\section{向量空间}\label{sec:II.2.1}
\subfile{Part II/II.2.1.tex}

\section{线性变换}\label{sec:II.2.2}
\subsection{线性变换的定义和基本性质}\label{sec:II.2.2.1}
\subfile{Part II/II.2.2.1.tex}

\subsection{线性变换的坐标矩阵}\label{sec:II.2.2.2}
\subfile{Part II/II.2.2.2.tex}

\subsection{线性变换的转置}\label{sec:II.2.2.3}
\subfile{Part II/II.2.2.3.tex}

\section{基变换与坐标变换公式}\label{sec:II.2.3}
\subfile{Part II/II.2.3.tex}

\section{线性算符}\label{sec:II.2.4}
在线性变换的章节中,我们已经了解线性算符的定义(\ref{def:II.2.13})。除了满足所有线性变换的性质外,由于线性算符的定义域和值域都是同一个向量空间,所以在线性算符之上能够定义比较符合直觉的“乘法”,从而使得其求逆运算也更普遍。线性算符的可逆性是十分重要的性质。在大学一年级的线性代数课程中我们知道,矩阵的可逆性对应于线性方程组是否有非全零解的问题。在例\ref{exp:II.2.9}中我们知道求导和积分运算也是线性变换。它们所对应的方程是微分方程和积分方程。它们是否可逆,对应的是微分方程和积分方程的解的存在性问题。线性代数的研究者很早就发现,线性算符的可逆性与其行列式、对角化、特征值等性质密切相关,这些也是我们已经在大学一年级的线性代数课程中以矩阵为例有所了解的。在第\ref{sec:II.2.4.1}节,我们将从给出行线性算符的行列式、迹和特征值的不依赖基的选择的抽象定义,但是它们在给定有序基下的坐标运算,跟以往我们所学过的知识无异。

\subsection{线性算符的行列式、迹和特征值}\label{sec:II.2.4.1}
\subfile{Part II/II.2.4.1.tex}

\subsection{内积空间上的线性算符}\label{sec:II.2.4.2}
当我们为一个向量空间赋予了内积定义,这个空间上的线性算符的性质也相应增加了。由于向量的内积与“投影”、“正交”等几何概念密切相关,向量的范与“长度”这一几何概念概念密切相关,欧几里得范与内积之间也密切相关,因此内积空间上的线性算符也富有几何意义。我们先分别介绍伴随算符(\S \ref{sec:II.2.4.2_adjoint}伴随算符)和幺正算符(\S \ref{sec:II.2.4.2_unitary}幺正算符)。它们都属于正规算符(\S \ref{sec:II.2.4.2_normal}正规算符)。在正式的线性代数教材中,这些算符的内容十分繁杂。本讲义在此只列举一些流变学理论用到的结论。

\subsubsection{伴随算符}\label{sec:II.2.4.2_adjoint}
\subfile{Part II/II.2.4.2_adjoint.tex}
\subsubsection{幺正算符}\label{sec:II.2.4.2_unitary}
\subfile{Part II/II.2.4.2_unitary.tex}
\subsubsection{正规算符}\label{sec:II.2.4.2_normal}
\subfile{Part II/II.2.4.2_normal.tex}



%\subsection{正规算符在实数域上的几何意义}\label{sec:II.2.4.3}
\chapter{欧几里得空间}\label{sec:II.3}
在经典力学中,我们总假定物理事件发生欧几里得空间中。在本节我们将以集合的语言\emph{重新}引入欧几里得空间。
\section{欧几里得空间的构建}\label{sec:II.3.1}
\subfile{Part II/II.3.1.tex}

\section{角、直线、位置向量和坐标系}\label{sec:II.3.2}
\subfile{Part II/II.3.2.tex}

\section{几种线性算符在3维欧几里得空间中的几何意义}\label{sec:II.3.3}
\subfile{Part II/II.3.3.tex}

\section{等距变换的表示定理}\label{sec:II.3.4}
\subfile{Part II/II.3.4.tex}



\chapter{向量函数的微积分}\label{sec:II.4}
\section{向量函数及其可视化}\label{sec:II.4.1}
\subfile{Part II/II.4.1.tex}

\section{向量函数的极限与连续性}\label{sec:II.4.2}
\subfile{Part II/II.4.2.tex}

\section{向量函数的微分与导数}\label{sec:II.4.3}
\subfile{Part II/II.4.3.tex}

\section{曲线、曲面和积分定理}\label{sec:II.4.4}
\subfile{Part II/II.4.4.tex}

\part{连续介质力学基础}\label{sec:III}
\chapter{时空观与参考系}\label{sec:III.5}
在数学或物理学中,\emph{不变性(invariance)}常常用于描述一个数学或物理对象的性质在某种变换操作前后的结果不变。这是一个比较笼统地一般叙述。每个具体的“不变性”概念都需要定义清楚,具体是何种“变换操作”,以及何谓“性质的变换结果不变”。

在物理学中,任意两个不同观察者对同一现象的观测记录,在了解了这两个观察者的相对状态后,就能够互相换算。这种从一个观察者的观测记录到另一个观察者的观测记录的换算操作就是一种变换操作。\emph{物理客观性(physical objectivity)}要求,一切关于客观规律的陈述必须具有不依赖观察者的不变性。

在电磁学发展之前,物理学主要关心由宏观物体的运动现象总结出来的运动定律。它的完善版本是今天所说的经典力学。它已具有\emph{伽俐略变换(Galilean transformation)}下的不变性,或称\emph{伽俐略相对性(Galilean relativity)}。它说,物体运动定律在惯性系间(即相对运动是匀速直线\footnote{“直线”的概念依赖已有的欧几里得空间,后者隐含在了\emph{伽俐略时空(Galilean spacetime)}的构建当中\cite{Weatherall2022}。}运动的观察者之间)的形式具有不变性。在非惯性系下,要保持物体运动定律在惯性系中的形式不变,需要引入\emph{惯性力(intertial force)}。

在电磁学理论完善后,人们发现其不具有伽俐略变换下的不变性,但能在考虑了光速的修正——\emph{洛伦兹变换(Lorentz transformation)}下保持惯性系中的形式不变性。爱因斯坦在进一步明确了\emph{所有}物理定律在惯性系下的形式具有不变性的原则后,建立了\emph{狭义相对论(spectial theory of relativity)}力学。在此基础上,他又通过提出物质和能量以某种方式使时空弯曲的假设,替代牛顿的万有引力理论,解决了后者在狭义相对论中的矛盾,建立了\emph{广义相对论(general theory of  relativity)}\footnote{这里的相关知识应该已经在大学物理中介绍过了\cite[\S 1,\S 2.5]{邓文基2009大物上}\cite[\S 24]{邓文基2009大物下}。}。

\emph{参考标架(frame of reference)}是任何时空理论的核心内容。它代表着不同观察者观测同一物理事件时,在“何谓静止”标准上的主观差异。不同的观察者所选择的参考标架可作为他们各自用于描述运动的“绝对时空”。如果物理事件是客观的,那么不同观察者对同一物理事件的观察结果的差别,应仅来自这些观察者之间的相对运动。因此,各类客观性概念都依赖参考标架的概念得以陈述。

在经典连续介质力学中,\emph{物质客观性(material frame indifference)}一度是本构关系理论的基本原则之一\cite[Sect.293]{Truesdell1960}\cite[Sect.19]{Truesdell2004}。这一原则被用于约束材料本构关系的数学形式,得出了十分重要的流变学理论\footnote{例如\emph{简单流体(simple fluids)}理论\cite{Noll1958}。由其导出的\emph{测粘流(viscometric flow)}理论是流变学测量得以实现的理论基础\cite{Coleman1966}\cite{Walters1975}。}。但是关于物质客观性原理的的具体含义、以及(无论在哪一种具体含义下)它是否必须作为本构关系的基本原则的讨论一直持续到今天\footnote{\cite{Gennes1983}\cite{Frewer2009}}。不管是为了理解主流连续介质力学资料,还是为了跟上这一话题的最新讨论,都需要清晰而仔细地明确时空和参考标架的构建过程。

经典力学的时空观构建方式有很多版本\cite{Weatherall2022}。本讲义无意陷入到时空观的历史回顾中,仅选择W. Noll\cite{Noll1973}构建的\emph{新经典时空(neo-classical spacetime)},原因包括但不限于:大部分连续介质力学教科书默认或明确介绍的就是这一时空观。
\section{新经典时空}\label{sec:III.5.1}
\subfile{Part III/III.5.1.tex}
\section{参考标架}\label{sec:III.5.2}
\subfile{Part III/III.5.2.tex}
\section{标架变换}\label{sec:III.5.3}
\subfile{Part III/III.5.3.tex}

\chapter{连续物体的运动学}\label{sec:III.6}
\section{速度和加速度}\label{sec:III.6.1}
\subfile{Part III/III.6.1.tex}
\section{形变}\label{sec:III.6.2}
\subfile{Part III/III.6.2.tex}



%\section{物体的运动}
%\subfile{Part III/III.2.tex}

%\section{物体的形变}
%\subfile{Part III/III.3.tex}

%\section{物质描述与空间描述}
%\subfile{Part III/III.4.tex}

%\section{应变率张量}
%\subfile{Part III/III.5.tex}


%\section{应力张量}

%\section{守恒律}

%\section{Navier--Stokes方程}

%\section{流变测量学}

%\newpage\part{线性粘弹性本构方程}
%\section{线性粘弹性本构方程的建立}

%\section{线性粘弹性本构关系的一般预测}

%\section{记忆函数的具体形式}

%\section{松弛时间谱}

%\newpage\part{非线性粘弹性本构方程概览}
%\section{非线性粘弹性本构方程的构建原则}
%\section{准线性粘弹性}
%\section{广义牛顿流体}
%\section{微分形本构方程}
%\section{积分型本构方程}
%\section{屈服应力流体}


\appendix
\newpage\part{附录}
\chapter{线性代数部分定理的证明}\label{sec:A}
\section{范的定义的等价}\label{sec:A.1}
\subfile{Part App/A.1.tex}

\section{内积空间中的正交投影}\label{sec:A.2}
\subfile{Part App/A.2.tex}

\section{内积空间上的线性算符相关证明}\label{sec:A.3}
\subfile{Part App/A.3.tex}

\section{欧几里得空间相关证明}\label{sec:A.4}
\subfile{Part App/A.4.tex}

\chapter{向量函数微积分部分的证明}\label{sec:B}
\section{实空间上的一些拓扑概念}\label{sec:B.1}
\subfile{Part App/B.1.tex}

\section{向量函数可微分的必要条件与充分条件}\label{sec:B.2}
\subfile{Part App/B.2.tex}

\section{复合函数求导的链式法则}\label{sec:B.3}
\subfile{Part App/B.3.tex}

\section{反函数定理和隐函数定理}\label{sec:B.4}
\subfile{Part App/B.4.tex}

%\section{等距变换的表示定理}\label{sec:B.5}
%\subfile{Part App/B.5.tex}

%\chapter{曲线坐标系}\label{sec:C}
%\subfile{Part App/C.1.tex}
% \subfile{Part App/C.tex}

\newpage\part*{参考文献}
\printbibliography[heading=none]
\end{document}