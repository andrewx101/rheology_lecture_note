\documentclass[main.tex]{subfiles}
\begin{document}
欧几里得空间中的点是独立于其坐标而存在的。设点$\mathcal{E}是一个$n$维欧几里得空间,$\mathcal{V}$是其平移空间。$\mathcal{E}$中的一点$X\in\mathcal{E}$在选定的原点$O\in\mathcal{E}$和规范正交基$\left\{\mathbf{\hat{e}}_1,\cdots,\mathbf{\hat{e}}_n\right\}\subset\mathcal{V}$后才具有一组坐标$\left(x_1,\cdots,x_n\right)\in\mathbb{R}^n$;它就是点$X$的位置向量在这组基下的坐标。如果选定另一个原点$O^\prime$和另一组规范正交基$\left\{\mathbf{\hat{e}}^\prime_1,\cdots,\mathbf{\hat{e}}_n^\prime\right\}$,$\mathcal{E}$中的同一点$X$一般将对应一组不同的坐标$\left(x_1^\prime,\cdots,x_n^\prime\right)$。它们之间由欧几里得空间的等距变换表示定理
\end{document}