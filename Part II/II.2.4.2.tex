\documentclass[main.tex]{subfiles}
% 内积空间上的线性算符
\begin{document}
\begin{definition}[伴随算符]\label{def:II.6.1}
    设$\mathcal{V}$是数域$\mathbb{F}$上的内积空间,$\mathbf{T}\in\mathcal{L}\left(\mathcal{V}\right)$是$\mathcal{V}$上的线性算符。若另一线性算符$\mathbf{T}^*\in\mathcal{V}$满足$\left(\mathbf{Ta}|\mathbf{b}\right)=\left(\mathbf{a}|\mathbf{T}^*\mathbf{b}\right),\forall\mathbf{a},\mathbf{b}\in\mathcal{V}$,则称$\mathbf{T}^*$是$\mathbf{T}$的伴随算符(adjoint operator)。若$\mathbf{T}=\mathbf{T}^*$,则称$\mathbf{T}$是一个自伴随(self-adjoint)或厄米(hermitian)算符。若$\mathbf{T}=-\mathbf{T}^*$,则称$\mathbf{T}$是一个反厄米(skew hermitian)算符。
\end{definition}

对于有限维内积空间,每个线性算符有且只有一个伴随算符(证明见附录\S\ref{sec:VI.1})。易验,若$\mathbf{T}$是厄米算符,则$i\mathbf{T}$就是反厄米算符。

以下定理列举一些伴随算符的运算规律。

\begin{theorem}\label{thm:II.6.1}
    设$\mathcal{V}$是数域$\mathbb{F}$上的有限维内积空间,
    \begin{enumerate}
        \item $\left(\mathbf{T}+\mathbf{U}\right)^*=\mathbf{T}^*+\mathbf{U}^*,\forall\mathbf{T},\mathbf{U}\in\mathcal{L}\left(\mathcal{V}\right)$
        \item $\left(\alpha\mathbf{T}\right)^*=\overline{\alpha}\mathbf{T}^*,\forall\mathbf{T}\in\mathcal{L}\left(\mathcal{V}\right),\alpha\in\mathbb{F}$
        \item $\left(\mathbf{TU}\right)^*=\mathbf{U}^*\mathbf{T}^*,\forall\mathbf{T},\mathbf{U}\in\mathcal{L}\left(\mathcal{V}\right)$
        \item $\left(\mathbf{T}^*\right)^*=\mathbf{T},\forall\mathbf{T}\in\mathcal{L}\left(\mathcal{V}\right)$
    \end{enumerate}
\end{theorem}
\begin{proof}
    利用相关定义易证,留作练习。
\end{proof}

从上面的运算规律可以看出,线性算符的伴随与复数的共轭有些类似。例如在复数域$\mathbb{C}$上的内积空间$\mathcal{V}$上,任一线性算符$\mathbf{T}$都可以写成“虚部与实部”的形式,即$\mathbf{T}=\mathbf{U}_1+i\mathbf{U_2}$,其中$\mathbf{U}_1=\frac{1}{2}\left(\mathbf{T}+\mathbf{T}^*\right),\mathbf{U}_2=\frac{1}{2i}\left(\mathbf{T}-\mathbf{T}^*\right)$是自伴随算符。注意到,$i\mathbf{U}_2$是反厄米算符,故我们也常说任一线性算符$\mathbf{T}$都可以分解成一个厄米算符和一个反厄米算符。

一对伴随算符在给定基下的坐标矩阵之间的关系是矩阵的共轭转置。以下我们给出证明,且在证明过程顺便可以注意到内积空间上的向量和线性算符如何通过内积来取得它们在某有序基下的坐标。

\begin{theorem}
    设$\mathcal{V}$是数域$\mathbb{F}$上的有限维内积空间,$B=\left\{\mathbf{\hat{e}}\right\}$是$\mathcal{V}$的一组规范正交基,线性算符$\mathbf{T}$及其伴随算符$\mathbf{T}^*$在有序基$B$下的坐标分别是$T_{ij},T^\prime_{ij},i,j=1,\cdots,n$,则$T_{ij}=\overline{T^\prime_{ji}}$。
\end{theorem}
\begin{proof}
    设$\mathbf{a}\in\mathcal{V}$是$\mathcal{V}$中的任一向量,则$\mathbf{a}=\sum_{i=1}^n\alpha_i\mathbf{\hat{e}}_i$,其中$\alpha_i,i=1,\cdots,n$是$\mathbf{a}$在有序基$B$下的坐标,则$\alpha_i=\left(\mathbf{a}|\mathbf{\hat{e}}_i\right)$,因为
    \begin{align*}
        \left(\mathbf{a}|\mathbf{\hat{e}}_i\right) & =\left(\sum_{j=1}^n\alpha_j\mathbf{\hat{e}}_j|\mathbf{\hat{e}}_i\right)=\sum_{j=1}^n\alpha_j\left(\mathbf{\hat{e}}_i|\mathbf{\hat{e}}_j\right)
        \\&=\sum_{j=1}^n\alpha_j\delta_{ij}=\alpha_i
    \end{align*}

    由\S\ref{sec:II.4.2}可知,线性算符$\mathbf{T}$在给定有序基$B$下的坐标$T_{ij}$满足$\mathbf{T\hat{e}}_i=\sum_{j=1}^nT_{ji}\mathbf{\hat{e}}_j,i=1,\cdots,n$。同时每个$\mathbf{T\hat{e}}_i$作为一个向量在有序基$B$下的坐标满足上面刚刚证明结论,故$\mathbf{T\hat{e}}_i=\sum_{j=1}^n\left(\mathbf{T\hat{e}}_i|\mathbf{\hat{e}}_j\right)\mathbf{\hat{e}}_j$。由于$\left\{\mathbf{\hat{e}}_j\right\}$线性无关,当$\mathbf{T}\neq\mathbf{0}$时比较上述两结果可得$T_{ij}=\left(\mathbf{T\hat{e}}_j|\mathbf{\hat{e}}_i\right)$。

    由伴随算符定义,
    \begin{align*}
        T_{ji}=\left(\mathbf{T\hat{e}}_i|\mathbf{\hat{e}}_j\right) & =\left(\mathbf{\hat{e}}_i|\mathbf{T}^*\mathbf{\hat{e}}_j\right)            \\
                                                                   & =\overline{\left(\mathbf{T}^*\mathbf{\hat{e}}_j|\mathbf{\hat{e}}_i\right)} \\
                                                                   & =\overline{T^\prime_{ij}}
    \end{align*}
\end{proof}

注意,这一定理证明的是厄米算符仅限于在规范正交基下的坐标规律。在一般有序基下的坐标规律表达式将含有该组基的格拉姆矩阵分量。

回顾线性变换的转置(\S\ref{sec:II.4.3}),如果由$\mathcal{V}$的每个向量$\mathbf{a}\in\mathcal{V}$定义一个相应的线性算符$f_\mathbf{a}\in\mathcal{V}^*,f_\mathbf{a}\left(\mathbf{b}\right)\equiv\left(\mathbf{a}|\mathbf{b}\right)$,则$\mathcal{V}$上的任一线性算符$\mathbf{T}\in\mathcal{L}\left(\mathcal{V}\right)$与其转置$\mathbf{T}^\intercal\in\mathcal{L}\left(\mathcal{V}^*\right)$满足$\left(\mathbf{T}^\intercal f_\mathbf{a}\right)\left(\mathbf{b}\right)=\left(\mathbf{a}|\mathbf{Tb}\right),\forall\mathbf{a},\mathbf{b}\in\mathcal{V}$,同时$\mathbf{T}$与$\mathbf{T}^\intercal$在给定一对基$B\subset\mathcal{V}$与对偶基$B^*\subset\mathcal{V}^*$下的坐标矩阵互为矩阵的转置。比较而言,线性算符$\mathbf{T}$与其伴随算符$\mathbf{T}^*$满足$\left(\mathbf{Ta}|\mathbf{b}\right)=\left(\mathbf{a}|\mathbf{T}^*\mathbf{b}\right),\forall\mathbf{a},\mathbf{b}\in\mathcal{V}$,$\mathbf{T}$与$\mathbf{T}^*$在给定基$B$下的坐标矩阵之间互为矩阵的共轭转置。在这一比较下,线性算符的转置与伴随在是否需要共轭上的差别,来自它们生效的是内积的第一个向量还是第二个向量;内积定义中规定第二向量具有共轭线性,所以坐标矩阵需要共轭的是定义在内积的第二个向量的伴随算符。然而,线性变换的转置和伴随有本质的不同,因为$\mathbf{T}^\intercal\in\mathcal{L}\left(\mathcal{V}^*\right)$而$\mathbf{T}^*\in\mathcal{L}\left(\mathcal{V}\right)$,即它们属于不同的空间,作用于不同空间的向量。在本讲义后续内容中凡涉及到实数域上的线性算符,都暂不区分其转置和伴随,而均写成$\mathbf{T}^\intercal$。

在矩阵代数中,有“对称矩阵”的概念。如果数域$\mathbb{F}$上的$n\times n$矩阵$A\in\mathbb{F}^{n\times n}$满足$A=A^\intercal$,则称矩阵$A$是对称矩阵(symmetric matrix);若$A=-A^\intercal$,则称矩阵$A$是斜称矩阵(skew-symmetric matrix)。由厄米与反厄米算符的定义可知,只有在实数域$\mathbb{R}$上的内积空间上,厄米和反厄米算符在给定基于的坐标矩阵才是对称和斜称矩阵。因此,我们又把实数域上的厄米和反厄米算符称为对称(symmetric operator)和斜称算符(skew-symmetrix operator)。
\end{document}