\documentclass[main.tex]{subfiles}
% 欧几里得空间
\begin{document}
在经典力学中我们假设物理事件所发生的几何空间是欧几里得空间。在本节我们将以集合的语言重新描述欧几里得空间。

\subsection{准备知识:度量空间与等距变换}

我们把一个非空集合的元素称为“点”,然后我们为该集合中任意两个点定义“距离”。

\begin{definition}[度量空间]
设$\mathcal{E}$是一个非空集合,如果映射$d:\mathcal{E}\times\mathcal{E}\rightarrow\left[0,+\infty\right)\subset\mathbb{R}$满足
\begin{enumerate}
    \item 不可区分者的同一性:$d\left(x,y\right)=0\Leftrightarrow x=y\forall x,y\in\mathcal{E}$
    \item 对称性:$d\left(x,y\right)=d\left(y,x\right)\forall x,y\in\mathcal{E}$
    \item 三角不等式:$d\left(x,z\right)\leq d\left(x,y\right)+d\left(x,y\right)\forall x,y,z\in\mathcal{E}$
\end{enumerate}
则称$d$是定义在$\mathcal{E}$上的一种距离(distance)或度量(metric),有序对$\left(\mathcal{E},d\right)$是一个度量空间(metric space)。
\end{definition}

由定义易证,度量总是非负的,即$d\left(x,y\right)\geq0\forall x,y\in\mathcal{E}$。由3),$d\left(x,y\right)+d\left(y,x\right)\geq d\left(x,x\right)$;再由2),$d\left(x,y\right)+d\left(x,y\right)\geq d\left(x,x\right)$;最后由1)有$2d\left(x,y\right)\geq0\Rightarrow d\left(x,y\right)\geq0$当且仅当$x=y$时取等号。由于这是定义中的规定能够推出的,因此就算它就是我们对距离的最直观要求,但无需写进定义中。

\begin{example}
数域$\mathbb{F}$上的赋范向量空间$\mathcal{V}$,连同$d\left(\mathbf{a},\mathbf{b}\right)\equiv\left\|\mathbf{a}-\mathbf{b}\right\|$,构成一个度量空间$\left(\mathcal{V},d\right)$。
\end{example}

度量空间的完备性(completeness)是重要的概念,但在本讲义不受度量空间的完备性问题所影响,故暂不作介绍。

在引入度量概念后,由一个度量空间到另一个度量空间有一类特殊的映射叫做等距变换,定义如下。

\begin{definition}[等距变换]
设$\left(A,d_A\right),\left(B,d_B\right)$是两个度量空间,若映射$i:A\rightarrow B$满足$d_B\left(i\left(a\right),i\left(b\right)\right)=d_A\left(a,b\right),\forall a,b\in A$,则称$i$是$\left(A,d_A\right),\left(B,d_B\right)$之间的一个等距变换(isometry)。
\end{definition}

由度量的定义易验等距变换都是单射,因为当$a=b,d_B\left(i\left(a\right),i\left(b\right)\right)=d_A\left(a,b\right)=0\Rightarrow i\left(a\right)=i\left(b\right)$。

在欧几里得空间中,对几何对象的平移、旋转一定的角度、镜象这几种操作,都属于等距变换,因为在这些操作前后,任意两点间的距离是不变的。

\begin{example}
等距变换的一些例子:
\begin{enumerate}
    \item 给定两个度量空间:$\left(\mathbb{R}^+,d_1\right),d_1\left(x,y\right)=\left|\mathrm{log}x-\mathrm{log}y\right|\forall x,y\in\mathbb{R}^+$,$\left(\mathbb{R},d_2\right),d_2\left(x,y\right)=\left|x-y\right|\forall x,y\in\mathbb{R}$,则映射$\mathrm{log}:\mathbb{R}^+\rightarrow\mathbb{R}$是这两个度量空间的等距变换。
    \item $\mathcal{V}$是数域$\mathbb{F}$上的赋范向量空间,度量空间$\left(\mathcal{V},\left\|\cdot\right\|\right)$到$\left(\mathcal{V},\left\|\cdot\right\|_r\right)$的映射$i:\mathcal{V}\rightarrow\mathcal{V},i\left(\mathbf{a}\right)=r^{-1}\mathbf{a}\forall\mathbf{a}\in\mathcal{V}$是等距变换,其中$\left\|\cdot\right\|_r=r\left\|\cdot\right\|,r\in\mathbb{F}$。
\end{enumerate}
\end{example}

\begin{definition}[等距群]
由一个度量空间$\left(\mathcal{E},d\right)$到其自身的所有等距变换的集合:
\[\mathcal{I}=\left\{i:\mathcal{E}\rightarrow\mathcal{E}|i\text{是等距变换}\right\}\]
加上映射的复合操作$i_1\circ i_2,i_1,i_2\in\mathcal{I}$,形成一个交换群,称为等距群(isometry group)
\end{definition}

我们通过说明为什么$\mathcal{I}$是一个“群”来同时介绍“群”的定义。类似向量空间的概念,群是一个非空集合加上一种满足某些规则的运算形成的代数结构。按照等距变换的定义,易验$\mathcal{I}$中的元素满足:
\begin{enumerate}
    \item 封闭性:$i_1\circ i_2\in\mathcal{I},\forall i_1,i_2\in\mathcal{I}$
    \item 结合律:$i_1\circ \left(i_2\circ i_3\right)=\left(i_1\circ i_2\right)\circ i_3,\forall i_1,i_2,i_3\in\mathcal{I}$
    \item 恒等元素:恒等映射$\mathrm{id}_\mathcal{E}\in\mathcal{I}$满足$\mathrm{id}_\mathcal{E}\circ i=i\forall i\in\mathcal{I}$。
    \item 逆:对任一$i\in\mathcal{I}$存在唯一$i^{-1}\in\mathcal{I}$满足$i^{-1}\circ i=\mathrm{id}_\mathcal{E}$
    \item 交换律:$i_1\circ i_2=i_2\circ i_1,\forall i_1,i_2\in\mathcal{I}$
    
    以上的要求1\textasciitilde 4是群(group)的一般定义要求,第5条的满足使该群是一个交换群(commutative group)。我们发现,交换群的这些规定跟向量空间中的向量加法部分是相同的。因此,作为一个交换群的等距群$\mathcal{I}$只要再定义一种“标量乘”,就能形成一个向量空间了。我们规定在$\mathcal{I}$的某个交换子群(即$\mathcal{I}$的一个满足交换群定义的子集)$\mathcal{V}$中,可有如下操作
    
    \item 标量乘法:$\alpha i\in\mathcal{V}$;$\alpha\left(\beta i\right)=\left(\alpha\beta\right)i$;$1i=i,\forall\alpha,\beta\in\mathbb{R},i\in\mathcal{V}$
    \item 分配律:$\alpha\left(i_1\circ i_2\right)=\left(\alpha i_1\right)\circ\left(\alpha i_2\right)$;$\left(\alpha+\beta\right)i=\left(\alpha i\right)\circ\left(\beta i\right),\forall\alpha,\beta\in\mathbb{R},i,i_1,i_2\in\mathcal{V}$
    
    则$\mathcal{V}$是一个向量空间。如果我们再定义一个欧几里得范:
    
    \item $\left\|i\right\|^2=\left(i|i\right)= d^2\left(x,y\right),\forall x,y\in\mathcal{E}$其中$i$满足$i\left(x\right)=y$。
    
    则$\mathcal{V}$就是一个赋范内积空间,且该范为欧几里得范。
\end{enumerate}

已证明,任一度量空间上的等距群最多只有一个满足上述要求向量空间子群\cite{Noll1974}。

\subsection{欧几里得空间及其平移向量空间}
\begin{definition}[欧几里得空间]
若度量$d:\mathcal{E}\times\mathcal{E}\rightarrow\left[0,+\infty\right)\subset\mathbb{R}$上的等距群$\mathcal{I}$有满足上述条件1)到8)的子群$\mathcal{V}$,则称$d$是一个欧几里得度量(Euclidean metric),$d$赋予集合$\mathcal{E}$以欧几里得空间的结构,或称$\mathcal{E}$是一个欧几里得空间(Euclidean space),内积空间$\mathcal{V}$称$\mathcal{E}$的平移空间(translation space),$\mathcal{V}$中的向量称为$\mathcal{E}$的平移向量。
\end{definition}

我们从这一定义可以看出,欧几里得空间本质上是一个度量空间。任何一个的度量空间,总可在其等距群的向量空间子群上形成一个欧几里得空间。

从此,我们把一个欧几里得空间$\mathcal{E}$的元素$X,Y,\cdots\in\mathcal{E}$称为点(point),并用向量的记法表示$\mathcal{E}$的平移空间$\mathcal{V}$中的等距变换$\mathbf{u},\mathbf{v},\cdots\in\mathcal{V}$,称为平移向量。同时我们把$\mathcal{V}$中的一个平移向量$\mathbf{u}$作用于$\mathcal{E}$中一个点$X$得到另一个点$Y$表示为$Y=X+\mathbf{u},\mathbf{u}=Y-X,X-Y=-\mathbf{u}$,故两个$\mathcal{E}$中的点“相减”的结果是$\mathcal{V}$中的一个平移向量。且由于平移向量本质上是一个等距变换,故$d\left(X,Y\right)=d\left(Y,X\right)=\left\|\mathbf{u}\right\|=\left\|Y-X\right\|=\left\|X-Y\right\|$。注意,我们没有定义两个点“相加”($X+Y$)的意义。

下面我们在上述这种定义下的欧几里得空间中引入角、直线、位置向量和基本坐标系。

为了引入角,我们考虑欧几里得空间$\mathcal{E}$中的给定三个不同的点$X,O,Y\in\mathcal{E}$,由于$\mathcal{E}$的平移空间$\mathcal{V}$是一个赋范内积空间,故有极化恒等式,
\begin{align*}
\left\|X-O\right\|^2+\left\|Y-O\right\|^2&=\left\|\left(X-O\right)-\left(Y-O\right)\right\|^2+2\left(X-O|Y-O\right)\\
&=\left\|X-Y\right\|^2+2\left(X-O|Y-O\right)
\end{align*}
再应用柯西--施瓦茨不等式,有
\[\left\|X-O\right\|^2\left\|Y-O\right\|^2\geq\left|\left(X-O|Y-O\right)\right|^2\Leftrightarrow-1\leq\frac{\left(X-O|Y-O\right)}{\left\|X-O\right\|\left\|Y-O\right\|}\leq1\]

\begin{definition}[角]
设$\left(\mathcal{E},d\right)$是欧几里得空间,$\mathcal{E}$中的角是一个映射$\angle:\mathcal{E}^3\rightarrow\mathbb{R}$满足
\[\angle XOY\equiv\cos^{-1}\frac{\left(X-O|Y-O\right)}{\left\|X-O\right\|\left\|Y-O\right\|},\forall X,O,Y\in\mathcal{E},X\neq O\neq Y\]
称“点XOY所夹的锐角”,或简称“角XOY”。其中余弦函数$\cos:\left[0,\pi\right]\rightarrow\mathbb{R},\cos\left(x\right)=\frac{1}{2}\left(e^{ix}+e^{-ix}\right)$。
\end{definition}

注意到,上述定义中的余弦函数是一个双射,故其逆映射$\cos^{-1}$也是双射,角的取值范围$\mathrm{ran}\angle=\left[0,\pi\right]$。同时$\angle XOY$的顺序是重要的,$\angle YOX=-\angle XOY$。等距变换不改变角度。设$i:\mathcal{E}\rightarrow\mathcal{E}$是一个等距变换,可验证$\angle i\left(X\right)i\left(O\right)i\left(Y\right)=\angle XOY\forall X,O,Y\in\mathcal{E},X\neq O\neq Y$。

\begin{definition}[过两点的直线]
设$\left(\mathcal{E},d\right)$是欧几里得空间,给定两点$X,Y\in\mathcal{E},X\neq Y$,则$\mathcal{E}$的子集$L_{XY}=\left\{C|C=X+\alpha\left(Y-X\right),\alpha\in\mathbb{R}\right\}$是过$X,Y$两点的一条直线。如果$\angle XOY=\frac{\pi}{2}$,则直线$L_{OX}$与$L_{OY}$垂直,记为$L_{OX}\perp L_{OY}$。
\end{definition}

由角的定义,如果$L_{OX}\perp L_{OY}$,则$\left(X-O|Y-O\right)=0$。再由内积空间的格拉姆--施密特正交化过程可知,过$\mathcal{E}$中任一点$O$的两两垂直的直线最大条数都相等且等于$\mathrm{dim}\mathcal{V}$,故欧几里得空间的维数就自然地定义为其平移空间的维数。

$L_{XY}$又可记为$L_{XY}=\left\{C|C-X=\alpha\left(Y-X\right),\alpha\in\mathbb{R}\right\}$,它对应着平移空间$\mathcal{V}$的子集$L^{\mathcal{V}}_{XY}=\left\{\mathbf{u}|\mathbf{u}=\alpha\left(X-Y\right),\alpha\in\mathbb{R}\right\}$,易知该子集是$\mathcal{V}$的子空间,维数是1\footnote{由欧几里得空间的完备性和实数集的完备性可知$L_{XY}$和$L^{\mathcal{V}}_{XY}$是同构的。}。

如果选定某点$O\in\mathcal{E}$为原点(origin),则对任一点$X\in\mathcal{E}$可定义映射$\mathbf{r}_O:\mathcal{E}\rightarrow\mathcal{V},\mathbf{r}\left(X\right)\equiv\mathbf{r}_X=X-O,\forall X\in\mathcal{E}$,我们称这个向量值函数$\mathbf{r}_X$就是选定原点$O$下点$X$的位置向量(position vector)。注意,当且只当选定了原点后,欧几里得空间$\mathcal{E}$中的点才与其平移空间$\mathcal{V}$的向量通过位置向量这个映射形成双射(一一对应关系)。

设$\left\{\mathbf{\hat{e}}_i\right\}$是$\mathcal{V}$的一组规范正交基,则$\left\{L_{OX_i}|\mathbf{r}\left(X_i\right)=\mathbf{\hat{e}}_i,i=1,\cdots,\mathrm{dim}\mathcal{V}\right\}$称为$\mathcal{E}$的一个直角坐标系(rectangular coordinates),又称笛卡尔坐标系(Cartesian coordinates)。点$X\in\mathcal{E}$对应的位置向量$\mathbf{r}_X$在这组基下的坐标称为点$X$在该坐标系下的坐标。我们将一个选定的原点和一组规范正交基的组合$\left(O,\left\{\mathbf{\hat{e}}_i\right\}\right)$称为欧几里得空间$\mathcal{E}$的一个直角坐标系。我们常常默认一个欧几里得空间必然已经自带一个直角坐标系,称为基本坐标系(common coordinates),从而直接采用$\mathbb{R}^n$来表示点的坐标,$n=\mathrm{dim}\mathcal{V}$。在基本坐标系下,原点坐标为$\left(0,\cdots,0\right)$,第$i$个基向量为$\left(0,\cdots,1,\cdots,0\right)^\intercal$,也就是除第$i$个分量为1外其他分量均为零的有序实数$n$元组。

\subsection{等距变换的表示定理}
下面我们介绍等距变换的一个重要定理:等距变换的表示定理。这个定理使得等距变换可以具体地表达成一个通式。这个定理也是后面介绍物理定律的标架变换不变性时的理论基础。

\begin{theorem}[等距变换的表示定理]\label{thm:II.9.1}
设$\left(\mathcal{E},d\right)$是一个欧几里德空间,$\mathcal{V}$是其平移空间,选定任一点$X_0\in\mathcal{E}$,则$\mathcal{E}$上的任一等距变换$i:\mathcal{E}\rightarrow\mathcal{E},i\in\mathcal{I}$都可表示为
\[
i\left(X\right)=i\left(X_0\right)+\mathbf{Q}_i\left(X-X_0\right)
\]
其中$\mathbf{Q}_i$是一个正交算符,关于$i$唯一存在。
\end{theorem}
\begin{proof}
见附录。
\end{proof}

\begin{corollary}
欧几里得空间上的等距变换都是双射。
\end{corollary}
\begin{proof}
定理\ref{thm:II.9.1}已经暗示欧几里得空间上的等距变换都是单射。故仅需再证明对任一$i:
\mathcal{E}\rightarrow\mathcal{E}$和$Y\in\mathcal{E}$总存在一个$X\in\mathcal{E}$满足$i\left(X\right)=Y$。我们可直接找出这样的$X$:
\[
X=X_0+\mathbf{Q}^{-1}\left(Y-i\left(X_0\right)\right)
\]
验证这就是满足条件的$X$:
\begin{align*}
i\left(X\right)&=i\left(X_0\right)+\mathbf{Q}\left(\mathbf{X}_0+\mathbf{Q}^{-1}\left(Y-i\left(X_0\right)\right)-X_0\right)\\
&=i\left(X_0\right)+Y-i\left(X_0\right)\\
&=Y
\end{align*}
\end{proof}

定理\ref{thm:II.9.1}的通俗解释:给定任一等距变换$i$,仅需知道它对某一参考点$X_0$的像是哪个点,以及该变换的特征正交算符$\mathbf{Q}_i$,就可以知道它对任意点$X$的像。因此这一定理给出了等距变换的通用表达式。

这里的等距变换$i$不一定是欧几里得空间$\mathcal{E}$的平移向量空间$\mathcal{V}$中的元素(前面提到过$\mathcal{V}$至多是$\mathcal{I}$的子群)。例如旋转和镜向反转都是等距变换,却不满足向量空间对向量的要求。

一般$i\left(X_0\right)$是容易找到的,但是$\mathbf{Q}_i$不是直接易得的。我们可以举例认识$\mathbf{Q}_i$的一般意义。

\begin{example}
考虑欧几里得空间$\left(\mathcal{E},d\right)$上的以下等距变换,其中$\mathbf{Q}$是一个正交算符,$X_0,C$是$\mathcal{E}$中固定的点:
\begin{align*}
    i_1\left(X\right)&=X+\left(C-X_0\right)\\
    i_2\left(X\right)&=X_0+\mathbf{Q}\left(X-X_0\right)\\
    i_3\left(X\right)&=X+\mathbf{Q}^{-1}\left(C-X_0\right)
\end{align*}

$i_1$把任一点向固定的方向平移固定距离($i_1\left(X\right)=X+\mathbf{u},\mathbf{u}\equiv C-X_0$)。

$i_1\circ i_2=i_2\circ i_3$(自行验证作为练习。)

当$\mathbf{Q}=\mathbf{I}$时,$i_2$是恒等映射。当$\mathbf{Q}\neq\mathbf{I}$时,由正交(幺正)算符性质$\mathrm{det}\mathbf{Q}=\pm 1$。当$\mathrm{det}\mathbf{Q}=1$时,$i_2$是一种旋转操作;当$\mathrm{det}\mathbf{Q}=-1$时,由$\mathbf{Q}=\left(-\mathbf{I}\right)\left(-\mathbf{Q}\right)$和$\mathrm{det}\left(-\mathbf{Q}\right)=1$可知$i_2$是先进行了一个旋转($-\mathbf{Q}$)再进行了反转($-\mathbf{I}$)的操作。
\end{example}

在连续介质力学中,我们只考虑$\mathrm{det}\mathbf{Q}=1$的情况,即等距变换中的正交算符仅表旋转。在此限定下,定理\ref{thm:II.9.1}说的就是,欧几里得空间的任一等距变换(镜像除外)都是平移加旋转。
\end{document}