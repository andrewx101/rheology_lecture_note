\documentclass[main.tex]{subfiles}
% 内积空间中的正交投影
\begin{document}
本节是\S\ref{sec:II.2.1.4}的补充知识,在不同的命题证明中将会用到。
\subsection{线性无关子空间的直和}
\begin{definition}[向量组的和]\label{def:A.1}
    若$S_1,\cdots,S_k$是向量空间$\mathcal{V}$的子集,则所有$\sum_{i=1}^k\mathbf{a}_i,\mathbf{a}_i\in S_i$的集合$S$,记作$\sum_{i=1}^k S_i$,称\emph{向量组$S_1,\cdots,S_k$的和(sum of the subsects)}。
\end{definition}

注意,定义\ref{def:A.1}中的$S$的元素,是从每个$S_i$中抽一个向量$\mathbf{a}_i$出来再求和的结果。求和式当然也可以写成加法:$\sum_{i=1}^k S_k=S_1+\cdots+S_k$。若$\mathcal{W}_1,\cdots,\mathcal{W}_k$是$\mathcal{V}$的子空间,则易验$\mathcal{W}=\sum_{i=1}^k\mathcal{W}_i$也是$\mathcal{V}$的一个子空间,且具体为由$\mathcal{W}_1,\cdots,\mathcal{W}_k$共同线性生成的子空间。我们还能为一系列子空间定义类似“线性无关”的概念。

\begin{definition}[线性无关子空间]\label{def:A.2}
    设$\mathcal{W}_1,\cdots,\mathcal{W}_k$是向量空间$\mathcal{V}$的子空间。若
    \[\sum_{i=1}^k\mathbf{a}_i=\mathbf{0},\mathbf{a}_i\in\mathcal{W}_i\Leftrightarrow \mathbf{a}_i=\mathbf{0},\quad\forall i=1,\cdots,k\]
    则称$\mathcal{W}_1,\cdots,\mathcal{W}_k$\emph{线性无关(linlearly independent)}。
\end{definition}

沿用上述符号设定,记$\mathcal{W}=\sum_{i=1}^k\mathcal{W}_i$,则任一向量$\mathbf{a}\in\mathcal{W}$可表示为$\sum_{i=1}^k\mathbf{a}_i,\mathbf{a}_i\in\mathcal{W}_i$。若$\left\{\mathcal{W}_i\right\}$是线性无关的,那么$\mathbf{a}$的这一表达式就是唯一的。因为若另有$\mathbf{b}_1,\cdots,\mathbf{b}_k$满足$\mathbf{a}=\sum_{i=1}^k\mathbf{b}_i,\mathbf{b}_i\in\mathcal{W}_i$,则
\[\mathbf{a}-\mathbf{a}=\mathbf{0}=\sum_{i=1}^k\left(\mathbf{a}_i-\mathbf{b}_i\right)\Rightarrow\mathbf{a}_i-\mathbf{b}_i=\mathbf{0},\forall i=1,\cdots,k\]
其中用到了$\left\{\mathcal{W}_i\right\}$线性无关的定义。换言之,若$\left\{\mathcal{W}_i\right\}$线性无关,则$\mathcal{W}=\sum_{i=1}^k$中的任一向量$\mathbf{a}$都可唯一对应于一个向量组$\left\{\mathbf{a}_i\right\},\mathbf{a}_i\in\mathcal{W}$,并由后者线性表出。但是,从线性无关的子空间$\left\{\mathcal{W}_i\right\}$中各取一个向量$\mathbf{a}_i\in\mathcal{W}_i$组成的向量组$\left\{\mathbf{a}_i\right\}$未必线性无关。因此一般地$\left\{\mathbf{a}_i\right\}$不是$\mathbf{a}$的基。

在有限维向量空间下,以下定理与线性无关子空间的定义之间互为充要条件。
\begin{theorem}\label{thm:A.2}
    设$\mathcal{V}$是有限维向量空间,$\mathcal{W}_1,\cdots,\mathcal{W}_k$是$\mathcal{V}$的子集,且$\mathcal{W}=\sum_{i=1}^k\mathcal{W}_i$,则以下命题相互等价:
    \begin{enumerate}
        \item $\left\{\mathcal{W}_i\right\}$线性无关;
        \item \[\forall j=2,\cdots,k,\quad\mathcal{W}_j\cup\sum_{i=1}^{j-1}\mathcal{W}_i=\left\{\mathbf{0}\right\}\]
        \item 若$B_i$是$\mathcal{W}_i$的一组有序基,则$B=\left(B_1,\cdots,B_k\right)$是$\mathcal{W}$的一组有序基。
    \end{enumerate}
\end{theorem}

我们把线性无关子空间的和称作\emph{直和(direct sum)}。记作$\mathcal{W}=\mathcal{W}_1\bigoplus\cdots\bigoplus\mathcal{W}_k$。



% ===============================================================================
\subsection{内积空间上的正交投影}
\begin{definition}[最好近似]\label{def:A.3}
    设$\mathcal{V}$是$\mathbb{F}$上的内积空间并自然定义了欧几里得范,$\mathcal{W}$是$\mathcal{V}$的子空间,对任一$\mathbf{b}\in\mathcal{V}$,如果$\mathbf{a}\in\mathcal{W}$满足
    \[\left\|\mathbf{b}-\mathbf{a}\right\|\leq\left\|\mathbf{b}-\mathbf{c}\right\|,\quad\forall\mathbf{c}\in\mathcal{W},\text{当且仅当$\mathbf{c}=\mathbf{a}$时取等号}\]
    则称$\mathbf{a}\in\mathcal{W}$是$\mathbf{b}$在$\mathcal{W}$中的\emph{最好近似(best approximation)}。
\end{definition}

向量$\mathbf{b}$在子空间上的最好近似,最直观的理解就是空间向量在平面上的投影向量。找到这一投影向量的方法是“作垂线”,且该平面上,$\mathbf{b}$的投影向量总是唯一存在。以下定理是这些直观认知的严格阐述。

\begin{theorem}[最好近似的表示]\label{thm:A.3}
    设$\mathcal{W}$是内积空间$\mathcal{V}$的子空间,$\mathbf{b}$是$\mathcal{V}$的任一向量,则
    \begin{enumerate}
        \item 当且仅当$\mathbf{a}\in\mathcal{W}$与$\mathbf{b}-\mathbf{a}$正交时,$\mathbf{a}$是$\mathbf{b}$在$\mathcal{W}$中的最好近似。
        \item $\mathbf{b}$在$\mathcal{W}$中总存在唯一一个最好近似。
        \item 若$\mathcal{W}$是有限维的,$\left\{\mathbf{\hat{e}}_1,\cdots\mathbf{\hat{e}}_n\right\}$是$\mathcal{W}$的一组规范正交基,则向量
              \[\mathbf{a}=\sum_{k=1}^n\frac{\left(\mathbf{b}|\mathbf{\hat{e}}_k\right)}{\left(\mathbf{\hat{e}}_k|\mathbf{\hat{e}}_k\right)}\mathbf{\hat{e}}_k\]
              是$\mathbf{b}$在$\mathcal{W}$中的唯一最小近似。
    \end{enumerate}
\end{theorem}
\begin{proof}
    由极化恒等式,以下关系式总成立
    \[\left\|\mathbf{b}-\mathbf{c}\right\|^2=\left\|\mathbf{b}-\mathbf{a}+\mathbf{a}-\mathbf{c}\right\|^2=\left\|\mathbf{b}-\mathbf{a}\right\|^2+2\mathrm{Re}\left(\mathbf{b}-\mathbf{a}|\mathbf{a}-\mathbf{c}\right)+\left\|\mathbf{a}-\mathbf{c}\right\|^2\]

    “1”的证明:若$\mathbf{b}-\mathbf{a}$与$\mathcal{W}$的所有向量都正交,且$\mathbf{c}\in\mathcal{W},\mathbf{c}\neq\mathbf{a}$,则由三角不等式有
    \[\left\|\mathbf{b}-\mathbf{c}\right\|^2=\left\|\mathbf{b}-\mathbf{a}\right\|^2+\left\|\mathbf{a}-\mathbf{c}\right\|^2 > \left\|\mathbf{b}-\mathbf{a}\right\|^2\]
    因此$\mathbf{b}-\mathbf{a}$与$\mathcal{W}$的所有向量都正交$\Rightarrow\mathbf{a}$是$\mathbf{b}$在$\mathcal{W}$的最好近似。

    若$\left\|\mathbf{b}-\mathbf{a}\right\|\leq\left\|\mathbf{b}-\mathbf{c}\right\|,\forall\mathbf{c}\in\mathcal{W}$,则由最开始的等式,有
    \[\left\|\mathbf{b}-\mathbf{c}\right\|^2=\left\|\mathbf{b}-\mathbf{a}\right\|^2=2\mathrm{Re}\left(\mathbf{b}-\mathbf{a}|\mathbf{a}-\mathbf{c}\right)+\left\|\mathbf{a}-\mathbf{c}\right\|^2\geq0\]
    而实际上,由于$\mathbf{c}$是$\mathcal{W}$中任意一个向量,所以$\mathbf{a}-\mathbf{c}$也是$\mathcal{W}$中任意一个向量。因此我们实际证明了,
    \[2\mathrm{Re}\left(\mathbf{b}-\mathbf{a}|\mathbf{t}\right)+\left\|\mathbf{t}\right\|^2\geq0,\quad\forall\mathbf{t}\in\mathcal{W}\]
    现不妨取
    \[\mathbf{t}=-\frac{\left(\mathbf{b}-\mathbf{a}|\mathbf{a}-\mathbf{c}\right)}{\left\|\mathbf{a}-\mathbf{c}\right\|^2}\left(\mathbf{a}-\mathbf{c}\right)\]
    则$\mathbf{t}$亦须满足
    \begin{align*}
                        & 2\mathrm{Re}\left(\mathbf{b}-\mathbf{a}|\mathbf{t}\right)+\left\|\mathbf{t}\right\|^2\geq 0                                                                                                                                                                                                                                                                  \\
        \Leftrightarrow & 2\mathrm{Re}\left[\left(\left.\mathbf{b}-\mathbf{a}\right.\left|-\frac{\left(\mathbf{b}-\mathbf{a}|\mathbf{a}-\mathbf{c}\right)}{\left\|\mathbf{a}-\mathbf{c}\right\|^2}\left(\mathbf{a}-\mathbf{c}\right)\right.\right)\right]+\frac{\left|\left(\mathbf{b}-\mathbf{a}|\mathbf{a}-\mathbf{c}\right)\right|^2}{\left\|\mathbf{a}-\mathbf{c}\right\|^2}\geq 0 \\
        \Leftrightarrow & -2\frac{\left|\left(\mathbf{b}-\mathbf{a}|\mathbf{a}-\mathbf{c}\right)\right|}{\left\|\mathbf{a}-\mathbf{c}\right\|^2}+\frac{\left|\left(\mathbf{b}-\mathbf{a}|\mathbf{a}-\mathbf{c}\right)\right|^2}{\left\|\mathbf{a}-\mathbf{c}\right\|^2}\geq 0                                                                                                          \\
        \Leftrightarrow & -\left|\left(\mathbf{b}-\mathbf{a}|\mathbf{a}-\mathbf{c}\right)\right|^2\left\|\mathbf{a}-\mathbf{c}\right\|^{-2}\geq 0\Leftrightarrow\left(\mathbf{b}-\mathbf{a}|\mathbf{b}-\mathbf{c}\right)=0
    \end{align*}
    故$\left\|\mathbf{b}-\mathbf{a}\right\|\leq\left\|\mathbf{b}-\mathbf{c}\right\|,\forall\mathbf{c}\in\mathcal{W}\Rightarrow\left(\mathbf{b}-\mathbf{a}|\mathbf{a}-\mathbf{c}\right)=0,\forall\mathbf{c}\in\mathcal{W}$,注意$\mathbf{a}-\mathbf{c}$是$\mathcal{W}$中的任意向量。“1”证毕。

    “2”的证明:设$\mathbf{a},\mathbf{a}^\prime\in\mathcal{W}$都是向量$\mathbf{b}\in\mathcal{V}$在$\mathcal{W}$中的最好近似,且一般地$\mathbf{b}\neq\mathbf{a},\mathbf{b}\neq\mathbf{a}^\prime$,则由最好近似的定义有$\left(\mathbf{b}-\mathbf{a}|\mathbf{a}^\prime\right)=\left(\mathbf{b}-\mathbf{a}|\mathbf{a}\right)=0$,故$\mathbf{a}^\prime=\mathbf{a}$。“2”证毕。

    “3”的证明:回顾格拉姆--施密特正交化过程,若$\mathcal{V}$中有规范正交向量组$\left\{\mathbf{\hat{e}}_1,\cdots,\mathbf{\hat{e}}_n\right\}$,$\mathbf{b}$与它们线性无关,则“3”中的表达式,是对线性无关向量组$\left\{\mathbf{\hat{e}}_1,\cdots,\mathbf{\hat{e}}_n,\mathbf{b}\right\}$进行格拉姆--施密特正交化,使$\mathbf{b}-\mathbf{a}$与所有$\mathbf{\hat{e}}_1,\cdots,\mathbf{\hat{e}}_n$都正交时所使用的表达式。换言之,若$\mathbf{a}$满足命题,则$\mathbf{b}-\mathbf{a}$与$\left\{\mathbf{\hat{e}}_1,\cdots,\mathbf{\hat{e}}_n\right\}$都正交,亦与它们线性生成的子空间(即$\mathcal{W}$)上的所有向量都正交。由“1”,$\mathbf{a}$是$\mathbf{b}$在$\mathcal{W}$中的最好近似。由“2”,$\mathbf{a}$是唯一的。“3”证毕。
\end{proof}

\begin{definition}[正交补集]\label{def:A.4}
    设$\mathcal{V}$是一个内积空间,$S$是$\mathcal{V}$的任一子集,$\mathcal{V}$中的与$S$中每个向量都正交的向量的集合记为$S^\perp$,称$S$的\emph{正交补集(orthogonal complement)}。
\end{definition}

虽然我们定义了正交补集是什么,但仍担心未必总能为任一集合找到其非空的正交补集。但由于$\mathbf{0}$与所有向量都正交,故甚至能说$\mathcal{V}$的正交补集是$\left\{\mathbf{0}\right\}$,反之$\left\{\mathbf{0}\right\}^\perp=\mathcal{V}$。

$S^\perp$总是$\mathcal{V}$的一个子空间。首先$S^\perp$总是含$\mathbf{0}$。其次,若$\mathbf{a},\mathbf{b}\in S^\perp$,则$\forall\mathbf{c}\in S$,
\[\left(\mathbf{a}+\gamma\mathbf{b}|\mathbf{c}\right)=\left(\mathbf{a}|\mathbf{c}\right)+\gamma\left(\mathbf{b}|\mathbf{c}\right)=0,\quad\forall\gamma\in\mathbb{F}\]
其中$\mathbb{F}$是$\mathcal{V}$所在的数域。

\begin{definition}[正交投影]\label{def:A.5}
    设$\mathcal{V}$是一个内积空间,$\mathbf{b}$是$\mathcal{V}$中的一个向量,$\mathcal{W}$是$\mathcal{V}$的一个子空间。若$\mathcal{W}$中存在向量$\mathbf{a}$是$\mathbf{b}$在$\mathcal{W}$中的最好近似,则称向量$\mathbf{a}$是\emph{向量$\mathbf{b}$在$\mathcal{W}$上的正交投影(orthogonal projection of $\mathbf{b}$ on $\mathcal{W}$)}。若$\mathcal{W}$的维数是$m\leq n$,$\left\{\mathbf{\hat{e}}_k\right\}_{k=1}^m$是$\mathcal{W}$的一组规范正交基,定义映射$\mathbf{E}:\mathcal{V}\to\mathcal{W}$,
    \[\mathbf{Eb}=\sum_{k=1}^m\frac{\left(\mathbf{b}|\mathbf{\hat{e}}_k\right)}{\left(\mathbf{\hat{e}}_k|\mathbf{\hat{e}}_k\right)}\mathbf{\hat{e}}_k,\quad\forall\mathbf{b}
        \in\mathcal{V}\]
    则称映射$\mathcal{E}$是\emph{内积空间$\mathcal{V}$到其子空间$\mathcal{W}$的正交投影(orthogonal projection of $\mathcal{V}$ to $\mathcal{W}$)}。
\end{definition}

请注意以上定义中,两种“正交投影”的对象。

定理\ref{thm:A.3}成立的内积空间$\mathcal{V}$未必是有限维的,故基于定理\ref{thm:A.3}“3”的概念的正交投影映射,可以无限维内积空间向的其有限维子空间的投影。由定理\ref{thm:A.3}“3”,这样的投影映射是总存在的,作为推论如下。

\begin{corollary}
    延用定理\ref{thm:A.3}的设定,设映射$\mathbf{E^\prime}:\mathcal{V}\to\mathcal{W},\mathbf{E}^\prime\mathbf{b}=\mathbf{b}-\mathbf{Eb},\forall\mathbf{b}\in\mathcal{V}$是由$\mathcal{V}$到$\mathcal{W}^\perp$的正交投影。
\end{corollary}
\begin{proof}
    由定义\ref{def:A.3}和定理\ref{thm:A.2},$\mathbf{Eb}$就是向量$\mathbf{b}$在$\mathcal{W}$中的最好近似,因此$\mathbf{b}-\mathbf{Eb}$与所有$\mathcal{W}$正交,即$\mathbf{E}^\prime\mathbf{b}\in\mathcal{W}^\perp$。由于$\mathbf{Eb}\in\mathcal{W}$,故$\mathbf{Eb}=\mathbf{b}-\mathbf{E}\prime\mathbf{b}$与$\mathcal{W}^\perp$中的所有向量都正交。由定理\ref{thm:A.3}“1”,$\mathbf{E}^\prime\mathbf{b}$是$\mathbf{b}$在$\mathcal{W}^\perp$上的最好近似。
\end{proof}

我们把正交投影映射写成线性算符的样子,因为它确实是线性算符,作为定理如下。
\begin{theorem}\label{thm:A.4}
    设$\mathcal{V}$是内积空间,$\mathcal{W}$是$\mathcal{V}$的一个子空间,$\mathbf{E}$是由$\mathcal{V}$到$\mathcal{W}$的正交投影,则$\mathcal{E}$是一个正定线性算符,$\mathcal{W}^\perp$是$\mathcal{E}$的零空间,$\mathcal{V}=\mathcal{W}\bigoplus\mathcal{W}^\perp$。
\end{theorem}
\begin{proof}
    设$\mathbf{b}\in\mathcal{V}$,则有$\mathbf{Eb}$是$\mathbf{b}$在$\mathcal{W}$上的最好近似,$\mathbf{E}\left(\mathbf{Eb}\right)=\mathbf{Eb}$,故$\mathbf{E}^2=\mathbf{E}$。

    设$\mathbf{a},\mathbf{b}\in\mathcal{V},\gamma\in\mathbb{F}$,其中$\mathbb{F}$是$\mathcal{V}$所在的数域,则向量$\mathbf{a}-\mathbf{Ea}$和$\mathbf{b}-\mathbf{Eb}$分别都与$\mathcal{W}$的所有向量正交,即这两个向量都属于$\mathcal{W}^\perp$。由于$\mathcal{W}^\perp$必是$\mathcal{V}$的子空间,故有
    \[
        \mathbf{a}-\mathbf{Ea}+\gamma\left(\mathbf{b}-\mathbf{Eb}\right)\in\mathcal{W}^\perp\Leftrightarrow\mathbf{a}+\gamma\mathbf{b}-\mathbf{E}\left(\mathbf{a}+\gamma\mathbf{b}\right)\in\mathcal{W}^\perp
    \]
    因此$\mathbf{E}$是正定线性算符。

    前文说过,$\mathbf{Eb}$是$\mathcal{W}$中唯一使$\mathbf{b}-\mathbf{Eb}\in\mathcal{W}^\perp$的向量,故若$\mathbf{Eb}=\mathbf{0}$,则$\mathbf{b}-\mathbf{Eb}=\mathbf{b}\in\mathcal{W}^\perp$。反之,若$\mathbf{b}\in\mathcal{W}^\perp$,设$\mathbf{a}\in\mathcal{W}$是使$\mathbf{b}-\mathbf{a}\in\mathcal{W}^\perp$的向量,则有$\left(\mathbf{a}|\mathbf{b}\right)=0$且$\left(\mathbf{a}|\mathbf{b}-\mathbf{a}\right)=0$,故只可能$\mathbf{a}=\mathbf{0}$。而事实上$\mathbf{Eb}$是$\mathcal{W}$中唯一使$\mathbf{b}-\mathbf{Eb}\in\mathcal{W}^\perp$的向量,故$\mathbf{a}=\mathbf{Eb}=\mathbf{0}$。换言之,$\mathcal{W}^\perp$中的向量全是使$\mathbf{Eb}=\mathbf{0}$的$\mathbf{b}$,即$\mathcal{W}^\perp=\mathrm{ker}\mathbf{E}$。


    由$\mathbf{Eb}$与$\mathbf{b}$和$\mathbf{b}-\mathbf{Eb}$的唯一性,有以下恒等式
    \[\mathbf{b}=\mathbf{Eb}+\mathbf{b}-\mathbf{Eb}, \forall\mathbf{b}\in\mathcal{V}\]
    说明总有$\mathcal{V}=\mathcal{W}+\mathcal{W}^\perp$。而由前面的证明又知$\mathcal{W}\cup\mathcal{W}^\perp=\left\{0\right\}$,由定理\ref{thm:A.2}有$\mathcal{V}=\mathcal{W}\bigoplus\mathcal{W}^\perp$。
\end{proof}
\end{document}