\documentclass[main.tex]{subfiles}
% 欧几里得空间
\begin{document}
\subsection{度量空间和等距变换}
我们把一个非空集合的元素称为“点”,然后我们为该集合中任意两个点定义“距离”。

\begin{definition}[度量空间]\label{def:II.3.1}
    设$\mathcal{E}$是一个非空集合,如果映射$d:\mathcal{E}\times\mathcal{E}\rightarrow\left[0,+\infty\right)\subset\mathbb{R}$满足
    \begin{enumerate}
        \item 不可区分者的同一性:$d\left(x,y\right)=0\Leftrightarrow x=y,\quad\forall x,y\in\mathcal{E}$
        \item 对称性:$d\left(x,y\right)=d\left(y,x\right),\quad\forall x,y\in\mathcal{E}$
        \item 三角不等式:$d\left(x,z\right)\leq d\left(x,y\right)+d\left(x,y\right),\quad\forall x,y,z\in\mathcal{E}$
    \end{enumerate}
    则称$d$是定义在$\mathcal{E}$上的一个\emph{度量(metric)},有序对$\left(\mathcal{E},d\right)$是一个\emph{度量空间(metric space)}。
\end{definition}

由定义易证,度量总是非负的,即$d\left(x,y\right)\geq0,\forall x,y\in\mathcal{E}$。由第3条,$d\left(x,y\right)+d\left(y,x\right)\geq d\left(x,x\right)$;再由第2条,$d\left(x,y\right)+d\left(x,y\right)\geq d\left(x,x\right)$;最后由第一条有$2d\left(x,y\right)\geq0\Rightarrow d\left(x,y\right)\geq0$(当且仅当$x=y$时取等号)。由于这是定义\ref{def:II.3.1}中的规定能够推出的,因此就算它是我们对距离的最直观要求,但却无需写进定义\ref{def:II.3.1}中。

\begin{example}\label{exp:II.3.1}
    数域$\mathbb{F}$上的赋范向量空间$\mathcal{V}$,连同$d\left(\mathbf{a},\mathbf{b}\right)\equiv\left\|\mathbf{a}-\mathbf{b}\right\|$,构成一个度量空间$\left(\mathcal{V},d\right)$。
\end{example}

由一个度量空间到另一个度量空间的映射当中,有一类特殊的映射叫做等距变换。定义如下。

\begin{definition}[等距变换]\label{def:II.3.2}
    设$\left(A,d_A\right),\left(B,d_B\right)$是两个度量空间,若映射$i:A\rightarrow B$满足$d_B\left(i\left(a\right),i\left(b\right)\right)=d_A\left(a,b\right),\forall a,b\in A$,则称$i$是$\left(A,d_A\right),\left(B,d_B\right)$之间的一个\emph{等距变换(isometry)}。
\end{definition}

等距变换都是单射,因为当$a=b,d_B\left(i\left(a\right),i\left(b\right)\right)=d_A\left(a,b\right)=0\Rightarrow i\left(a\right)=i\left(b\right)$。

\begin{example}\label{exp:II.3.2}
    等距变换的一些例子:
    \begin{enumerate}
        \item 给定两个度量空间:$\left(\mathbb{R}^+,d_1\right),d_1\left(x,y\right)=\left|\mathrm{log}x-\mathrm{log}y\right|,\quad\forall x,y\in\mathbb{R}^+$,和$\left(\mathbb{R},d_2\right),d_2\left(x,y\right)=\left|x-y\right|\forall x,y\in\mathbb{R}$,则映射$\mathrm{log}:\mathbb{R}^+\rightarrow\mathbb{R}$是这两个度量空间的一个等距变换。
        \item $\mathcal{V}$是数域$\mathbb{F}$上的赋范向量空间,度量空间$\left(\mathcal{V},\left\|\cdot\right\|\right)$到$\left(\mathcal{V},\left\|\cdot\right\|_r\right)$的映射$i:\mathcal{V}\rightarrow\mathcal{V},i\left(\mathbf{a}\right)=r^{-1}\mathbf{a},\quad\forall\mathbf{a}\in\mathcal{V}$是一个等距变换,其中$\left\|\cdot\right\|_r=r\left\|\cdot\right\|,r\in\mathbb{F}$。
    \end{enumerate}
\end{example}

除了例\ref{exp:II.3.2}之外,在欧几里得空间中,对几何对象的平移、旋转、镜象这几种操作,都属于等距变换,因为在这些操作前后,任意两点间的距离是不变的。这几种操作都有个特点:重复一种操作多次后的结果,仍然属于这一种操作。比如平移两次之后的结果还是相当于一个平移;旋转两次之后的结果还是相当于一个旋转。在数学上,我们用“群”的概念来更加精确地描述这类情况。

\begin{definition}[群]\label{def:II.3.3}
    设$G$是一个非空集合,若为$G$的元素规定一个二元运算(binary operation),记为$x\circ y,\quad x,y\in G$\footnote{跟复合映射的符号一样,可算是一种推广意义的使用,因为映射的复合就是一种二元运算。},且该运算具有以下性质:
    \begin{enumerate}
        \item 封闭性:$x\circ y\in G,\quad\forall x,y\in G$
        \item 结合律:$\left(x\circ y\right)\circ z=x\circ\left(y\circ z\right),\quad\forall x,y,z\in G$
        \item 单位元:$\exists e\in G,\quad e\circ x=x\circ c=x,\quad\forall x\in G$
        \item 逆元:$\forall x\in G,\exists x^{-1}\in G,xx^{-1}=x^{-1}x=e$
    \end{enumerate}
\end{definition}

类似定义\ref{def:II.3.3}的定义方式,我们在向量空间的定义\ref{def:II.2.1}中已经领教过了。易证,单位元和逆元总是唯一的,这跟向量空间的情况类似。一般地,群的定义不包括交换律。满足交换率的群叫\emph{交换群(commutative group)}。交换群的二元操作比较像平时的“加法”。

\begin{example}
    设$\mathcal{V}$是数域$\mathbb{F}$上的$n$维向量空间,$\mathcal{V}$上的所有\emph{可逆}算符的集合,基于算符的复合操作,形成一个群。我们称这个群为数域$\mathbb{F}$上的$n$维\emph{一般线性群(generalized linear group)},记作$GL\left(n,\mathbb{F}\right)$;在所讨论的数域维持不变时,简记为$GL\left(n\right)$。
\end{example}

如果一个等距变换的值域与定义域是同一个度量空间,我们可以称其为这个度量空间上的等距变换\footnote{按定义,这样的等距变换都是满射。另外易验等距变换都是单射,故一个度量空间上的等距变换都是双射。}。一个度量空间上的所有等距变换的集合,加上映射的复合运算,形成一个群,正式定义如下——

\begin{definition}[等距群]\label{def:II.3.4}
    由一个度量空间$\left(\mathcal{E},d\right)$到其自身的所有等距变换的集合:
    \[\mathcal{I}=\left\{i:\mathcal{E}\rightarrow\mathcal{E}|i\text{是等距变换}\right\}\]
    加上映射的复合操作$i_1\circ i_2,\quad i_1,i_2\in\mathcal{I}$,形成一个交换群,称为\emph{等距群(isometry group)}。
\end{definition}

在能作上述定义之前,当然需要先验证为什么所称的集合是一个交换群。因此我们要按照定义\ref{def:II.3.3}检查以下条件是否满足:
\begin{enumerate}
    \item 封闭性:$i_1\circ i_2\in\mathcal{I},\quad\forall i_1,i_2\in\mathcal{I}$
    \item 结合律:$i_1\circ \left(i_2\circ i_3\right)=\left(i_1\circ i_2\right)\circ i_3,\quad\forall i_1,i_2,i_3\in\mathcal{I}$
    \item 单位元:恒等映射$\mathrm{id}_\mathcal{E}\in\mathcal{I}$满足$\mathrm{id}_\mathcal{E}\circ i=i\forall i\in\mathcal{I}$。
    \item 逆元:对任一$i\in\mathcal{I}$存在唯一$i^{-1}\in\mathcal{I}$满足$i^{-1}\circ i=\mathrm{id}_\mathcal{E}$
    \item 交换律:$i_1\circ i_2=i_2\circ i_1,\quad\forall i_1,i_2\in\mathcal{I}$

          易验,以上条件都满足\footnote{其中“逆元”的唯一存在性,基于等距变换的双射性。严格的说这来自等距变换表示定理的推论。},所以$\mathcal{I}$是一个交换群。现在我们为$\mathcal{I}$添加规定,把它补成一个实数域$\mathbb{R}$上的向量空间——

    \item 标量乘法:$\alpha i\in\mathcal{V}$;$\alpha\left(\beta i\right)=\left(\alpha\beta\right)i$;$1i=i,\forall\alpha,\beta\in\mathbb{R},i\in\mathcal{V}$
    \item 分配律:$\alpha\left(i_1\circ i_2\right)=\left(\alpha i_1\right)\circ\left(\alpha i_2\right)$;$\left(\alpha+\beta\right)i=\left(\alpha i\right)\circ\left(\beta i\right),\forall\alpha,\beta\in\mathbb{R},i,i_1,i_2\in\mathcal{V}$

          则$\mathcal{V}$是一个向量空间。如果我们再定义一个欧几里得范:

    \item $\left\|i\right\|^2=\left(i|i\right)= d^2\left(x,y\right),\forall x,y\in\mathcal{E}$其中$i$满足$i\left(x\right)=y$。

          则$\mathcal{V}$就是一个赋范内积空间,且该范为欧几里得范。
\end{enumerate}

注意,条件1~5是可验证在所有等距变换上满足的。但条件6~8则是我们新增的要求,没有验证是否所有等距变换都满足,所以只能说$\mathcal{V}$是$\mathcal{I}$的子集。已证明,任一度量空间上的等距群最多只有一个满足上述要求向量空间子群\cite{Noll1974}。每个度量空间,都可经过上述的流程为其装配一个向量空间。我们把一个度量空间以及通过其上的等距群得到的向量空间一同称为一个欧几里得空间,详见下一节。
\end{document}