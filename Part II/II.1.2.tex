\documentclass[../main.tex]{subfiles}
\begin{document}
\section{关系}
一个集合的元素可与另一个集合的元素形成对应关系。例如,设$A$是所有成年公民的集合,那么“婚姻”就是定义在$A$的任意两个不同元素之间的关系。每对夫妻都是一个有序对$\left(a,b\right)\in A\times A$。在实际社会生活中,我们是先用其他概念对婚姻关系进行定义(例如当地的《婚姻法》),再辨别任意两个公民之间是否具有婚姻关系的。但是在集合论中,我们没有其他超出集合论的其他概念以供我们独立地定义元素间的一种关系。我们只能视符合某关系的\emph{所有}有序对的集合为这一关系的定义。例如,我们不采用既有的《婚姻法》来定义何谓婚姻关系,而是把所有具有婚姻关系的公民对全部列出来组成一个集合,作为关于“何谓婚姻关系”的一种完整的界定。要辩认$a, b\in A$是否婚姻关系,就只看有序对$\left(a,b\right)$是否属于上述集合。这种定义关系的方法才是集合论可以普适地采用的。

正式地,若集合$R$的元素都是有序对,则集合$R$就是一个\emph{关系(relation)}。若有序对$\left(x,y\right)\in R$,则记为$xRy$。习惯上,一般的关系常用符号“$\sim$”表示,各种特殊的关系会用特定的符号表示。设$\sim$是一个关系,若$\left(x,y\right)\in \sim$,则$x\sim y$;若$\left(x,y\right)\notin\sim$,则记为$x\not\sim y$。关系的定义告诉我们:

\begin{figure}[htbp]
    \centering
    \includegraphics{../images/relation.pdf}
    \caption{图中展示了一个关系$\sim$。箭头表示一个有序对中两个元素的先后次序。}
    \label{fig:II.1.1}
\end{figure}

\begin{enumerate}
    \item 关系$\sim$是一个有序对的集合。图\ref{fig:II.1.1}的箭头表示法可协助我们把“有序对的集合”联系到“关系”一词的日常意义。
    \item 任一关系$\sim$总可以写成两个集合的笛卡尔积的子集。证明的方法:验证关系$\sim$至少可以是下列笛卡尔积
          \[
              \left(\bigcup_{X\in\bigcup_{X^\prime\in\sim}X^\prime}X\right) \times\left(\bigcup_{X\in\bigcup_{X^\prime\in\sim}X^\prime}X\right)
          \]
          的子集\footnote{提示:回顾有序对的定义$\left(a,b\right)\equiv\left\{\left\{a\right\},\left\{a,b\right\}\right\}$,把表达式
              \[
                  \bigcup_{X\in\bigcup_{X^\prime\in\sim}X^\prime}X
              \]
              所形成的集合写出来,可以发现它是关系$\sim$的所有有序对中的元素的集合。读者可以尝试以图\ref{fig:II.1.1}的例子写出图中关系的$\bigcup_{X\in\bigcup_{X^\prime\in\sim}X^\prime}X$;它就是$\left\{A,B,C,D,E,F,G,H,I,J\right\}$。}
    \item 关系的元素未必是同一个集合与其自身的笛卡尔集。两个不同集合$X$与$Y$之间也可以定义某关系$\sim\subset X\times Y$。只要$x\in X,y\in Y,\left(x,y\right)\in\sim$,则$x\sim y$。若$\sim\subset X\times X$,则称“$\sim$是集合$X$上的关系”;若$\sim\subset X\times Y$,则称“$\sim$是从集合$X$到集合$Y$的关系”。
\end{enumerate}

\begin{example}
    设$A=\left\{a,b\right\}, B=\left\{1,2\right\}$,则$A\times B=\left\{\left(a,1\right),\left(a,2\right),\left(b,1\right),\left(b,2\right)\right\}$。设关系$\sim=\left\{\left(a,1\right),\left(b,2\right)\right\}$。我们不难留意到,$\sim\subset A\times B$。按照关系$\sim$的定义,我们可以写$a\sim 1,a\not\sim 2$。

    留意到,$A\times B$本身就是一个关系。若$\sim=A\times B$,则$A$的任一元素与$B$的任一元素之间都有$\sim$关系,即$\forall a\in A\forall b\in B,a\sim b$。

    等于“$=$”关系是任一集合与其自身的笛卡积的子集。具体地,设$X$是一个非空集合,则$X\times X$中所有满足$x=y$的有序对$\left(x,y\right)$的集合就是等于关系。

    属于“$\in$”也是一个关系。具体地,它是$X\times\mathcal{P}\left(X\right)$满足$x\in A$的所有有序对$\left(x,A\right)$的集合。

    空集是有序对的集合(因为空集是集合,且空集不含有任何不是有序对的元素),因此空集也可以是一个关系。
\end{example}

接上列的第2条,给定一个关系$\sim$,记$U_\sim\equiv\bigcup_{X\in\bigcup_{X^\prime\in\sim}X^\prime}X$,遵循分类公理所构建的集合
\[
    \left\{a|a\in U_\sim\wedge\left(\exists b,b\in U_\sim\wedge a\sim b\right)\right\}
\]
为关系$\sim$的\emph{定义域(domain)},记作$\mathrm{dom}\sim$\footnote{式中的记法“$\exists b,\left(\text{关于$b$的语句}\right)$表示“存在符合关于$b$的语句的一个$b$”。例如“$\exists b,b\in U_\sim$”表示“存在一个属于集合$U_\sim$的$b$”。}。集合
\[
    \left\{b|b\in U_\sim\wedge \left(\exists a,a\in U_\sim\wedge a\sim b\right)\right\}
\]
为关系$\sim$的\emph{值域(range)},记作$\mathrm{ran}\sim$。图\ref{fig:II.1.1}给出了所示关系的定义域和值域。不难留意到,对于任一集合上的等于关系,有$\left(\mathrm{dom}=\right)=\left(\mathrm{ran}=\right)$;对于任一集合上的属于关系,若$\left(\mathrm{dom}\in\right)=X$,则$\left(\mathrm{ran}\in\right)=\mathcal{P}\left(X\right)\setminus\left\{\emptyset\right\}$\footnote{因为$\emptyset\in\mathcal{P}\left(X\right)$,但没有元素能够属于$\emptyset$。}。

\begin{definition}\label{def:II.1.2}
    设$\sim$是集合$X$上的一个关系。若
    \begin{enumerate}
        \item 对任一$x\in X$都有$x\sim x$,则称关系$\sim$是\emph{自反的(reflextive)}。
        \item 对任意$x\in X$和$y\in X$,只要$x\sim y$就有$y\sim x$,则称关系$\sim$是\emph{对称的(symmetric)}。
        \item 对任意$x\in X$、$y\in X$和$z\in X$,只要$x\sim y\text{且}y\sim z$就有$x\sim z$,则称关系$\sim$是\emph{传递的(transitive)}
    \end{enumerate}
    若集合$X$上的一个关系$\sim$同时满足上述3个性质,则称$\sim$是$X$上的一个\emph{等价关系(equivalent relation)}。
\end{definition}

以图\ref{fig:II.1.1}为例,如果图\ref{fig:II.1.1}中的每个元素都有一个从自己回到自己的箭头,那么图中的关系就是自反的;如果图\ref{fig:II.1.1}中的所有箭头都是双向箭头,那么图中的关系就是对称的。读者可尝试把图\ref{fig:II.1.1}的关系修改成非自反、非对称,但传递的关系\footnote{注意这几条定义所要求的“对任意……”,只要有一个例外就可以失效。}。

\begin{example}
    设$X=\left\{a,b,c\right\}$。$X$上的等于关系是$X$上的等价关系。它是集合
    \[
        \left\{\left(a,a\right),\left(b,b\right),\left(c,c\right)\right\}
    \]
    这一集合共有3个元素。同时,$X\times X$也是$X$上的等价关系,它有$2^3=8$个元素。

    一般地,任一非空集合$X$上的等于关系是$X$上的(除空集外)“最小”的等价关系,$X\times X$是$X$上的最大等价关系。

    不等于$\neq$只满足对称性。

    任意集合的包含关系具有自反性和传递性,但集合的包含关系没有对任意集合均成立的对称性。事实上,在上一节关于外延公理的段落中已经介绍过,具有对称性的包含关系就是集合的等于关系。

    显然,“婚姻”不是“全体成年公民”集合上的等价关系。“婚姻”只满足对称性。
\end{example}

此时我们回过头讨论集合论最基本的关系——从属关系的自反性、对称性和传递性。从属关系的自反性是指,一个集合$A$属于它自己,$A\in A$。从属关系的对称性是指,若$A\in B\equiv B\in A$。从属关系的传递性是指,$A\in B \wedge B\in C\Rightarrow A\in C$。直觉上,这三种性质都不令人舒适。但它们未必需要被禁止。若想禁止上述性质,需要\emph{正则公理(axiom of regularity)}:给定任一非空集合$X$,则$X$中必含有一个元素$y\in x$满足$y\cup X=\emptyset$,用符号表述为$X\neq \emptyset\Rightarrow\exists y\left(y\in X\wedge y\cup X=\emptyset\right)$。正则公理同时禁止了三种性质(证明从略)。本讲义所使用的集合论遵循正则公理。

\begin{figure}[htbp]
    \centering
    \includegraphics[width=0.5\textwidth]{../images/partition.pdf}
    \caption{集合的划分与集合上的等价关系}
    \label{fig:II.1.2}
\end{figure}

回到等价关系的讨论。如果集合$X$的非空子集的集合$\mathcal{C}$满足$\bigcup_{Y\in\mathcal{C}}Y=X$且$\mathcal{C}$的元素两两不相交,则称集合$\mathcal{C}$是$X$的一个\emph{划分(partition)}。换言之,如果$X$的若干个非空子集两两不相交,但它们的并集又恰好得到$X$,那么这些子集就好像对集合$X$进行“切蛋糕”所得到结果(如图\ref{fig:II.1.2}右下的情况)。

若集合$\mathcal{C}$是集合$X$的一个划分,我们可以由此定义一个关系$\sim\equiv X/\mathcal{C}$\footnote{该记法与刚刚介绍完的$X/\sim$无关,是符号“$/$”的滥用。},使得当且仅当$X$的元素$x,y$属于$\mathcal{C}$的同一个元素时,$x\sim y$。正式地,
\begin{equation}\label{eq:relation_induced_by_partition}
    X/\mathcal{C}=\left\{\left(x,y\right)|\left(x,y\right)\in X\times X\wedge\left(\exists A\in \mathcal{C},\left\{x,y\right\}\subset A\right)\right\}
\end{equation}
图\ref{fig:II.1.2}中,将右下所示的集合的划分中同属一个子集的元素两两连线,就能得到左上所示的关系。这一关系在给定划分$\mathcal{C}$下的唯一性由外延公理保证。

可以证明,这一关系是等价关系:
\begin{proof}
    验证关系$\sim\equiv X/\mathcal{C}$是等价关系,需一一验证其自反性、对称性和传递性。
    \begin{enumerate}
        \item 自反性:需证明对任意$x\in X$,有序对$\left(x,x\right)\in\sim$。这需要:
              \begin{enumerate}
                  \item $\left(x,x\right)\in X\times x$。由$x\in X$这显然满足。
                  \item $\exists A\in\mathcal{C},x\in A$。由并集的定义(式\eqref{eq:set_union}),$\mathcal{C}$作为一个集合的集合,对任一$x\in\bigcup_{A\in\mathcal{C}}A$必存在$A\in\mathcal{C}$满足$x\in A$。现在$\mathcal{C}$是集合$X$的一个划分,即$\bigcup_{A\in\mathcal{C}}A=X$,故对任一$x\in X$必存在$A\in\mathcal{C}$满足$x\in A$。自反性证毕。
              \end{enumerate}
        \item 对称性:需证明$\left(x,y\right)\in\sim\Leftrightarrow\left(y,x\right)\in\sim$,其中$x,y\in X,x\neq y$。显然,由于$x,y\in X$,$\left(x,y\right)\in X\times X$且$\left(Y,x\right)\in X\times X$。若$\left(x,y\right)\in\sim$,则由$\sim$的定义$\exists A\in\mathcal{C}\left\{x,y\right\}\subset A$,故自然有$\left(y,x\right)\in\sim$;反之亦然。对称性证毕。
        \item 传递性:需证明$\left(x,y\right)\in\sim\wedge\left(y,z\right)\in\sim\Rightarrow\left(x,z\right)\in\sim$,其中$x,y,z\in X,x\neq y,x\neq z,y\neq z$。显然,由于$x,z\in X$,$\left(x,z\right)\in X\times X$。由$\sim$的定义,$x\sim y\Rightarrow\exists A\in\mathcal{C},\left\{x,y\right\}\subset A$,$y\sim z\Rightarrow\exists A^\prime\mathcal{C},\left\{y,z\right\}\subset A^\prime$。由于$\mathcal{C}$是$X$的一个划分,由划分的定义,要么$A=A^\prime$,要么$A\cap A^\prime=\emptyset$,故$A=A^\prime$,即$\left\{x,y,z\right\}\subset A$。再由关系$\sim$的定义有$x\sim z$。传递性证毕。
    \end{enumerate}
\end{proof}
因此我们称式\eqref{eq:relation_induced_by_partition}定义的等价关系是\emph{由集合$X$的划分$\mathcal{C}$引出的等价关系(equivalent relation induced by the partition $\mathcal{C}$ of $X$)}。

上一段介绍的是图\ref{fig:II.1.2}右下到左上的定理,下面我们将介绍相当于图\ref{fig:II.1.2}中从左上到右下的定理。

设关系$\sim$是集合$X$上的一个等价关系,则集合$\left\llbracket x\right\rrbracket_\sim\equiv\left\{y|y\in X\wedge\left(\exists x\in X,y\sim x\right)\right\}$称$x$关于$\sim$的\emph{等价类(equivalent class)}\footnote{“类”与“集合”在概念上无实质区别。}。$X$的元素关于$\sim$的所有等价类的集合,记作$X/\sim$\footnote{注意与相对补集的符号相区别。},称为集合$X$在等价关系$\sim$下的\emph{商集(quotient set)}\footnote{正式地,
    \[
        X/\sim\equiv\left\{A|A\in\mathcal{P}\left(X\right)\wedge\left(\forall x\in A,\left\llbracket x\right\rrbracket_\sim = A\right)\right\}
    \]
}。如图\ref{fig:II.1.2}左上的情况所示,在一个集合上定义了等价关系。每个元素,都能通过这一等价关系的传递性联系若干个共同关联的元素,而形成$X$的一个子集。每个这样的子集,都是$X$关于这一等价关系的等价类。

\begin{theorem}[等价关系基本定理]
    设$\sim\subset X\times X$是集合$X$上的一个等价关系,则$X$在$\sim$下的商集$S/\sim$是$S$的一个划分。
\end{theorem}
\begin{proof}
    根据划分的定义,要使$S/\sim$是$S$的一个划分,以下3条必须同时满足:
    \begin{enumerate}
        \item $S/\sim$的元素都不是空集,即$\forall\left\llbracket x\right\rrbracket_\sim\in S/\sim,\left\llbracket x\right\rrbracket_\sim\neq\emptyset$;
        \item $S/\sim$的元素两两不交,即$\left\llbracket x\right\rrbracket_\sim\neq\left\llbracket y\right\rrbracket_\sim\Leftrightarrow\left\llbracket x\right\rrbracket_\sim\cap\left\llbracket y\right\rrbracket_\sim=\emptyset$;
        \item 所有$S/\sim$的元素并集得到集合$S$,即$\bigcup_{Y\in S/\sim}Y=S$。
    \end{enumerate}
    具体证明过程暂略\footnote{\href{https://proofwiki.org/wiki/Fundamental_Theorem_on_Equivalence_Relations}{证明过程}}。
\end{proof}
\begin{corollary}
    由$X/\sim$引出的等价关系就是$\sim$。
\end{corollary}
\begin{proof}
    证明过程是十分直接的,暂略。提示:利用外延公理,即集合相等的概念。
\end{proof}

由等价关系基本定理及其推论,我们可以写
\[
    \sim=X/\mathcal{C}\Leftrightarrow \mathcal{C}=X/\sim
\]
可见,符号$/$的用法使得等价关系、划分和商集之间有类似“集合的除法”的意义(故称“商集)。
\end{document}