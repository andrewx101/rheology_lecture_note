\documentclass[main.tex]{subfiles}
% 向量空间
% 大一线性代数已经介绍过的知识:
% 1. 数域F^n的记法和意思。但没有介绍域是什么
% 2. 第四章定义2.6定义了“向量空间F^n”,这里“向量”本身是默认为F上的数组
% 3. 最后一章才定义了抽象的n维向量空间,并给了各种例子,还是比较完整。但称作“线性空间”。
% 4. 书把“线性变换”基本介绍了,还介绍了核空间、维数定理等等。
% 为了强调对“集合”、“映射”的反复熟练,尽管大一线性代数已介绍过很多内容,但本讲义希望重述这部分内容,用读者已经熟悉的材料练习“集合与映射”。这一点要明确出来。在此意义下,向量空间的抽象只是“用集合与映射重述”的其中一个自然后果而已,而并非品味选择。



\begin{document}
% 添加“域”的简单介绍,可参考Hoffman & Kunze sec: 1.1

为一个集合添加一些性质和运算法则,可以形成不同类型的空间,乃至不同的数学分支。在本节中我们介绍其中一种空间——向量空间(又称线性空间\cite{周胜林2012线性代数})。

\begin{definition}[向量空间]\label{def:II.2.1}
    给定一个数域$\mathbb{F}$和一个集合$\mathcal{V}$,如果它们满足:
    \begin{enumerate}
        \item $\mathcal{V}$的元素的加法运算。$\forall\mathbf{a},\mathbf{b},\mathbf{c}\in \mathcal{V}$:
              \begin{enumerate}
                  \item 封闭性:$\mathbf{a}+\mathbf{b}\in\mathcal{V}$
                  \item 交换律:$\mathbf{a}+\mathbf{b}=\mathbf{b}+\mathbf{a}$
                  \item 结合律:$\left(\mathbf{a}+\mathbf{b}\right)+\mathbf{c}=\mathbf{a}+\left(\mathbf{b}+\mathbf{c}\right)$
                  \item 恒等元素:$\exists \bm{0}\in\mathcal{V}:\mathbf{a}+\bm{0}=\mathbf{a}$
                  \item 逆:$\forall \mathbf{a}\in\mathcal{V},\exists -\mathbf{a}:\mathbf{a}+\left(-\mathbf{a}\right)=\bm{0}$
              \end{enumerate}
        \item $\mathbb{F}$与$\mathcal{V}$的元素间的乘法运算。$\forall\alpha,\beta\in\mathbb{F},\mathbf{a},\mathbf{b}\in\mathcal{V}$:
              \begin{enumerate}
                  \item $\alpha\mathbf{a}\in\mathcal{V}$
                  \item $\alpha\left(\beta\mathbf{a}\right)=\left(\alpha\beta\right)\mathbf{a}$
                  \item $1\mathbf{a}=\mathbf{a}$
                  \item $\alpha\left(\mathbf{a}+\mathbf{b}\right)=\alpha\mathbf{a}+\alpha\mathbf{b}$
                  \item $\left(\alpha+\beta\right)\mathbf{a}=\alpha\mathbf{a}+\beta\mathbf{a}$
              \end{enumerate}
    \end{enumerate}
    则称$\mathcal{V}$是数域$\mathbb{F}$上的向量空间(vector space);$\mathcal{V}$的元素$\mathbf{a},\mathbf{b},\cdots$是向量(vector);$\mathbb{F}$中的数是$\mathcal{V}$的标量(scalar)。
\end{definition}

在定义\ref{def:II.2.1}中,数域$\mathbb{F}$可以是实数域$\mathbb{R}$或复数域$\mathbb{C}$,它是关于标量乘的运算规定中的标量所属的数域。任何非空集合,只要具备定义\ref{def:II.2.1}的运算法则,就是一个向量空间,其元素就是向量。因此,向量是抽象的一般概念。大一的《线性代数与解析几何》课本中的“行/列向量”和“矩阵”,都只是特例。

读者可尝试利用定义证明$0\mathbf{a}=\mathbf{0}\forall\mathbf{a}\in\mathcal{V}$。

\begin{example}\label{exp:II.2.1}
    以下是一些向量空间的例子。
    \begin{enumerate}
        \item $\mathbb{R}^n$是所有有序实数$n$元组$\left(a_i\right)=\left(\alpha_1,\cdots,\alpha_n\right),\alpha_i\in\mathbb{R},i=1,\cdots,n$的集合。若$\forall \left(a_i\right),\left(b_i\right)\in\mathbb{R}^n$:
              \begin{enumerate}
                  \item $\left(a_i\right)+\left(b_i\right)=\left(\alpha_1+\beta_1,\cdots,\alpha_n+\beta_n\right)$;
                  \item $\alpha\left(a_i\right)=\left(\alpha\alpha_1,\cdots,\alpha\alpha_n\right),\forall\alpha\in\mathbb{R}$;
                  \item $\left(0\right)=\left(0,\cdots,0\right)$;
                  \item $-\left(a_i\right)=\left(-\alpha_1,\cdots-\alpha_n\right)$;
              \end{enumerate}
              则$\mathbb{R}^n$连同上述的运算规定形成数域$\mathbb{R}$上的一个向量空间,又称为实坐标空间。符号$\mathbb{R}$或$\mathbb{C}$既可表示数域,又可表示一个一维实或复坐标空间。可以验证,一维实坐标空间$\mathbb{R}$不是数域$\mathbb{C}$上的向量空间。
        \item 数域$\mathbb{F}$上的所有$m\times n$矩阵的集合$\mathbb{F}^{m\times n}$(连同矩阵的加法和矩阵的标量乘法规定\footnote{见\cite{周胜林2012线性代数}\S 2.1矩阵与矩阵的运算})是一个向量空间。其零向量是$m\times n$全零矩阵。
        \item 在开区间$\left(a,b\right)$上的所有实值一元连续函数的集合是实数域上的向量空间。
        \item 验证:记$\mathcal{C}^n\left(a,b\right)$为开区间$\left(a,b\right)$上的所有$n$阶连续可导实值一元函数的集合,它是实数域上的向量空间。
        \item 验证:记$\mathcal{C}^\infty\left(a,b\right)$为开区间$\left(a,b\right)$上的所有光滑实值一元函数的集合,它是实数域上的向量空间。
    \end{enumerate}
\end{example}

接下来,我们将逐步发现,定义\ref{def:II.2.1}中的性质和运算法则将进一步导致向量空间有一定的维数。由于“封闭性”的要求,一个向量空间内的任一向量总能被这一向量空间中的其他向量按所规定的运算法则表达出来。具体地,若$\mathcal{V}$是数域$\mathbb{F}$上的向量空间,$\mathbf{a}_1,\cdots,\mathbf{a}_n\in\mathcal{V}$,$\alpha_1,\cdots,\alpha_n\in\mathbb{F}$,则依据向量空间的封闭性,$\sum_{i=1}^n\alpha_i\mathbf{a}_i$也属于$\mathcal{V}$,即这一求和的结果也是一个向量,该向量可以用前面这个求和表达式来表达。由此可引入如下定义。

\begin{definition}[线性组合、线性表出、线性无关]\label{def:II.2.2}
    若$\mathcal{V}$是$\mathbb{F}$上的向量空间,$\mathbf{a}_1,\cdots,\mathbf{a}_n\in\mathcal{V}$。若$\alpha_1,\cdots,\alpha_n\in\mathbb{F}$,则$\sum_{i=1}^n\alpha_i\mathbf{a}_i$称为这$n$个向量$\left\{\mathbf{a}_1,\cdots,\mathbf{a}_n\right\}$的线性组合(linear combination)。令$\mathbf{b}=\sum_{i=1}^n\alpha_i\mathbf{a}_i$,则称$\mathbf{b}$被$\left\{\mathbf{a}_i\right\}$线性表出\footnote{集合$\left\{\mathbf{a}_1,\cdots,\mathbf{a}_n\right\}\subset\mathcal{V}$可写为$\left\{\mathbf{a}_i\right\}_{i=1}^n$或$\left\{\mathbf{a}_i\right\}$,$\left\{\mathbf{a}_i\right\}$的一个有序序列则记为$\left(\mathbf{a}_i\right)$。}。若$\sum_{i=1}^n\alpha_i\mathbf{a}_i=\bm{0}$当且仅当$\alpha_i=0\forall i$,则称向量$\left\{\mathbf{a}_i\right\}$线性无关(linear independent)。反之,若存在某不全为零的一组$\left\{\alpha_i\right\},\alpha_i\in\mathbb{F},i=1,\cdots,n$使得$\sum_{i=1}^n\alpha_i\mathbf{a}_i=\bm{0}$则称向量$\left\{\mathbf{a}_i\right\}$线性相关(linear dependent)。
\end{definition}

由定义\ref{def:II.2.2}易得如下结论(证明略):
\begin{enumerate}
    \item 任何真包含一组线性无关向量的向量集合是线性相关的\cite[定理3.1、3.2, p.~98]{周胜林2012线性代数}。
    \item 任何线性无关向量组的子集也是线性无关向量组\cite[\S7.2“(2)”,p.~171]{周胜林2012线性代数}。
    \item 任何含有$\mathbf{0}$向量的向量组线性相关,因为总有$1\neq 0$使得$1\mathbf{0}=\mathbf{0}$。
    \item 一个向量组$S$是线性无关向量组当且仅当$S$的所有子集都是线性无关向量组。
\end{enumerate}

向量空间定义中的封闭性要求保证了向量可类似于我们习惯的数字那样被用作数学表达和运算。因此,封闭性是一个很重要的性质。那么,一个向量空间之内,会不会有一部分子集本身就满足了封闭性呢(就好像复数与实数之间的关系)?我们通过考察$\mathbb{C}^n$和$\mathbb{R}^n$可以举出很多正面的例子。一般地,如果一个向量空间的子集本身也满足封闭性,那么它自己也是一个向量空间(即满足定义\ref{def:II.2.1})。

\begin{definition}[子空间]\label{def:II.2.3}
    令$\mathcal{V}$是数域$\mathbb{F}$上的一个向量空间,如果$\mathcal{V}$的子集$\mathcal{W}\subseteq\mathcal{V}$也是一个向量空间(并与$\mathcal{V}$采用相同的加法和标量乘定义),则称$\mathcal{W}$是$\mathcal{V}$的一个子空间(subspace)。
\end{definition}

按照这一定义,易证$\mathcal{W}$的任意两个向量$\mathbf{a},\mathbf{b}\in\mathcal{W}$和任一标量$\alpha\in\mathbb{F}$构成的线性组合$\alpha\mathbf{a}+\mathbf{b}$也在$\mathcal{W}$内\cite[\S 7.1定理1.1,p.169]{周胜林2012线性代数}。

就算向量空间$\mathcal{V}$的某子集$S$因不满足封闭性而成为不了$\mathcal{V}$的子空间,$S$内的向量的所有线性组合表出的向量可以形成一个比$S$更大的集合,且满足封闭性,因而我们可以说由$S$生成了一个$\mathcal{V}$的子空间。

\begin{definition}[线性生成空间]\label{def:II.2.4}
    若$S$是向量空间$\mathcal{V}$的非空子集,即$S\subseteq\mathcal{V},S\neq\emptyset$,那么$S$内的向量的所有线性组合的集合$\mathcal{W}_S$也是一个向量空间,称为由$S$线性生成的子空间(the subspace spanned by $S$)。
\end{definition}

换句话说,$\mathcal{W}_S$中的向量都能由$S$的向量线性表出。一个直接的结论就是$S\subseteq\mathcal{W}_S$,因为一个向量总能被它自己线性表出。

\begin{example}\label{exp:II.2.2}
    \quad
    \begin{itemize}
        \item 易验证,三维实坐标空间$\mathbb{R}^3$的子集$P=\left\{\left(x_1,x_2,x_3\right)|x_1=0,x_2,x_3\in\mathbb{R}\right\}$具有封闭性,因此它是$\mathbb{R}^3$的子空间。
        \item 设$Q$是$\mathbb{R}^3$的子集$\left\{\left(2,1,3\right),\left(1,0,1\right)\right\}$线性生成的子空间,常记为\[Q=\mathrm{span}\left\{\left(2,1,3\right),\left(1,0,1\right)\right\}\]则可验$\left\{\left(7,2,9\right)\right\}\in Q$。
        \item 如果用有序实数三元组表示从原点$\left(0,0,0\right)$出发的矢量,则上例中的$\left(2,1,3\right),\left(1,0,1\right)$的两个矢量不共线(易验它们线性无关),子空间$Q$是由这两个矢量所确定的平面。
    \end{itemize}
\end{example}

从上面的例子看到,一个向量空间与其子空间之间的关系,暗示了某种维度的概念。我们首先可以明确“一个向量空间维数”的一般意义,但需要先引入“基”的概念。

\begin{definition}[基]\label{def:II.2.5}
    如果向量空间$\mathcal{V}$是其子空间$B$的线性生成空间(即$\mathcal{V}=\mathrm{span}B$,且$B$内的所有向量线性无关,则称$B$是$\mathcal{V}$的一组基(basis)。如果$B$含有有限个向量,则称$\mathcal{V}$是有限维向量空间。
\end{definition}

上一定义仅引入了“有限维”的概念,但没有具体涉及到“维数是几”的问题。因为按定义,$\mathcal{V}$内可能可以找出不止一组基,而这些基是否必然都具有相同个数的向量?只有当这个问题的答案是肯定的,我们才能通过把这它们的个数统一定义为$\mathcal{V}$的维数,来使得向量空间具有唯一确定的维数。下面的定理解决了这个问题。

\begin{theorem}\label{thm:II.2.1}
    有限维向量空间的每组基具有相同个数的线性无关向量。
\end{theorem}
\begin{proof}
    此略\cite[“(3)的证明”,p.~171]{周胜林2012线性代数}\cite[\S 2.3,Theorem 4,p.~44]{Hoffman1971}。
\end{proof}

有了这一定理,我们就可以直接把有限维向量空间的维数定义为它的任一组基的向量个数——

\begin{definition}[有限维向量空间的维数]\label{def:II.2.6}
    设$\mathcal{V}$是数域$\mathbb{F}$上的有限维向量空间,它的维数,记为$\mathrm{dim}\mathcal{V}$,是它的任一组基的向量个数。规定:零向量空间(只有一个零向量组成的向量空间)是0维。
\end{definition}

由定理\ref{thm:II.2.1}还可直接得到如下推论,它们的证明与定理\ref{thm:II.2.1}的证明过程很类似,故从略。

\begin{corollary}
    设$\mathcal{V}$是一个有限维向量空间,其维数$\mathrm{dim}\mathcal{V}=n$,则
    \begin{enumerate}
        \item $\mathcal{V}$的任一含有多于$n$个向量的子集都是线性相关向量组\cite[“(3)的证明”,p.~171]{周胜林2012线性代数}。
        \item $\mathcal{V}$的任一向量个数少于$n$的子集都不能线性生成整个$\mathcal{V}$(即这样的子集的线性生成空间总是$\mathcal{V}$的真子集)。
        \item $\mathcal{V}$的任一子空间$\mathcal{W}$的维数不大于$\mathcal{V}$的维数,即$\mathrm{dim}\mathcal{W}\leq\mathrm{dim}\mathcal{V}$;当且仅当$\mathcal{W}=\mathcal{V}$时取等号。
    \end{enumerate}
\end{corollary}

这一推论的第3条的一个例子就是之前的例\ref{exp:II.2.2}。

\begin{example}
    验证以下命题——
    \begin{itemize}
        \item 如果把一维实坐标空间$\mathbb{R}$看作是实数域$\mathbb{R}$上的向量空间,则$\left\{1\right\}$是其一组基,故一维实坐标空间$\mathbb{R}$是实数域$\mathbb{R}$上的一维向量空间。
        \item 如果把一维复坐标空间$\mathbb{C}$看作是实数域$\mathbb{R}$上的向量空间,则$\left\{1,i\right\}$是其一组基,故一维复坐标空间$\mathbb{C}$是实数域$\mathbb{R}$上的二维向量空间。
        \item 如果把一维复坐标空间$\mathbb{C}$看作是复数域$\mathbb{C}$上的向量空间,则$\left\{1\right\}$是其一组基,故一维复坐标空间$\mathbb{C}$是复数域$\mathbb{C}$上的一维向量空间。
    \end{itemize}
\end{example}

到此为止,我们未具体地阐明“向量”是什么,也未具体地规定加法和标量乘法如何进行。“这种抽象性使我们可以把不同的数学对象统一到线性空间这一概念之下。”\cite[p.~167]{周胜林2012线性代数}不过,通过引入“坐标”的概念,我们又使得任一抽象向量都能用一组有序数组来唯一地表示,从而使抽象的向量之间的运算得以由具体的有序数组的运算来代替(就像我们以往在《线性代数》课中所熟悉的那样)。

\begin{definition}[向量在给定有序基下的坐标]\label{def:II.2.7}若$\mathcal{V}$是一个$n$维向量空间,$\mathcal{V}$内的一组线性无关的有序向量序列$\left(\mathbf{a}_i\right)$线性生成整个$\mathcal{V}$,则称这组有序向量序列$B$为$\mathcal{V}$的一组有序基(ordered basis)。由定义\ref{def:II.2.5},任一向量$\mathbf{x}\in\mathcal{V}$均可表达为$\mathbf{x}=\sum_{i=1}^nx_i\mathbf{a}_i$,进而,任一$\mathbf{x}\in\mathcal{V}$在给定有序基下都唯一对应$\mathbb{F}^n$中的一个有序$n$元数组$\left(x_i\right)$\footnote{这里的唯一性可再次参考\cite[“(3)的证明”,p.~171]{周胜林2012线性代数}。},我们称这一有序$n$元数组$\left(x_i\right)$为向量$\mathbf{x}$在有序基$B$下的坐标(coordinate)。
\end{definition}

基的原始定义仅要求基是一个集合,没有有序性的规定。故我们可以书写“$B=\left\{\mathbf{a}_1,\cdots\mathbf{a}_n\right\}$是某向量空间的一组基。但由于集合的元素是无序的,用一组基向量表出任意一个$n$维向量时所使用的$n$个标量若只是一个集合,那么这一标量集合变换不同的顺序去与基向量组合,可以表出不同的向量。因此我们必须给基向量规定顺序,才能说出“一个向量的第$i$个坐标”。换句话说,向量坐标的定义需要基于有序的基向量,而非基向量的一个集合。所以定义\ref{def:II.2.7}才特别增加了“有序”的要求。在本讲义中,我们都有同一个符号来表示作为集合而言的一组基(例如“$B=\left\{\mathbf{a}_1,\cdots,\mathbf{a}_n\right\}$是某向量空间的一组基”)和一组有序基(例如“$B=\left(\mathbf{a}_1,\cdots,\mathbf{a}_n\right)$是某向量空间的一组有序基”)。

在这一定义中,向量$\mathbf{x}$与其在有序基$B$下的坐标$\left(x_1,\cdots,x_n\right)$之间的唯一对应性,可以很方便地证明。沿用定义中的设定,若另有某$\left(y_1,\cdots,y_n\right)\in\mathbb{F}^n$满足$\mathbf{x}=\sum_{i=1}^ny_i\mathbf{a}_i$,其中至少有一$y_k\notin\left\{x_1,\cdots,x_n\right\}$,则$\mathbf{x}-\mathbf{x}=\sum_{i=1}^n\left(x_i-y_i\right)\mathbf{a}_i=\mathbf{0}$且$\left(x_1-y_1,\cdots,x_n-y_n\right)$中有一个数$x_k-y_k\neq 0$,与$\left\{\mathbf{a}_1,\cdots,\mathbf{a}_n\right\}$线性无关矛盾。故$\left(x_1,\cdots,x_n\right)$对$\mathbf{x}$是唯一的。

有了坐标的定义,在给定基下,$n$维向量空间$\mathcal{V}$中的每一个向量就都与$\mathbb{F}^n$中的一个有序$n$元数组形成了一一对应的关系。由$\mathcal{V}$和$\mathbb{F}$的封闭性,没有一个向量不对应一个数组,反之亦然。易验,通过向量的加法和标量乘法所得到的新向量所对应的坐标,就是原向量的坐标按照$\mathbb{F}^n$上的加法和标量乘法运算的结果\footnote{“利用基和坐标可把线性空间的运算变得更具体。”\cite[p.173]{周胜林2012线性代数}。}。但是要注意,$n$维实坐标空间$\mathbb{R}^n$中的一个向量$\mathbf{x}\in\mathbb{R}^n$本身就是一个有序实数$n$元组$\mathbf{x}=\left(x_1,\cdots,x_n\right),x_i\in\mathbb{R},i=1,\cdots,n$。在选定$\mathbb{F}^n$某组有序基$B=\left(\mathbf{e}_1,\cdots,\mathbf{e}_n\right)$下(这些基向量本身也都是有序实数$n$元组),向量$\mathbf{x}$的坐标可能又是另一个不同的有序实数$n$元组$\left(\chi_1,\cdots,\chi_n\right)$。正确的表示是$\mathbf{x}=\sum_{i=1}^{n}\chi_i\mathbf{e}_i$,或称向量$\mathbf{x}$在有序基$B$下的坐标是$\left(\chi_1,\cdots,\chi_n\right)$,但不能写成$\mathbf{x}=\left(\chi_1,\cdots,\chi_n\right)$\footnote{这里的概念区分可参见\cite[例题2.1,p.~173]{周胜林2012线性代数}。}。

在本讲义中,无论是$n$维向量$\mathbf{x}$本身还是其在某基的下的坐标,都一律写成$n\times 1$矩阵(列向量);为方便,在文字段落中表示为$1\times n$矩阵(行向量)的转置,即$\mathbf{x}=\left(x_1,\cdots,x_n\right)^\intercal$。这里的转置可直接按以前在《线性代数》课中学过的意义来理解。但本讲义会对“转置”的概念进行正式的定义。
\end{document}