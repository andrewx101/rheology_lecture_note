\documentclass[zihao=-4,linespread=1.5,heading=true,a4paper,twoside]{ctexart}
\usepackage[utf8]{inputenc}
\pagestyle{empty}
% packages for page layout
\usepackage{geometry}
\geometry{
a4paper,
total={171.8mm,246.2mm},
left=19.1mm,
top=25.4mm,
}

% package for colors
\usepackage{xcolor}

% packages for fonts and symbols
\usepackage{amsmath}
\usepackage{amssymb}
\usepackage{bm}
\usepackage{calrsfs}
\usepackage{stmaryrd}


% packages for graphics
\usepackage{graphicx}
\graphicspath{ {./images/}{./images/} }
\usepackage{wrapfig}
\usepackage{subcaption}
\usepackage{capt-of}
\usepackage{cutwin}

% packages for page layout
\usepackage{geometry}
\geometry{
a4paper,
total={171.8mm,246.2mm},
left=19.1mm,
top=25.4mm,
}
\usepackage{fancyhdr}
\pagestyle{fancy}
\fancyhead{}
\fancyhead[LE,RO]{\leftmark}
\fancyfoot{}
\fancyfoot[LE,RO]{\thepage}
\fancyfoot[LO,RE]{更新至\today}

% package for quotes
\usepackage{csquotes}

% redefine itemize
\usepackage{enumitem}% http://ctan.org/pkg/enumitem
\setlist{nosep}

% package for hyperlinks
\usepackage{hyperref}
\hypersetup{colorlinks=true}

% packages for footnotes
\renewcommand{\thefootnote}{\fnsymbol{footnote}}
\usepackage{perpage}
\MakePerPage{footnote} 

% packages for bibliography
\usepackage[
backend=biber,
style=gb7714-2015,
gbpub=false,
sorting=none
]{biblatex}
\addbibresource{./ref/ref.bib}

% packages for definitions, theorems, proofs.
\usepackage{amsthm}
\theoremstyle{definition}
\newtheorem{definition}{定义}
\newtheorem*{definition*}{定义}
\newtheorem{theorem}{定理}
\newtheorem*{theorem*}{定理}
\newtheorem{lemma}{引理}
\newtheorem{corollary}{推论}[theorem]
\let\oldproof\proof
\renewcommand{\proof}{\color{gray}\oldproof}
\newtheorem{example}{例}




% counter controlling
\usepackage{chngcntr}
\counterwithin{figure}{section}
\counterwithin{equation}{section}
\counterwithin{definition}{section}
\counterwithin{theorem}{section}
\counterwithin{lemma}{section}
\counterwithin{example}{section}
\counterwithin{section}{part}
\setcounter{tocdepth}{2}

% package for subfiles
\usepackage{subfiles}

\usepackage{datetime2}

\title{流变学}
\author{孙尉翔\\mswxsun@scut.edu.cn}
%=====================================================================================================================
\begin{document}

\part*{前言}\label{sec:preface}
\subfile{preface.tex}

\newpage\tableofcontents

\newpage\part{引言}

\newpage\part{数学部分}
\section{集合与映射}\label{sec:II.1}
\subfile{Part II/II.1.1.tex}
\subfile{Part II/II.1.2.tex}

\section{向量空间}\label{sec:II.2}
\subfile{Part II/II.2.tex}

\section{内积空间与赋范向量空间}\label{sec:II.3}
\subfile{Part II/II.3.tex}

\section{线性变换}\label{sec:II.4}
\subsection{线性变换的定义和基本性质}\label{sec:II.4.1}
\subfile{Part II/II.4.1.tex}

\subsection{线性变换的坐标矩阵}\label{sec:II.4.2}
\subfile{Part II/II.4.2.tex}

\subsection{线性变换的转置}\label{sec:II.4.3}
\subfile{Part II/II.4.3.tex}

\section{基变换与坐标变换公式}\label{sec:II.5}
\subfile{Part II/II.5.tex}

\section{内积空间上的线性算符}\label{sec:II.6}
\subfile{Part II/II.6.tex}

\section{线性算符的行列式、迹和特征值}\label{sec:II.7}
\subfile{Part II/II.7.tex}

\section{正规算符及其谱分解}\label{sec:II.8}
\subfile{Part II/II.8.tex}

\section{欧几里得空间}
\subfile{Part II/II.9.tex}

\section{向量函数及其图像}
\subfile{Part II/II.10.tex}

\section{向量函数的极限与连续性}
\subfile{Part II/II.11.tex}

\section{向量函数的微分与导数}
\subfile{Part II/II.12.tex}

\section{曲线、曲面和积分定理}
\subfile{Part II/II.13.tex}

\newpage\part{连续介质力学基础}
\section{标架与参考系}
\subfile{Part III/III.1.tex}

\section{物体的运动}
\subfile{Part III/III.2.tex}

\section{物体的形变}
\subfile{Part III/III.3.tex}

\section{物质描述与空间描述}
\subfile{Part III/III.4.tex}

\section{应变率张量}
\subfile{Part III/III.5.tex}


%\section{应力张量}

%\section{守恒律}

%\section{Navier--Stokes方程}

%\section{流变测量学}

%\newpage\part{线性粘弹性本构方程}
%\section{线性粘弹性本构方程的建立}

%\section{线性粘弹性本构关系的一般预测}

%\section{记忆函数的具体形式}

%\section{松弛时间谱}

%\newpage\part{非线性粘弹性本构方程概览}
%\section{非线性粘弹性本构方程的构建原则}
%\section{准线性粘弹性}
%\section{广义牛顿流体}
%\section{微分形本构方程}
%\section{积分型本构方程}
%\section{屈服应力流体}

\newpage\part{附录}
\section{线性代数部分定理的证明}\label{sec:VI.1}
\subfile{Part VI/VI.1.tex}

\section{向量函数微积分部分的证明}\label{sec:VI.2}
\subsection{$\mathbb{R}^n$空间上的一些拓扑概念}
\subfile{Part VI/VI.2.1.tex}

\subsection{向量函数可微分的必要条件与充分条件}
\subfile{Part VI/VI.2.2.tex}

\subsection{复合函数求导的链式法则}
\subfile{Part VI/VI.2.3.tex}

\subsection{反函数定理和隐函数定理}
\subfile{Part VI/VI.2.4.tex}

\subsection{等距变换的表示定理}
\subfile{Part VI/VI.2.5.tex}

%\section{曲线坐标系}
%\subfile{Part VI/VI.4.tex}



\newpage\part*{参考文献}
\printbibliography[heading=none]
\end{document}