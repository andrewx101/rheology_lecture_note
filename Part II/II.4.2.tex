\documentclass[main.tex]{subfiles}
% 函数的极限与连续性
\begin{document}
我们用“$\epsilon-\delta$语言”来定义函数极限

\begin{definition}[向量函数的极限]\label{def:II.4.5}
    给定函数$\mathbf{f}:\mathbb{R}^n\supseteq D\rightarrow\mathbb{R}^m$,如果对任意正实数$\varepsilon$总存在正实数$\delta$使得只要$0<\left\|\mathbf{r}-\mathbf{r}_0\right\|<\delta,\mathbf{r},\mathbf{r}_0\in D$,就总有$\left\|\mathbf{f}\left(\mathbf{r}\right)-\mathbf{y}_0\right\|<\epsilon,\mathbf{y}_0\in\mathbb{R}^m$,就称$\mathbf{y}_0$是函数$\mathbf{f}\left(\mathbf{r}\right)$在$\mathbf{r}_0$处的\emph{极限(limit)},记为
    \[\lim_{\mathbf{r} \to\mathbf{r}_0} \mathbf{f}\left(\mathbf{r}\right)=\mathbf{y}_0\]
\end{definition}

\begin{theorem}\label{thm:II.4.1}
    函数$\mathbf{f}:\mathbb{R}^n\supseteq D\rightarrow\mathbb{R}^m$在$\mathbf{r}_0\in D$处存在极限$\lim_{\mathbf{r}\to\mathbf{r}_0}\mathbf{f}\left(\mathbf{r}\right)=\mathbf{a},\mathbf{a}=\left(a_i\right)\in\mathbb{R}^m$的充要条件是其坐标函数的极限$\lim_{\mathbf{r}\to\mathbf{r}_0}f_i\left(\mathbf{r}\right)=a_i,\forall i=1,\cdots,m$都存在。
\end{theorem}
\begin{proof}
    由向量函数定义,$\mathbf{f}$与$f_i,i=1,\cdots,m$的定义域都相同。

    若已知$\lim_{\mathbf{r}\to\mathbf{r}_0}\mathbf{f}\left(\mathbf{r}\right)=\mathbf{a}$,即对任一实数$\varepsilon>0$都存在实数$\delta>0$使得只要$0<\left\|\mathbf{r}-\mathbf{r}_0\right\|<\delta$就有$\left\|\mathbf{f}\left(\mathbf{r}\right)-\mathbf{a}\right\|<\epsilon$。那么,取同一对正实数$\varepsilon$和$\delta$,对任一$i\in\left\{1,\cdots,m\right\}$都有
    \begin{align*}
        \left|f_i\left(\mathbf{r}\right)-\alpha_i\right| & \leq\sum_{i=1}^m\left|f_i\left(\mathbf{r}\right)-a_i\right|\quad\text{(当}m=1\text{或}f_j=a_j\forall j\neq i\text{时取等号。)} \\
                                                         & \leq\left|\sum_{i=1}^m\left(f_i\left(\mathbf{r}\right)-\alpha_i\right)\right|\quad\text{(三角不等式。)}                       \\
                                                         & =\left\|\mathbf{f}\left(\mathbf{r}\right)-\mathbf{a}\right\|<\varepsilon
    \end{align*}
    故按照极限的定义,$\lim_{\mathbf{r}\to\mathbf{r}_0}f_i\left(\mathbf{r}\right)=\alpha_i,i=1,\cdots,m$,其中$a_i,i=1,\cdots,n$是$\mathbf{a}$的坐标。

    反之,若极限$\lim_{\mathbf{r}\to\mathbf{r}_0}f_i\left(\mathbf{r}\right)=a_i,i=1,\cdots,m$都存在,即对任一实数$\varepsilon>0$都能找到实数$\delta_i>0,i=1,\cdots,m$,使得只要$f_i$的定义域$D$内一向量$\mathbf{r}$到$\mathbf{r}_0$的距离满足$0<\left\|\mathbf{r}-\mathbf{r}_0\right\|<\delta_i$,就有$\left|f_i\left(\mathbf{r}\right)-a_i\right|<\varepsilon/\sqrt{m}$,那么令$\delta=\mathrm{min}\left\{\delta_1,\cdots,\delta_m\right\}$,则当$0<\left\|\mathbf{r}-\mathbf{r}_0\right\|<\delta$,就有$\mathrm{max}\left\{\left|f_i\left(\mathbf{r}\right)-a_i\right|\right\}<\varepsilon/\sqrt{m}$。

    利用事实\footnote{
        由$\left\|\mathbf{x}\right\|^2=\sum_{i=1}^nx_i^2\leq n\mathrm{max}\left\{x_1^2,\cdots,x_n^2\right\}$得到。
    }
    \[\left\|\mathbf{x}\right\|\leq\sqrt{n}\mathrm{max}\left\{\left|x_1\right|,\cdots,\left|x_n\right|\right\},\forall\mathbf{x}\in\mathbb{R}^n\]
    可得
    \[\left\|\mathbf{f}\left(\mathbf{r}\right)-\mathbf{a}\right\|\leq\sqrt{m}\mathrm{max}\left\{\left|f_i\left(\mathbf{r}\right)-a_i\right|\right\}<\varepsilon\]
    这样的$\delta$是总能取到的,故由函数极限的定义,$\lim_{\mathbf{r}\to\mathbf{r}_0}\mathbf{f}\left(\mathbf{r}\right)=\mathbf{a}$。
\end{proof}

\begin{definition}[函数的连续性]\label{def:II.4.6}
    给定函数$\mathbf{f}:\mathbb{R}^n\supseteq D\rightarrow\mathbb{R}^m$和其定义域中一点$\mathbf{r}_0\in D$。如果极限$\lim_{\mathbf{r}\to\mathbf{r}_0}\mathbf{f}\left(\mathbf{r}\right)=\mathbf{f}\left(\mathbf{r}_0\right)$存在,则称$\mathbf{f}\left(\mathbf{r}\right)$\emph{在$\mathbf{r}_0$处连续(continuous at $\mathbf{r}_0$)}。
\end{definition}

\begin{theorem}\label{thm:II.4.2}
    函数$\mathbf{f}:\mathbb{R}^n\rightarrow\mathbb{R}^m$在$\mathbf{r}_0\in\mathbb{R}^n$处连续的充要条件是其坐标函数在$\mathbf{r}_0$处都连续。
\end{theorem}
\begin{proof}
    由定理\ref{thm:II.4.1}直接得证。
\end{proof}

如果一个函数在其定义域内处处都连续,我们就直接称该函数在其定义域内是连续的,或称该函数是连续函数。以下给出的定理及其推论告诉我们线性变换总是连续的。

\begin{theorem}\label{thm:II.4.3}
    设$\mathcal{V}$、$\mathcal{W}$分别是数域$\mathbb{F}$上的$n,m$维赋范向量空间,$\mathbf{L}:\mathcal{V}\rightarrow\mathcal{W}$是一个线性变换,则总存在正实数$k>0$使得$\left\|\mathbf{Lx}\right\|\leq k\left\|\mathbf{x}\right\|,\forall\mathbf{x}\in\mathcal{V}$,当且仅当$n=1$时取等号。
\end{theorem}
\begin{proof}
    设$\left\{\mathbf{e}_i\right\}$是$\mathcal{V}$的一组基,则任一向量$\mathbf{x}\in\mathcal{V}$可表示为$\mathbf{x}=\sum_{i=1}^nx_i\mathbf{e}_i$。设$\left\|\cdot\right\|_\mathrm{E}$是欧几里得范,即$\left\|\mathbf{x}\right\|_\mathrm{E}\equiv\left(\sum_{i=1}^nx_i^2\right)^\frac{1}{2}$。由定理\ref{thm:II.2.23}及其推论,包括欧几里得范在内的任意范都不依赖基的选择。故在任一基下总有$\left|x_i\right|\leq\left(\sum_{i=1}^nx_i^2\right)^\frac{1}{2}=\left\|\mathbf{x}\right\|_\mathrm{E}$。

    由引理\ref{lem:A.1},对$\mathcal{V}$上的任一范的定义$\left\|\cdot\right\|$,总存在$K>0$使得$\left\|\cdot\right\|_\mathrm{E}\leq K\left\|\cdot\right\|$。故
    \begin{align*}
        \left\|\mathbf{Lx}\right\| & =\left\|\sum_{i=1}^nx_i\mathbf{Le}_i\right\|                                                     \\
                                   & \leq\sum_{i=1}^n\left|x_i\right|\left\|\mathbf{Le}_i\right\|                                     \\
                                   & \leq\sum_{i=1}^n\left\|\mathbf{x}\right\|_\mathrm{E}\left\|\mathbf{Le}_i\right\|                 \\
                                   & \leq\sum_{i=1} K\left\|\mathbf{x}\right\|\left\|\mathbf{Le}_i\right\|=k\left\|\mathbf{x}\right\|
    \end{align*}
    其中$k=K\sum_{i=1}^n\left\|\mathbf{Le}_i\right\|$。上面的第2个不等号用到范的三角不等式和范的调和性,第三个不等号用到证明开头提到的事实,第四个不等号用到引理\ref{lem:A.1}。由于引理\ref{lem:A.1}的等号当且仅当$n=1$时成立,这一条件已经强于其余两个不等号的取等充要条件,故整个不等式的取等充要条件就是$n=1$。
\end{proof}

\begin{corollary}
    线性变换是连续函数。
\end{corollary}
\begin{proof}
    由定理\ref{thm:II.4.3},对任意$\mathbf{x},\mathbf{x}_0\in\mathcal{V}$有$\left\|\mathbf{Lx}-\mathbf{Lx}_0\right\|=\left\|\mathbf{L}\left(\mathbf{x}-\mathbf{x}_0\right)\right\|\leq k\left\|\mathbf{x}-\mathbf{x}_0\right\|$,故对任一$\varepsilon>0$总可取$\delta=\frac{\varepsilon}{k}$使得只要$\left\|\mathbf{x}-\mathbf{x}_0\right\|<\delta$就有$\left\|\mathbf{Lx}-\mathbf{Lx}_0\right\|\leq k\left\|\mathbf{x}-\mathbf{x}_0\right\|<k\delta=\varepsilon$。
\end{proof}

以下极限性质定理,已在本科的高等数学课中教过了。
\begin{theorem}[极限基本性质]\label{thm:II.4.4}
    在实数域内,以下命题成立:
    \begin{enumerate}
        \item 设$a,b$是常数,则$\lim_{x\to a}b=b$。
        \item 若$a$是常数,则$\lim_{x\to a} x=a$。
        \item 设$\lim_{x\to a}f\left(x\right)=L$,$c$是常数,则$\lim_{x\to a}cf\left(x\right)=c\lim_{x\to a}f\left(x\right)=cL$。
        \item 设$\lim_{x\to c}f\left(x\right)=L,\lim_{x\to c}g\left(x\right)=M$,则$\lim_{x\to c}\left[f\left(x\right)+g\left(x\right)\right]=L+M$。
        \item 设$\lim_{x\to c}f\left(x\right)=L,\lim_{x\to c}g\left(x\right)=M$,则$\lim_{x\to c}\left[f\left(x\right)g\left(x\right)\right]=LM$。
        \item (夹逼定理)若$c$是常数,$g\left(x\right)\leq f\left(x\right)\leq h\left(x\right)$至少在点$x=c$之外都成立,且$\lim_{x\to c}g\left(x\right)=\lim_{x\to c}h\left(x\right)=L$,则$\lim_{x\to c}f\left(x\right)=L$。
        \item 设$\lim_{x\to c}f\left(x\right)=L$,则$\left|\lim_{x\to c}f\left(x\right)\right|=\left|L\right|=\lim_{x\to c}\left|f\left(x\right)\right|$。
    \end{enumerate}
\end{theorem}
\end{document}