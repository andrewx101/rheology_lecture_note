\documentclass[../main.tex]{subfiles}
\begin{document}
\subsection{集合的基础概念}
集合是近代数学的基本语言。用集合与映射重述人的直观经验,是近世数学和基于其构建的理论物理的普遍特点。连续介质力学中的大量概念都依赖集合和映射来引入的。如果不熟悉集合和映射的语言和符号,就很难读懂后面的内容。

\begin{definition}\label{def:II.1.1}
\emph{集合(set)}是具有某种特性的事物的整体。构成集合的事物或对象称为\emph{元素(element)}。集合还必须满足:
\begin{itemize}
    \item 无序性:一个集合中,每个元素的地位是相同的,元素之间是无序的
    \item 互异性:一个集合中,任何两个元素都不相同,即每个元素只出现一次
    \item 确定性:给定一个集合及一个事物,该事物要么属于要么不属于该集合,不允许模棱两可。
\end{itemize}
\end{definition}

具体地,若$A$是集合,$x$是$A$的一个元素,则记$x\in A$;若$y$不是$A$的元素,则记为$y\notin A$。记号“$x\in A$”规定了“$A$是集合且$x$是$A$的元素”。

如果只要有$x\in A$就有$x\in B$,则称$A$是$B$的\emph{子集(subset)},或称$A$包含于$B$、$B$包含$A$,记为$A\subset B$。显然,任一集合都是它自己的子集。

如果$A\subset B$且$B\subset A$,则集合$A$就是$B$,记为$A=B$\footnote{此处等号“$=$”的意义应理解为:等号两边的字母是同一集合的不同代号。若写$A=B$,那么$A$\emph{就是}$B$。相应地,不等号“$\neq$”两边的字母是不同集合的代号。若写$A\neq B$,那么$A$\emph{不是}$B$。}。换言之,如果集合$A$与$B$的元素都相同,则$A$与$B$就是同一个集合。再换言之,一个集合由其所有元素\emph{唯一}确定。这是集合论的\emph{外延公理(axiom of extension)}。

如果$A\subset B$且$A\neq B$,则称$A$是$B$的\emph{真子集(proper subset)},或称$A$真包含于$B$、$B$真包含$A$,记作$A\subsetneqq B$。

没有元素的集合称为\emph{空集(empty set)},记作$\emptyset$。空集是唯一的。简要证明:若$\emptyset_1$和$\emptyset_2$都是空集且$\emptyset_1\neq\emptyset_2$,则由空集的上述定义,要么存在$x\in\emptyset_1$且$x\notin\emptyset_2$,要么存在$y\in\emptyset_2$且$y\notin\emptyset_1$;无论哪种情况与$\emptyset_1$和$\emptyset_2$是空集相矛盾。因此要么$\emptyset_1$和$\emptyset_2$有一个不是空集,要么它们是同一个集合,即“空集是唯一的”。

给定一个集合$A$,我们可以根据$A$的元素所需要满足的附加要求,构建出$A$的子集。例如,设$A$是所有偶数的集合,附加的要求是“比1大、比9小”,我们就从$A$中找出了2、4、6、8这四个元素组成的集合$B$。一般地,我们将此记作:
\[
\left\{x|x\in A\wedge\left(\text{$x$需要满足的条件}\right)\right\}
\]
其中符号$\wedge$表示“且”的意思。注意,预先给定一个基本集合$A$这一步原则上是不可省略的;“$\wedge$”后的语句形式是对$A$的元素$x$的规定,而不能是其他形式。这是集合论的\emph{分类公理(axiom of specification)}。

给定两个集合$A$和$B$,总存在唯一一个这样的集合$V$:只要$x\in V$,就有$x\in A$且$x\in B$;反之,若$x\in A$且$x\in B$,则有$x\in V$。集合$V$的存在性来自分类公理;$V$可表示成:
\[
V=\left\{x|x\in A\wedge x\in B\right\}
\]
$V$的唯一性简证如下:若另有一$V^\prime=\left\{x|x\in A\wedge x\in B\right\}$且$V^\prime\neq V$,则必存在$x\in V^\prime$且$x\notin V$。若$x\in V^\prime$,则$x\in A$且$x\in B$;若$x\notin V$,则$x\notin A$或$x\notin B$,这两个推论相互矛盾\footnote{由集合定义\ref{def:II.1.1}中的“确定性”。}。因此$V^\prime=V$或$V^\prime$不存在,即$V$是唯一的。我们把这样的集合$V$称作集合$A$与$B$的\emph{交集(interset)},记作$V=A\cap B$。

设$\mathcal{C}$是集合的集合,且$\mathcal{C}\neq\emptyset$。设$A\in\mathcal{C}$,则由分类公理存在以下集合
\[
    V=\left\{x|x\in A\wedge \left(x\in X\Leftrightarrow X\in\mathcal{C}\right)\right\}
\]
其中$\Leftrightarrow$是“当且仅当”的意思。这一集合$V$的唯一性可类似上一段那样得证,此略\footnote{下文构建的集合的唯一性,不作说明时,都由外延公理保证。}。此时称$V$是$\mathcal{C}$的元素的交集,记作
\[
V=\bigcap_{X\in\mathcal{C}}X
\]

集合的交集有如下性质:
\begin{enumerate}
\item $A\cap\emptyset=\emptyset$
\item $A\cap B=B\cap A$(交换律)
\item $A\cap\left(B\cap C\right)=\left(A\cap B\right)\cap C$(结合律)
\item $A\cap A=A$(幂等)
\item $A\subset B\Leftrightarrow A\cap B=A$
\end{enumerate}

如果集合$A$和$B$的交集是空集,即$A\cap B=\emptyset$,则称$A$与$B$是\emph{不相交的(disjoint)}。比如,我们有时会说,某集合的集合$\mathcal{C}$中的元素\emph{两两不相交(pair-wise disjoint)}。

分类公理只允许我们“收窄”一个给定的集合。以下规定的原则将允许我们从已有集合构建出“更大的”集合。

我们可以把任意两个集合$a$、$b$组成一对,变成一个新的集合,记作$\left\{a,b\right\}$,并\emph{规定}这样的集合可以存在。这是集合论的\emph{配对公理(axiom of paring)}。于是有$a\in \left\{a,b\right\}$和$b\in \left\{a,b\right\}$。特别地,一个集合$a$可与其自身“成对”,得到“$\left\{a,a\right\}$”,但由于集合的元素要满足互异性,故实际所得到的集合应是$\left\{a\right\}$。这种只有一个元素的集合,称为\emph{单元素集(singleton)}。注意理解以下事实:$\emptyset\in\left\{\emptyset\right\}$,$\left\{\emptyset\right\}\neq\emptyset$。

我们可以让若干个集合形成并集。具体地,我们\emph{规定}对任意集合的集合$\mathcal{C}$,总存在一个集合$U$,它含有$\mathcal{C}$的至少一个元素的元素\footnote{$\mathcal{C}$的元素是集合。}。换言之,只要有一个$X\in\mathcal{C}$满足$x\in X$,就有$x\in U$。我们称$U$是$\mathcal{C}$的所有元素的\emph{并集(union)},记作
\[
U=\bigcup_{X\in\mathcal{C}}X
\]
这是集合论的\emph{并集公理(axiom of unions)}。特别地,若$\mathcal{C}=\emptyset$,则$\bigcup_{X\in\mathcal{C}}X=\emptyset$,简单证明:由并集的定义,若存在$a\in\bigcup_{X\in\mathcal{C}}X$,则至少存在一个$X\in\mathcal{C}=\emptyset$满足$x\in X$,但是显然不存在属于空集的元素$X$,因此不存在所述的$a$,即$\bigcup_{X\in\mathcal{C}}X$是空集。

如果$\mathcal{C}$是由两个集合$A$和$B$配对而成,即$\mathcal{C}=\left\{A,B\right\}$,则$\mathcal{C}$的元素的并集常以中缀的记法写成:
\[
\bigcup_{X\in\left\{A,B\right\}}X\equiv A\cup B
\]

集合的并集有如下性质:

\begin{enumerate}
\item $A\cup \emptyset=A$
\item $A\cup B=B\cup A$(交换律)
\item $A\cup\left(B\cup C\right)=\left(A\cup B\right)\cup C$(结合律)
\item $A\cup A=A$(幂等)
\item $A\subset B\Leftrightarrow A\cup B=B$
\end{enumerate}

我们利用并集操作把若干个单元素集结合成一个含有限个元素的集合。例如,
\[
\left\{a,b,c\right\}=\left\{a\right\}\cup\left(\left\{b\right\}\cup\left\{c\right\}\right)=\left(\left\{a\right\}\cup\left\{b\right\}\right)\cup\left\{c\right\}=\bigcup_{X\in\left\{\left\{a\right\},\left\{b\right\},\left\{c\right\}\right\}}X
\]
依此类推,任意有限个元素的集合都可由此方法构建\footnote{至此,我们可以回过头来重新理解集合的定义\ref{def:II.1.1}中的“无序性”、“互异性”和“确定性”。头两个规定,其实是外延公理的推论。例如,若集合$A=\left\{a,b,c\right\}$,$B=\left\{c,b,a\right\}$,则由外延公理$A=B$(无序性)。若集合$A=\left\{a,a\right\}$,$B=\left\{a\right\}$,则由外延公理$A=B$(互异性)。最后,定义\ref{def:II.1.1}中的“确定性”,只是逻辑上排中律的要求。}。



给定集合$A$和$B$,集合$C=\left\{x|x\in A\text{且}x\notin B\right\}$称$B$在$A$中的\emph{相对补集(relative complement of $B$ in $A$)},记为$C=A\setminus B$。注意,此处$B$不必包含于$A$。

我们常在给定一个全集$E$之下讨论其子集在$E$中的相对补集。如果默认这一前提,则可简称任一$E$的子集$A\subset E$在$E$中的相对补集为$A$的补集,记为$A^\complement\equiv E\setminus A$。

给定全集$E$下集合的补集有如下性质:
\begin{enumerate}
\item $\left(A^\complement\right)^\complement=A$
\item $\emptyset^\complement=E$
\item $E^\complement=\emptyset$
\item $A\cap A^\complement=\emptyset$
\item $A\cup A^\complement=E$
\item $A\subset B\Leftrightarrow B^\complement\subset A^\complement$
\end{enumerate}

关于集合的并集、交集,以及给定全集下的补集还有一条重要定律——\emph{德摩根定律(De Morgan's Laws)}:对任意集合$A$、$B$,
\begin{align*}
\left(A\cup B\right)^\complement&=A^\complement\cap B^\complement\\
\left(A\cap B\right)^\complement&=A^\complement\cup B^\complement
\end{align*}

给定一个集合$A$,我们\emph{规定}存在一个集合,暂记作$\mathcal{P}^\prime\left(A\right)$,包含所有$A$的子集。这是集合论的\emph{幂集公理(axiom of powers)}。我们于是可遵守分类公设给出恰好包括$A$的所有子集(包括空集和集合$A$本身),而不包括其他元素的集合:
\[
\mathcal{P}\left(A\right)=\left\{X|X\in\mathcal{P}^\prime\left(A\right)\wedge X\subset A\right\}
\]
我们称$\mathcal{P}\left(A\right)$为集合$A$的\emph{幂集(power set)}。

为什么把这样的集合称为“幂集”呢?若$A=\emptyset$,则$\mathcal{P}\left(A\right)=\left\{\emptyset\right\}$,有$1=2^0$个元素;若$A=\left\{a\right\}$,则$\mathcal{P}\left(A\right)=\left\{\emptyset,\left\{a\right\}\right\}$,有$2=2^1$个元素;若$A=\left\{a,b\right\}$,则$\mathcal{P}\left(A\right)=\left\{\emptyset,\left\{a\right\},\left\{b\right\},\left\{a,b\right\}\right\}$,有$4=2^2$个元素;……

注意到,若$X\in\mathcal{P}\left(E\right)$,则有$X^\complement\in\mathcal{P}\left(E\right)$。于是,默认以$E$为全集时,我们不必逐个讨论$E$的子集的补集。设$\mathcal{C}\subset\mathcal{P}\left(E\right)$,则$\mathcal{C}$中的元素的补集的集合为
\[
\mathcal{D}=\left\{X|X\in\mathcal{P}\left(E\right)\wedge X^\complement\in\mathcal{C}\right\}
\]
这时,德摩根定律有如下更一般的形式:
\begin{align*}
\left(\bigcup_{X\in\mathcal{C}}X\right)^\complement=\bigcap_{X\in\mathcal{C}}X^\complement\\
\left(\bigcap_{X\in\mathcal{C}}X\right)^\complement=\bigcup_{X\in\mathcal{C}}X^\complement
\end{align*}
其中我们引入了以下记法惯例:
\[
\bigcap_{X\in\mathcal{C}}X^\complement\equiv\bigcap_{X\in\mathcal{D}}X, \bigcup_{X\in\mathcal{C}}X^\complement\equiv\bigcup_{X\in\mathcal{D}}X
\]

由集合的定义\ref{def:II.1.1}所要求的无序性,给定两个集合$a$、$b$,它们的配对集合$\left\{a,b\right\}$是不表示顺序的。我们可以用$a$的单元素集$\left\{a\right\}$和$\left\{a,b\right\}$这两个集合,配对得到集合$\left\{\left\{a\right\},\left\{a,b\right\}\right\}\equiv\left(a,b\right)$,来表示\emph{有序对(ordered pair)}。在$\left(a,b\right)$的元素中,$\left\{a,b\right\}$表明我们讨论哪两个元素的有序对,而$\left\{a\right\}$表明哪一个元素放在前面。所以,$\left(a,b\right)\neq\left(b,a\right)$。

给定两个集合$A$、$B$,是否可以给出所有$a\in A$且$b\in B$的有序对$\left(a,b\right)$的集合?我们留意到,对任意$a\in A$和$b\in B$有$\left\{a\right\}\subset A$,$\left\{b\right\}\subset B$,$\left\{a,b\right\}\subset A\cup B$,$\left\{a\right\}\subset A\cup B$。可见,$\left\{a\right\}$和$\left\{a,b\right\}$都是集合$A\cup B$的子集,故$\left(a,b\right)\in\mathcal{P}\left(A\cup B\right)$,$\left(a,b\right)\subset\mathcal{P}\left(\mathcal{P}\left(A\cup B\right)\right)$。换言之,只要$a\in A$且$b\in B$,就有$\left(a,b\right)\subset\mathcal{P}\left(\mathcal{P}\left(A\cup B\right)\right)$。于是我们可以遵守分类公理,把所有$a\in A$且$b\in B$的有序对$\left(a,b\right)$的集合规定出来,且其唯一性由外延公理保证\footnote{
用分类公理、并集公理、配对公理和幂集构建集合$A$与$B$的笛卡尔集的过程,用自然语言描述将十分繁琐。以下是采用\emph{合式公式(well-formed formula)}表达的结果,仅供熟悉此知识的读者参考。
\[A\times B=\left\{X|X\in\mathcal{P}\left(\mathcal{P}\left(A\cup B\right)\right)\vee\varphi\left(X\right)\right\}
\]
其中,$\varphi\left(X\right)$表示关于$X$的语句,
\[
\varphi\left(X\right)=\exists U\exists V\exists W \exists Y\left(U\in A \wedge V\in B\wedge \phi\left(W,U,U\right)\wedge\phi\left(Y,U,V\right)\wedge\phi\left(X,W,Y\right)\right)
\]
而关于$X$、$U$和$V$的语句
\[
\phi\left(X,U,V\right)=\forall Z\left(Z\in X\leftrightarrow\left(\left(X=U\right)\vee\left(X=V\right)\right)\right)
\]
其中,记号$\forall a\left(\text{关于$a$的语句}\right)$表示“对每一/任一满足关于$a$的语句规定的$a$”。注意到,语句$\phi\left(X,U,V\right)$表示的就是$X=\left\{U,V\right\}$这件事,故语句$\varphi\left(X\right)$就是让$X=\left\{\left\{U\right\},\left\{U,V\right\}\right\}$。
}。故我们可定义由集合$A$和$B$形成的所有满足$a\in A$且$b\in B$的有序对的集合为$A$与$B$的\emph{笛卡尔积(Cartesian product)},记为$A\times B$。

以下是与笛卡尔集有关的一些性质:
\begin{enumerate}
\item $\left(A\cup B\right)\times X=\left(A\times X\right)\cup\left(B\times X\right)$
\item $\left(A\cap B\right)\times \left(X\cap Y\right)=\left(A\times X\right)\cap\left(B\times Y\right)$
\item $\left(A\setminus B\right)\times X=\left(A\times X\right)\setminus \left(B\times X\right)$
\item $A=\emptyset\text{或}B=\emptyset\Leftrightarrow A\times B=\emptyset$
\item $A\subset X\text{且}B\subset Y\Rightarrow A\times B\subset X\times Y$\footnote{符号$\Rightarrow$的意义:(语句1)$\Rightarrow$(语句2)表示“若(语句1 ),则(语句2)。”}
\item $A\times B\subset X\times Y\text{且}A\times B\neq\emptyset\Rightarrow A\subset X\text{且}B\subset Y$
\end{enumerate}

% ==========================================================================

\end{document}
