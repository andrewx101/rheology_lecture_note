\documentclass[main.tex]{subfiles}
\begin{document}
笔者在多年的学习和交流过程中发现,流变学有一些基本问题是许多教科书没有强调的。对这些问题的生疏,是部分流变学同行产生各种误解的根源。以下是笔者觉得需要先在前言中抛出的两种说法。

一、在流变仪上测得的“应力”、“应变”,跟流变学本构关系里的“应力”、“应变”是两码事。

本构关系里的“应力”、“应变”都是张量场函数。“场函数”是指自变量是3维空间位置和时间的函数;“张量”是指采用具有几何意义的矢量和线性变换来表示的物理量,这些物理量要遵守不同坐标系选取之间的坐标变换法则。因此,具有本构关系意义的“应力--应变关系”是函数与函数之间的关系(泛函关系)。

流变学本构关系呈现出这样的数学形式,并非小部分人的特殊趣味,而是物理客观性的自然结果,即“材料的流变学性质不应该依赖参考系和坐标系的选取方式,亦不应该依赖被研究体系的规模和形变方式”。这一客观性要求并非一句哲学口号;它对定量表征应力、应变的物理量的构建方式,乃至本构关系的形式,都提出了非常具体的要求或限制,因此是流变学理论所不可缺少的内容。

而平时在流变仪上测得的“应力”、“应变”则是限于某种理想流场(例如简单剪切、扭转、单轴拉伸、3点弯曲等)假定之下才能比较完整地表征材料的应力和应变状态的标量值变量,它们往往不依赖空间位置。用这样的“应力”和“应变”表示的关系是变量与变量之间的关系(函数关系);它们虽然简单,但仅限于在所假定的流场下才适用。这样的“应力--应变关系”不应被称为“本构关系”。

二、流变学研究分为软物质物理研究和非牛顿流体力学研究两类。

流变学的发展和高分子科学的发展不仅在时间上几乎重合,在研究内容上也高度重合。在二十世纪,高分子科学是一个物理理论和工业技术都同时大发展的学科,流变学也一样。近几十年,胶体体系的研究也与流变学研究结合得非常紧密,一齐构成了所谓的软凝聚态物理学科。流变学研究的“上下文”(context),可以大致分为各自比较独立两类,一类是软凝聚态物理研究,一类是工业实际问题研究。软凝聚态物理与流变学的结合,主要是从非平衡态统计力学的角度切入;而工业实际问题对流变学理论的需求,主要从非牛顿流体力学的角度切入。流变学求问于非平衡态统计力学,主要是为材料的本构关系探求微观机理。但流变学要能解决工业实际,并不能停留在本构关系本身,还需要求解不同实际情况的流动问题。这两种研究需的实验条件和实验内容是很不同的。为前一种研究建立的实验条件,未必能做后一种研究。

想要理解上述两种说法并进行批判,形成自己的观点,就需要先严格掌握流变学的理论基础。但是这超出了大部分化学类或高分子专业本科的数学基础。这是从流变学诞生开始就面临的问题\footnote{
      I said, ``This branch of physics already exists: it is called mechanics of continuous media, or mechanics of continua.''``No, this will not do,'' Bingham replied, ``Such a designation will frighten away the chemists.'' So he consulted the professor of classical languages and arrived at the designation of rheology, taking as the motto of the subject Heraclitus' $\pi\alpha\nu\tau\alpha \rho\varepsilon\iota$ or ``everything flows.''\cite{Reiner1964}
}。因此,已有大量作者致力于撰写“化学家看得懂的流变学教材”。但是,若想要成为能够利用流变学知识解决实际问题的专业人士,是无法回避上列两个问题的,而此时我们仍然面临着数学和物理基础缺口的问题。

流变学理论涉及到的数学基础,常常在面向的物理系学生的教材中介绍,是适应物理学人才的培养体系写就的。为了物理系人才往量子物理、凝聚态物理、天体物理……等不同方向发展的可能性,面向物理系学生的数学教材往往形成了远超流变学必需的广度和深度。因此当化学背景人士仅为了流变学的兴趣,去找数学书补习时,往往会发现他要么选择放弃,要么就只能先变成一个数学基础扎实的物理系毕业生。事实上,化学学科有其特有的旨趣和不可替代的角色,成为一名化学家——无论是一名有机合成专家还是一名高分子材料专家——所需要经受的学术和工程训练已经十分丰富和系统。流变学应该是化学家不得不关心的,需要用到最多数学和物理的学科了。仅为流变学裁剪出来的最少必要数学和物理学讲义,也许对任何一个物理系学生的发展是没有帮助的,却是面临流变学问题的化学家所需要的。这样的讲义,当然需要针对化学类专业本科的数学和物理学教育背景去帮读者补习数学和物理学,或至少指出适应化学类专业人士的自学路径。

笔者从攻读研究生时期至今,自己闭门造车地完成了这件事情,没有机会受到批评。写就这本讲义,第一目的就是为了广泛接受批评,并不断修改它。如果它现在还不能对他人有正面的益处,也希望它有朝一日能有。请把批评意见发至邮箱mswxsun@scut.edu.cn。

基于上述的思想,本讲义的内容有很多个人做法,在此作部分解释。

讲义主要内容分两部分。第一部分是必要的数学基础。第二部分是流变学的物理理论基础。关于数学基础部分的内容深度和广度,我作了如下考虑:
\begin{enumerate}
      \item 从集合论开始介绍。在物理学中,数学不仅用于表示数量关系,还用于传达\emph{物理概念}\footnote{A perhaps unorthodox feature of this book is its use of mathematics as a conceptual tool rather than merely as a device to express complex numerical relations.---Coleman, Markovitz and Noll (1966)\cite{Coleman1966}.},因此我们才需要采用近世的、基于集合论的语言来铺叙流变学原理。有很多常见的困惑,使用简单的数学语言确实是无法回答的;数学语言太简单本身恰恰就是导致这些困惑的原因。不过,数学语言的抽象化和一般化是无止境的;我们不盲目追求这个。那么,一个起码的数学语言深度,应当要恰好能满足物理思想的准确表达。在这一宗旨下,集合和映射的语言是必须的。近几年工科大学的高等数学教学以为集合论知识已下沉到中学,就普遍不再作正面介绍\footnote{例如同济大学《高等数学(第七版)》详见其前言。此外,一般工科数学还会修《概率论与数理统计》课程,可能会在学习“概率的抽象定义”时再涉及到一些集合论基础。但是化学专业不修该课的也很普遍。}。笔者不完全调查发现也不是所有中学都教了集合论;就算教了,内容也非常浅。因此本讲义从集合论开始介绍。
      \item 以化学类工科专业大一学习的微积分和线性代数为基础。具体地,我假定读者已掌握华南理工大学数学系编著的《高等数学(上、下册)》和《线性代数与解析几何》,原则上只介绍上述课本中没有涉及到的数学知识。我在讲义中尽可能多地提示了所介绍的新内容跟上述课本中已经介绍过哪些内容相关(精确到章节和小标题)。这么做的目的是为了让化学类专业的同学具体地衔接数学上的缺口,顺利实现数学语言的近世化。
      \item 没有完全建立在张量分析的基础上。张量分析是为了处理在不仅限于3维、不仅限于欧几里得空间、更不仅限于直角坐标系下的物理问题的。这在基于非相对论经典力学的流变学当中是不必要的。因此本讲义打算仅介绍一个不涉及度规张量的3维欧几里得空间上的曲线坐标系理论。事实上,在这样的语境下,流变学中使用的张量,在数学上都可仅称作线性算符。按照流变学学科的惯例,它们都仍被称作张量,这表示当这些线性算符被用于表示3维欧几里得空间上的物理概念时,必须不依赖坐标系的选择(即遵循相应的曲线坐标变换法则)。
      \item 直接有助于理解物理概念的数学基础写在正文,仅为了证明数学定理所需的数学基础写在附录。定理的证明过程用浅灰字体。没有习题。这些做法都是因为数学内容在本讲义所扮演的角色不是正式的数学教学,而是展示流变学理论的概念体系。讲义引用到的教材,都是充当正式学习这些数学知识的优秀教材。
\end{enumerate}

概念在被定义时会通过字体的改变来突出。例如,\emph{集合(set)}是具有某种特性的事物的整体。有时,定义是以带编号的方式独立列出的,有时则在正文叙述当中引入。定义是极其重要的。它在文中只出现一次,因此难免要经常反复回顾。有时我不采用带编号式的定义,只是因为需要定义的概念很多。如果每个定义都带编号,定义将会严重打断行文的流畅性,让本来就抽象的内容更难以阅读。但是这么做的代价则是使定义失去了引用链接的便利,故在此敬请读者在阅读时贴上标签,不厌其烦地翻阅和回顾定义。将来笔者将会编制名词索引,彻底解决概念定义的定位问题,并进一步减少定义以编号的方式出现现象。

如同许多不尽人意的教科书那般,我在前言中宣称要做的事,可能恰恰做得特别差,的确是水平和时间有限所致,请见谅!请同学们学习时不要只信一本书,要带着质疑同时看几本书,并让它们相互“对质”,以防被不学无术的作者坑害!下列书籍是在同类主题的书中我特别推荐的,包括了数学、力学和流变学。如能直接看懂这些书,就完全不必使用本讲义(序号同时代表着建议的学习顺序)。
\begin{enumerate}
      \item P. Halmos (1960), \textit{Naive Set Theory}, D. Van Nostrand Company, Inc.
      \item K. Hoffman \& R. Kunze (1971), \textit{Linear Algebra, 2nd ed.}, Prentice-Hall Inc.
      \item R. Williamson, R. Crowell \& H. Trotter (1972), \textit{Calculus of Vector Functions, 3rd ed.}, Prentice-Hall Inc.
      \item D. Smith (1993), \textit{An Introduction to Continuum Mechanics---after Truesdell and Noll}, Springer
      \item A. Murdoch (2012), \textit{Physical Foundations of Continuum Mechanics}, Cambridge University Press
      \item W. Schowalter (1978), \textit{Mechanics of Non-Newtonian Fluids}, Pergamon Press
      \item R. Tanner (2000), \textit{Engineering Rheology, 2nd ed.}, Oxford University Press
      \item 许元泽(1988),\emph{高分子结构流变学},四川教育出版社
      \item 王玉忠、郑长义(1993),\emph{高聚物流变学导论},四川大学出版社
\end{enumerate}

虽然这个讲义还没写完,这只是临时前言,但也要列出早应表达的感恩。感谢我的博士生导师童真教授在我工作后的继续庇护,为我营造了(一定程度上)不用理会“科研产出”的宽松环境!这样的环境离不开钱,感谢国家的资助!这样的环境也离不开我所在单位的宽容,感谢华南理工大学人事处和材料科学与工程学院的领导和同事!

更多说明和致谢将在讲义完成之后补充。

% 许元泽;李俊杰,帮助和鼓励我学习数学;华南理工大学材料科学与工程学院历任领导,给我一个宽松的环境。知乎上的素未谋面的人。

\begin{flushright}
      孙尉翔\\
      \today
\end{flushright}
\end{document}

