\documentclass[main.tex]{subfiles}
\begin{document}
本讲义不是一个完整、深入的流变学教材,仅为了帮助化工类本科专业的人士建立必要的数学基础,为理解其他正规的流变学教材和论文作准备。

讲义主要内容分两部分。

第一部分是必要的数学基础。第二部分是流变学的物理理论基础。

关于数学基础部分的内容深度和广度,我作了如下考虑:
\begin{enumerate}
    \item 以化学类工科专业大一学习的微积分和线性代数为基础。具体地,我假定读者已完整地、熟练地掌握华南理工大学xxx,故只介绍上述课本中没有涉及到的数学知识。我在讲义中尽可能多地提示了所介绍的新内容跟上述课本中已经介绍过哪些内容相关(精确到章、节、页码)。希望读者在比较后能够同意,我在此讲义中新介绍的数学知不是完全陌生的。
    \item 第一优先介绍第二部分的物理理论基础需要用到的数学概念。例如,在引入时空的概念(第X章)时,需要用到集合的划分与等价类的知识,因此我在集合论的章节中介绍了关系的知识。又例如,流变学中的各种张量,在本讲义已简化成有限维向量空间上的线性变换。按照上一条假定考虑读者的数学基础,仍需要比较详尽地介绍有限维向量空间及其上的线性变换的知识。
    \item 采用不依赖坐标的表达风格(不介绍也不使用求和约定)。我希望这种做法本身能彰显流变学理论的建立者希望实现的物理客观性,既材料的响应规律(数学表达式)不依赖包括坐标系的选择在内的任何主观选择。但是面临具体应用问题时,常需要建立曲线坐标系。此时向量和线性变换的坐标关系需要曲线坐标系的知识。
    \item 仅为了证明某定理所需的数学概念和引理,要么不正式介绍、要么放在附录中介绍。
    \item 定理的证明过程,仅供有兴趣的读者参考,故未必都提供。不提供时,我尽可能提出其他提供了证明过程的参考资料。例如:雷诺传输定理的证明(附录XX)、等距变换的表示定理的证明(附录XX)。
    \item 本讲义用不到的数学知识,不作介绍。例如,在集合论的章节中提到了部分公理集合论的公理,但又假定关于自然数、算术运算、偏序、全序、数学归纳法等知识为读者已知,包括皮亚诺公设在内的相关的公设就不作介绍了。故集合论的章节的最终状态,深度上似乎要介绍公理集合论,但广度上又不完整。这是我故意为之的。因为本讲义的目标不是要向化学专业的读者大肆介绍数学知识,而是补充看懂流变学理论的最少必要的、作了严格定义的数学知识。
\end{enumerate}

一般情况下,概念的定义仅通过字体的改变来暗示。例如,\emph{集合(set)}是具有某种特性的事物的整体。仅在需要时,定义才以带编号的方式引入。而定理、引理和例子则均带编号。定义是极其重要的。它在文中只出现一次,因此难免要经常反复回顾。不采用带编号式的引入,只是因为需要定义的概念很多,如果每个定义都带编号定义将会严重打断行文的流畅性,让本来就抽象的内容更难以阅读,而代价则是使定义失去了引用链接的便利,故在此敬请读者在学习时要不厌其烦地翻阅回顾定义。

关于物理理论基础部分的内容深度和广度,我作了如下考虑:(待补充。)

更多说明将在讲义完成之后再在此解释。
\begin{flushright}
孙尉翔\\
2020年10月
\end{flushright}
\end{document}

