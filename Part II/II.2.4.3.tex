\documentclass[main.tex]{subfiles}
% 内积空间上的线性算符
\begin{document}
\begin{definition}[幺正算符]\label{def:II.6.2}
    设$\mathcal{V}$是数域$\mathbb{F}$上的内积空间,若线性算符$\mathbf{Q}\in\mathcal{L}\left(\mathcal{V}\right)$保持内积,即$\left(\mathbf{Qa}|\mathbf{Qb}\right)=\left(\mathbf{a}|\mathbf{b}\right),\forall\mathbf{a},\mathbf{b}\in\mathcal{V}$,则称$\mathbf{Q}$是一个幺正算符(unitary operator)。
\end{definition}

以下定理给出幺正算符的双射性。

\begin{theorem}\label{thm:II.6.3}
    设$\mathcal{V}$是数域$\mathbb{F}$上的有限维内积空间,$\mathbf{T}\in\mathcal{L}\left(\mathcal{V}\right)$是一个线性算符,则以下命题相互等价:
    \begin{enumerate}
        \item $\mathbf{T}$保持内积
        \item $\mathbf{T}$是双射
        \item $\mathbf{T}$把$\mathcal{V}$的某个规范正交基映射为另一个规范正交基
        \item $\mathbf{T}$把$\mathcal{V}$的每个规范正交基映射为一个规范正交基
    \end{enumerate}
\end{theorem}
\begin{proof}
    1)$\Rightarrow$2):由1),$\left(\mathbf{Ta}|\mathbf{Ta}\right)=\left(\mathbf{a}|\mathbf{a}\right)\geq0\forall\mathbf{a}\in\mathcal{V}$,当且仅当$\mathbf{a}=\mathbf{0}$时取等号。故$\mathbf{T}$是非奇异的。由于$\mathbf{T}:\mathcal{V}\rightarrow\mathcal{V}$,故$\mathbf{T}$是双射\footnote{这里用到了线性变换的维数定理及其推论。}。

    2)$\Rightarrow$3):由于$\mathbf{T}$是内积空间上的同构映射,令$\left\{\mathbf{\hat{e}}_i\right\}$是$\mathcal{V}$的一组规范正交基,则由向量空间上的同构映射性质,$\left\{\mathbf{T\hat{e}}_i\right\}$是$\mathcal{V}$的一组基,且有$\left(\mathbf{T\hat{e}}_i|\mathbf{T\hat{e}}_j\right)=\left(\mathbf{\hat{e}}_i|\mathbf{\hat{e}}_j\right)=\delta_{ij},i,j=1,\cdots,\mathrm{dim}\mathcal{V}$,故$\left\{\mathbf{T\hat{e}}_i\right\}$是$\mathcal{V}$的一组规范正交基。

    3)$\Rightarrow$4):显然易证;

    4)$\Rightarrow$1):由4),若已知$\left\{\mathbf{\hat{e}}_i\right\}$是$\mathcal{V}$的一组规范正交基,且$\left\{\mathbf{T\hat{e}}_i\right\}$也是规范正交基,则有$\left(\mathbf{T\hat{e}}_i|\mathbf{T\hat{e}}_j\right)=\delta_{ij}=\left(\mathbf{\hat{e}}_i|\mathbf{\hat{e}}_j\right)$。对任意$\mathbf{a},\mathbf{b}\in\mathcal{V}$,又有$\mathbf{a}=\sum_{i=1}^{n}\alpha_i\mathbf{\hat{e}}_i,\mathbf{b}=\sum_{i=1}^{n}\beta_i\mathbf{\hat{e}}_i,n\equiv\mathrm{dim}\mathcal{V}$,则$\left(\mathbf{a}|\mathbf{b}\right)=\sum_{i=1}^n\alpha_i\overline{\beta_j},\left(\mathbf{Ta}|\mathbf{Tb}\right)=\left(\sum_{i=1}^n\mathbf{T\hat{e}}_i|\sum_{j=1}^n\mathbf{T\hat{e}}_j\right)=\sum_{i=1}^n\alpha_i\overline{\beta_j}=\left(\mathbf{a}|\mathbf{b}\right)$,故$\mathbf{T}$保持内积。
\end{proof}

上述定理告诉我们,幺正算符必可逆,且幺正算符的逆算符也是幺正算符。如果在内积空间$\mathcal{V}$上还定义了欧几里得范$\left\|\mathbf{a}\right\|^2\equiv\left(\mathbf{a}|\mathbf{a}\right),\forall\mathbf{a}\in\mathcal{V}$,那么当$\mathbf{T}\in\mathcal{L}\left(\mathcal{V}\right)$是一个幺正算符时,就有$\left\|\mathbf{Ta}\right\|=\left\|\mathbf{a}\right\|\forall\mathbf{a}\in\mathcal{V}$,即幺正算符不改变向量的长度。此外还易证,两个幺正算符的复合也是幺正算符。特别地,恒等算符$\mathbf{I}$是幺正算符。

下一条定理告诉我们幺正算符的伴随算符有何性质。

\begin{theorem}\label{thm:II.6.4}
    设$\mathcal{V}$是数域$\mathbb{F}$上的内积空间,$\mathbf{U}\in\mathcal{L}\left(\mathcal{V}\right)$是线性算符,则$\mathbf{U}^*\mathbf{U}=\mathbf{I}\Leftrightarrow\mathbf{U}$是幺正算符。
\end{theorem}
\begin{proof}
    设$\mathbf{U}$是幺正算符,则$\mathbf{U}$可逆,且$\left(\mathbf{Ua}|\mathbf{b}\right)=\left(\mathbf{Ua}|\mathbf{UU}^{-1}\mathbf{b}\right)=\left(\mathbf{a}|\mathbf{U}^{-1}\mathbf{b}\right),\forall\mathbf{a},\mathbf{b}\in\mathcal{V}$。因此$\mathbf{U}^{-1}=\mathbf{U}^*$即$\mathbf{UU}^*=\mathbf{I}=\mathbf{U}^*\mathbf{U}$。

    设$\mathbf{U}^*\mathbf{U}=\mathbf{UU}^*=\mathbf{I}$,则$\mathbf{U}^{-1}=\mathbf{U}^*,\left(\mathbf{Ua}|\mathbf{Ub}\right)=\left(\mathbf{a}|\mathbf{U}^*\mathbf{Ub}\right)=\left(\mathbf{a}|\mathbf{Ib}\right)=\left(\mathbf{a}|\mathbf{b}\right),\forall\mathbf{a},\mathbf{b}\in\mathcal{V}$。
\end{proof}

在矩阵代数中,有“正交矩阵”的概念。如果数域$\mathbb{F}$上的$n\times n$矩阵$A\in\mathbb{F}^{n\times n}$满足$A^\intercal A=I$,其中$I$是$n\times n$单位矩阵,则称矩阵$A$是正交矩阵(orthogonal matrix)。由上面的定理可知,只有在实数域$\mathbb{R}$上的内积空间上,幺正算符在给定基下的坐标矩阵是才是一个正交矩阵。因此,我们又把实数域上的幺正算符称为正交算符(orthogonal operator)。
\end{document}