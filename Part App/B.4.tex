\documentclass[main.tex]{subfiles}
% VI.2.4 反函数定理和隐函数定理
\begin{document}
\begin{lemma}\label{thm:inv_func_l1}
    线性变换$\mathbf{L}:\mathcal{V}_n\rightarrow\mathcal{W}_m$是单射当且仅当存在正实数$p>0$满足$\left\|\mathbf{Lx}\right\|\geq p \left\|\mathbf{x}\right\|\forall\mathbf{x}\in\mathcal{V}_n$。
\end{lemma}
\begin{proof}
    如果$\mathbf{L}$不是单射,则存在$\mathbf{x}_0\neq\mathbf{0}$满足$\mathbf{Lx}_0=\mathbf{0}$,即有$\left\|\mathbf{Lx}_0\right\|=0<m\left\|\mathbf{x}\right\|$。故其逆否命题成立。

    如果$\mathbf{L}$是单射,则其存在逆$\mathbf{L}^{-1}$满足$\mathbf{L}^{-1}\mathbf{L}=\mathbf{I}_\mathcal{V}$且$\mathbf{L}^{-1}$也是线性变换。由定理\ref{thm:II.4.3},$\exists k>0, \left\|\mathbf{L}^{-1}\mathbf{y}\right\|\leq k\left\|\mathbf{y}\right\|\forall\mathbf{y}\in\mathcal{W}_m$。令$p=1/k$则有$p\left\|\mathbf{x}\right\|=m\left\|\mathbf{L}^{-1}\mathbf{Lx}\right\|\leq pk\left\|\mathbf{Lx}\right\|=\left\|\mathbf{Lx}\right\|$
\end{proof}

\begin{lemma}\label{thm:inv_func_l2}
    设$\mathbf{f}:\mathbb{R}^n\rightarrow\mathbb{R}^m$是可微函数,且在$\mathbf{x}_0$处连续可微。再假设$\mathbf{f}$在$\mathbf{x}_0$处的导数$\left.\frac{d\mathbf{f}\left(\mathbf{x}\right)}{d\mathbf{x}}\right|_{\mathbf{x}=\mathbf{x}_0}$是单射。则存在$\delta>0$和$M>0$使得只要$\left\|\mathbf{x}-\mathbf{x}_0\right\|<\delta$就有
    \[
        \left\|\left(\mathrm{d}_{\mathbf{x}=\mathbf{x}_0}\mathbf{f}\left(\mathbf{x}\right)\right)\mathbf{y}\right\|\geq M\left\|\mathbf{y}\right\|,\quad\forall\mathbf{y}\in\mathbb{R}^n
    \]
\end{lemma}
\begin{proof}
    记函数$\mathbf{f}$的导函数为$\mathbf{L}$。由引理\ref{thm:inv_func_l1},存在$p>0$使得$\left\|\mathbf{L}\left(\mathbf{x}_0\right)\mathbf{y}\right\|\geq p\left\|\mathbf{y}\right\|,\forall\mathbf{y}\in\mathbb{R}^n$。同时,由于$\mathbf{f}$在$\mathbf{x}_0$处连续可微,故对任一正实数——此处选择$p/2>0$——存在$\delta>0$使得只要$\left\|\mathbf{x}-\mathbf{x}_0\right\|<\delta$就有$\left\|\mathbf{L}\left(\mathbf{x}\right)-\mathbf{L}\left(\mathbf{x}_0\right)\right\|\leq p/2$。由线性变换的范的定义,有不等式
    \[
        \left\|\left(\mathbf{L}\left(\mathbf{x}\right)-\mathbf{L}\left(\mathbf{x}_0\right)\right)\mathbf{y}\right\|\leq\left\|\mathbf{L}\left(\mathbf{x}\right)-\mathbf{L}\left(\mathbf{x}_0\right)\right\|\left\|\mathbf{y}\right\|\leq\frac{p}{2}\left\|\mathbf{y}\right\|\]
    由三角不等式又有
    \begin{align*}
        \left\|\mathbf{L}\left(\mathbf{x}_0\right)\mathbf{y}\right\|                                                                                                                         & =\left\|\mathbf{L}\left(\mathbf{x}_0\right)\mathbf{y}+\mathbf{L}\left(\mathbf{x}\right)\mathbf{y}-\mathbf{L}\left(\mathbf{x}\right)\mathbf{y}\right\|                   \\
                                                                                                                                                                                             & \leq\left\|\mathbf{L}\left(\mathbf{x}_0\right)\mathbf{y}-\mathbf{L}\left(\mathbf{x}\right)\mathbf{y}\right\|+\left\|\mathbf{L}\left(\mathbf{x}\right)\mathbf{y}\right\| \\
        \Leftrightarrow\left\|\mathbf{L}\left(\mathbf{x}_0\right)\mathbf{y}\right\|-\left\|\mathbf{L}\left(\mathbf{x}\right)\mathbf{y}-\mathbf{L}\left(\mathbf{x}_0\right)\mathbf{y}\right\| & \leq\left\|\mathbf{L}\left(\mathbf{x}\right)\mathbf{y}\right\|
    \end{align*}
    上式不等号左边可代入刚刚确定的结论:$\left\|\mathbf{L}\left(\mathbf{x}_0\right)\mathbf{y}\right\|\leq p\left\|\mathbf{y}\right\|, -\left\|\mathbf{L}\left(\mathbf{x}\right)\mathbf{y}-\mathbf{L}\left(\mathbf{x}_0\right)\mathbf{y}\right\|\geq-\frac{p}{2}$,得
    \begin{align*}
        \left\|\mathbf{L}\left(\mathbf{x}\right)\mathbf{y}\right\| & \geq\left\|\mathbf{L}\left(\mathbf{x}_0\right)\mathbf{y}\right\|-\left\|\mathbf{L}\left(\mathbf{x}\right)\mathbf{y}-\mathbf{L}\left(\mathbf{x}_0\right)\mathbf{y}\right\| \\
                                                                   & \geq p\left\|\mathbf{y}\right\|-\frac{p}{2}\left\|\mathbf{y}\right\|=\frac{p}{2}\left\|\mathbf{y}\right\|
    \end{align*}

    故存在$M=p/2>0$满足命题。
\end{proof}

\begin{lemma}\label{thm:inv_func_l3}
    设$\mathbf{f}:\mathbb{R}^n\rightarrow\mathbb{R}^m$是可微函数,且在$\mathbf{x}_0$处连续可微。再假设$\mathbf{f}$在$\mathbf{x}_0$处的导数$\mathrm{d}_{\mathbf{x}=\mathbf{x}_0}\mathbf{f}\left(\mathbf{x}\right)$是单射,则存在正实数$\delta>0$和$M>0$使得只要$\left\|\mathbf{x}^\prime-\mathbf{x}_0\right\|<\delta$就有$\left\|\mathbf{f}\left(\mathbf{x}^\prime\right)-\mathbf{f}\left(\mathbf{x}\right)\right\|\leq M\left\|\mathbf{x}^\prime-\mathbf{x}\right\|$。
\end{lemma}
\begin{proof}
    记函数$\mathbf{f}$的导函数为$\mathbf{L}$。由引理\ref{thm:inv_func_l1},存在$m>0$使得$\left\|\mathbf{L}\left(\mathbf{x}_0\right)\mathbf{y}\right\|\geq m\left\|\mathbf{y}\right\|\forall\mathbf{y}\in\mathbb{R}^n$。

    由于$\mathbf{f}$在$\mathbf{x}_0$处连续可微,故对任一正实数——此处选择$M=m/\left(2\sqrt{n}\right)>0$——存在$\delta>0$使得只要$\left\|\mathbf{x}-\mathbf{x}_0\right\|<\delta$就有$\left\|\mathbf{L}\left(\mathbf{x}\right)-\mathbf{L}\left(\mathbf{x}_0\right)\right\|\leq m/\left(2\sqrt{n}\right)$。

    按照命题叙述,设$\mathbf{x}$和$\mathbf{x}^\prime$是$\mathbb{R}^n$的任意两向量满足$\left\|\mathbf{x}^\prime-\mathbf{x}_0\right\|<\delta,\left\|\mathbf{x}-\mathbf{x}_0\right\|<\delta$,令$\mathbf{z}=\mathbf{x}^\prime-\mathbf{x}$,则对$0\leq t\leq 1$有
    \begin{align*}
        \left\|\mathbf{x}+t\mathbf{z}-\mathbf{x}_0\right\| & =\left\|t\mathbf{x}^\prime+\left(1-t\right)\mathbf{x}-\mathbf{x}_0\right\|                                                                       \\
                                                           & =\left\|t\left(\mathbf{x}^\prime-\mathbf{x}\right)+\left(1-t\right)\left(\mathbf{x}-\mathbf{x}_0\right)\right\|                                  \\
                                                           & \leq t\left\|\mathbf{x}^\prime-\mathbf{x}_0\right\|+\left(1-t\right)\left\|\mathbf{x}-\mathbf{x}_0\right\|<t\delta+\left(1-t\right)\delta=\delta
    \end{align*}
    上述推导结论在几何上的意义是,只要点$\mathbf{x}^\prime,\mathbf{x}$在由$\left\|\mathbf{x}_0\right\|<\delta$的开集内部,则它们的连线上的点$\mathbf{x}+t\mathbf{z}$都在此开集内部,或称“$\delta$-球是凸的”。由于导函数连续是在整个$\delta$-球内都成立的,因此对由$0\leq t\leq 1$定义的所有点$\mathbf{x}+t\mathbf{z}$均有$\left\|\mathbf{L}\left(\mathbf{x}+t\mathbf{z}\right)-\mathbf{L}\left(\mathbf{x}_0\right)\right\|<m/\left(2\sqrt{n}\right)$。又由线性变换的模的定义有$\left\|\left(\mathbf{L}\left(\mathbf{x}+t\mathbf{z}\right)-\mathbf{L}\left(\mathbf{x}_0\right)\right)\mathbf{y}\right\|\leq\left\|\mathbf{L}\left(\mathbf{x}+t\mathbf{z}\right)-\mathbf{L}\left(\mathbf{x}_0\right)\right|\left\|\mathbf{y}\right\|<M\left\|\mathbf{y}\right\|\forall\mathbf{y}\in\mathbb{R}^n$。

    引入“取坐标函数”,$\pi_k\left(\mathbf{x}\right)=x_k,\mathbf{x}=\left(x_1,\cdots,x_n\right)\in\mathbf{R}^n,k=1,\cdots,n$。易验证$\frac{d\pi_k\left(\mathbf{x}\right)}{d\mathbf{x}}\equiv\pi_k\left(\mathbf{x}\right)$。若定义$g_k\left(t\right)=\pi_k\left(\mathbf{f}\left(\mathbf{x}+t\mathbf{z}\right)\right),0\leq t\leq 1$,则由链式法则可得如下关系
    \[
        \frac{dg_k}{dt}=\pi_k\left(\mathbf{L}\left(\mathbf{x}+t\mathbf{z}\right)\mathbf{z}\right)\]
    由微分中值定理,存在$t_k\in\left[0,1\right]$使得$g_k\left(1\right)-g_k\left(0\right)=\frac{dg_k}{dt_k}$。代入$g_k$、$\frac{dg_k}{dt_k}$的表达式得$\pi_k\left(\mathbf{f}\left(\mathbf{x}^\prime\right)\right)-\pi_k\left(\mathbf{f}\left(\mathbf{x}\right)\right)=\pi_k\left(\mathbf{L}\left(\mathbf{x}+t_k\mathbf{z}\right)\mathbf{z}\right)$。
    注意到,函数$\pi_k\left(\mathbf{x}\right)$就是向量$\mathbf{x}$在第$k$个基上的投影长度。由投影长度不大于向量长度(代数意义是使用柯西--施瓦茨不等式),有
    \begin{align*}
        \left\|\mathbf{f}\left(\mathbf{x}^\prime\right)-\mathbf{f}\left(\mathbf{x}\right)\right\| & \geq\left|\pi_k\left(\mathbf{f}\left(\mathbf{x}^\prime\right)-\mathbf{f}\left(\mathbf{x}\right)\right)\right|                \\
                                                                                                  & =\left|\pi_k\left(\mathbf{f}\left(\mathbf{x}^\prime\right)\right)-\pi_k\left(\mathbf{f}\left(\mathbf{x}\right)\right)\right| \\
                                                                                                  & =\left|\pi_k\left(\mathbf{L}\left(\mathbf{x}+t_k\mathbf{z}\right)\mathbf{z}\right)\right|                                    \\
    \end{align*}
    另有以下三角不等式成立:
    \[
        \left\|\left(\mathbf{L}\left(\mathbf{x}+t_k\mathbf{z}\right)-\mathbf{L}\left(\mathbf{x}_0\right)\right)\mathbf{z}\right\|+\left\|\mathbf{L}\left(\mathbf{x}_0\right)\mathbf{z}\right\|\leq\left\|\mathbf{L}\left(\mathbf{x}+t_k\mathbf{z}\right)\mathbf{z}\right\|
    \]
    上式左右取投影也成立,即
    \[\left|\pi_k\left(\left(\mathbf{L}\left(\mathbf{x}+t_k\mathbf{z}\right)-\mathbf{L}\left(\mathbf{x}_0\right)\right)\mathbf{z}\right)\right|+\left|\pi_k\left(\mathbf{L}\left(\mathbf{x}_0\right)\mathbf{z}\right)\right|\leq\left|\pi_k\left(\mathbf{L}\left(\mathbf{x}+t_k\mathbf{z}\right)\mathbf{z}\right)\right|
    \]
    以上不等式联合有
    \[
        \left\|\mathbf{f}\left(\mathbf{x}^\prime\right)-\mathbf{f}\left(\mathbf{x}\right)\right\|\geq\left|\pi_k\left(\left(\mathbf{L}\left(\mathbf{x}+t_k\mathbf{z}\right)-\mathbf{L}\left(\mathbf{x}_0\right)\right)\mathbf{z}\right)\right|+\left|\pi_k\left(\mathbf{L}\left(\mathbf{x}_0\right)\mathbf{z}\right)\right|
    \]
    由事实$\left\|\mathbf{x}\right\|\leq\sqrt{n}\mathrm{max}\left\{\left|x_i\right|\right\}\equiv\sqrt{n}\mathrm{max}\left\{\pi_i\left(\mathbf{x}\right)\right\}$(之前在说明范的定义的等价性时证明过该事实)知,在$k=1,\cdots,m$中至少有一个$k$满足
    \[
        \sqrt{n}\left|\pi_k\left(\mathbf{L}\left(\mathbf{x}_0\right)\mathbf{z}\right)\right|\geq\left\|\mathbf{L}\left(\mathbf{x}_0\right)\mathbf{z}\right\|
    \]
    再次利用投影不大于原长,有
    \[
        \left|\pi_k\left(\left(\mathbf{L}\left(\mathbf{x}+t_k\mathbf{z}\right)-\mathbf{L}\left(\mathbf{x}_0\right)\right)\mathbf{z}\right)\right|\leq\left\|\left(\mathbf{L}\left(\mathbf{x}+t_k\mathbf{z}\right)-\mathbf{L}\left(\mathbf{x}_0\right)\right)\mathbf{z}\right\|\]
    再次联合这些不等式有
    \begin{align*}
        \left\|\mathbf{f}\left(\mathbf{x}^\prime\right)-\mathbf{f}\left(\mathbf{x}\right)\right\| & \geq\frac{1}{\sqrt{n}}\left\|\mathbf{L}\left(\mathbf{x}_0\right)\mathbf{z}\right\|-\left\|\left(\mathbf{L}\left(\mathbf{x}+t_k\mathbf{z}\right)-\mathbf{L}\left(\mathbf{x}_0\right)\right)\mathbf{z}\right\| \\
                                                                                                  & \geq2M\left\|\mathbf{z}\right\|-M\left\|\mathbf{z}\right\|=M\left\|\mathbf{x}^\prime-\mathbf{x}\right\|
    \end{align*}
\end{proof}

有了上面三个引理,我们可正式给出反函数定理的证明。
% =============== the theorem =========================
\begin{theorem}[反函数定理]
    设$\mathbf{f}:\mathbb{R}^n\rightarrow\mathbb{R}^n$是连续可微函数,记函数$\mathbf{f}$的导函数为$\mathbf{L}\left(\mathbf{x}\right)\equiv\mathrm{d}_{\mathbf{x}=\mathbf{x}_0}\mathbf{f}\left(\mathbf{x}\right)$。若$\mathbf{L}\left(\mathbf{x}_0\right)$是单射,则总存在$\mathbf{x}_0$的一个邻域$N$使得$\mathbf{f}$在$N$上有连续可导的逆函数$\mathbf{f}^{-1}$;$\mathbf{f}$的像的集合$\mathbf{f}\left(N\right)$也是开集;对$N$内任意一点$\mathbf{x}$都有
    \[
        \mathrm{d}_{\mathbf{y}=\mathbf{f}\left(\mathbf{x}\right)}\mathbf{f}^{-1}\left(\mathbf{y}\right)=\mathbf{L}^{-1}\left(\mathbf{x}\right)
    \]
    其中$\mathbf{y}=\mathbf{f}\left(\mathbf{x}\right)$。
\end{theorem}
\begin{proof}
    我们先列出引理\ref{thm:inv_func_l2}和\ref{thm:inv_func_l3}的结论。由于$\mathbf{f}$在$\mathbf{x}_0$处连续可微,且$\mathbf{L}\left(\mathbf{x}_0\right)$是单射,故:
    \begin{itemize}
        \item 由引理\ref{thm:inv_func_l2},对$\mathbf{x}_0$的任一邻域$N=\left\{\mathbf{x}|\left\|\mathbf{x}-\mathbf{x}_0\right\|<\delta\right\}$($\delta>0$为任一正实数),都能找到正实数$M\left(\mathbf{y}\right)>0$满足$\left\|\mathbf{L}\left(\mathbf{x}\right)\mathbf{y}\right\|\geq M\left\|\mathbf{y}\right\|\forall\mathbf{y}\in\mathbb{R}^n$。进一步地,再由引理\ref{thm:inv_func_l1}可知导函数$\mathbf{L}\left(\mathbf{x}\right)$在$N$上的每个值都是单射线性变换。再由于$\mathbf{L}\left(\mathbf{x}\right)$在$N$上都是单射线性变换且其定义域和陪域维数相同,故$\mathbf{L}\left(\mathbf{x}\right)$在$N$上的每个值都是双射(同构)线性变换。
        \item 由引理\ref{thm:inv_func_l3},对$N$内部任一$\mathbf{x}^\prime$,总能找到正实数$M^\prime\left(\mathbf{x}^\prime\right)>0$满足$\left\|\mathbf{f}\left(\mathbf{x}^\prime\right)-\mathbf{f}\left(\mathbf{x}\right)\right\|\geq M^\prime\left\|\mathbf{x}^\prime-\mathbf{x}\right\|$。
    \end{itemize}
    我们令$M^\prime=M$,这相当于联系了$\mathbf{y}$和$\mathbf{x}^\prime$。

    我们证明的任务包括:
    \begin{enumerate}[label=\Roman*]
        \item\label{thm:inv_func_sub1} 函数$\mathbf{f}$存在逆函数$\mathbf{f}^{-1}$;
        \item\label{thm:inv_func_sub2} 开集$N$经$\mathbf{f}$的像集$\mathbf{f}\left(N\right)$也是开集;
        \item\label{thm:inv_func_sub3} $\forall \mathbf{x}\in N,\mathbf{L}^{-1}\left(\mathbf{x}\right)$是$\mathbf{f}^{-1}$的导数;
        \item\label{thm:inv_func_sub4} $\mathbf{f}^{-1}$连续可微。
    \end{enumerate}

    \ref{thm:inv_func_sub1}的证明:由引理\ref{thm:inv_func_l3}的结论,若$\mathbf{x}^\prime\neq\mathbf{x}$则$\mathbf{f}\left(x\right)\neq\mathbf{f}\left(\mathbf{x}^\prime\right)$,即$\mathbf{f}$是单射,故必存在逆$\mathbf{f}^{-1}$。\ref{thm:inv_func_sub1}证毕。

    \ref{thm:inv_func_sub2}的证明:首先我们确认一些比较直接的接论:
    \begin{itemize}
        \item 由于$N$是开集,故对任一$\mathbf{x}_1\in N$,总能找到足够小的$\delta_1$使得$B=\left\{\mathbf{x}|\left\|\mathbf{x}-\mathbf{x}_1\right\|\leq\delta_1\right\}$在$N$的内部。注意这里的$B$是一个闭集。
        \item 由于函数$\mathbf{f}$存在逆函数$\mathbf{f}^{-1}$,故$\mathbf{x}\in N\Leftrightarrow N\ni \mathbf{x}= \mathbf{f}^{-1}\left(\mathbf{f}\left(\mathbf{x}\right)\right)\mathbf{f}^{-1}\left(\mathbf{y}\right)\forall \mathbf{y}\in\mathbf{f}\left(N\right)=\left\{\mathbf{y}|\mathbf{y}=\mathbf{f}\left(\mathbf{x}\right),\mathbf{x}\in N\right\}$。即给定任一$\mathbf{y}_1\in\mathbf{f}\left(N\right)$有且只有一个$\mathbf{x}_1\in N$满足$\mathbf{f}\left(\mathbf{x}_1\right)=\mathbf{y}_1$。
    \end{itemize}

    要证明$\mathbf{f}\left(N\right)$是开集,就是要证明,对任一$\mathbf{y}_1\in\mathbf{f}\left(N\right)$,总能找到足够小的$\widetilde{M}>0$使得开集$C=\left\{\mathbf{y}|\left\|\mathbf{y}-\mathbf{y}_1\right\|<\widetilde{M}\right\}$在$\mathbf{f}\left(N\right)$的内部。

    如何由已知条件来找到这个$\widetilde{M}$呢?由于$N$是开集,我们通过$\mathbf{x}_1=\mathbf{f}^{-1}\left(\mathbf{y}_1\right)\in N$,可以找到使得闭集$B=\left\{\mathbf{x}|\left\|\mathbf{x}-\mathbf{x}_1\right\|\leq\delta_1\right\}$在$N$的内部的一个正实数$\delta_1$。

    如果$\widetilde{M}$存在,则对任一$\mathbf{y}\in C$,我们可以
    从$B$中找到一个$\mathbf{x}^\prime$使得$\mathbf{y}^\prime=\mathbf{f}\left(\mathbf{x}^\prime\right)$到$\mathbf{y}\in C$的距离最短,并由引理\ref{thm:inv_func_l3},总能找到足够小的正实数$M^\prime$使得
    \[\left\|\mathbf{f}\left(\mathbf{x}^\prime\right)-\mathbf{y}_1\right\|=\left\|\mathbf{f}\left(\mathbf{x}^\prime\right)-\mathbf{f}\left(\mathbf{x}_1\right)\right\|\geq M^\prime\left\|\mathbf{x}^\prime-\mathbf{x}_1\right\|=M^\prime\delta_1
    \]

    接下来我们将证明:
    \begin{enumerate}[label=\roman*]
        \item\label{thm:inv_func_sub2_sub1} 如果$\widetilde{M}=M^\prime\delta/2$,那么上述的$\mathbf{x}^\prime$在$B$的内部(即不在$B$的边界上);
        \item\label{thm:inv_func_sub2_sub2} 这一$\mathbf{y}^\prime$就是$\mathbf{y}$。
    \end{enumerate}
    上面两条若得证,则给定任一$\mathbf{y}_1\in\mathbf{f}\left(N\right)$,总有正实数$\widetilde{M}$(且具体地$\widetilde{M}=M^\prime\delta_1/2$)使得开集$C=\left\{\mathbf{y}|\left\|\mathbf{y}-\mathbf{y}_1\right\|<\widetilde{M}\right\}$在$\mathbf{f}\left(N\right)$的内部。\ref{thm:inv_func_sub2}也就得证了。

    \ref{thm:inv_func_sub2_sub1}的证明:反证法。设$\mathbf{x}^\prime$在$B$的边界上,即$\left\|\mathbf{x}^\prime-\mathbf{x}_1\right\|=\delta_1$,则由引理\ref{thm:inv_func_l3},总能找足够小的正实数$M^\prime$使得
    \[\left\|\mathbf{f}\left(\mathbf{x}^\prime\right)-\mathbf{y}_1\right\|=\left\|\mathbf{f}\left(\mathbf{x}^\prime\right)-\mathbf{f}\left(\mathbf{x}_1\right)\right\|\geq M^\prime\left\|\mathbf{x}^\prime-\mathbf{x}_1\right\|=M^\prime\delta_1
    \]
    那么,由三角不等式,对任一$\mathbf{y}\in C$(即总有$\left\|\mathbf{y}-\mathbf{y}_1\right\|<M^\prime\delta_1/2$),
    \begin{align*}
        \left\|\mathbf{f}\left(\mathbf{x}^\prime\right)-\mathbf{y}\right\| & \geq\left\|\mathbf{f}\left(\mathbf{x}^\prime\right)-\mathbf{y}_1\right\|-\left\|\mathbf{y}-\mathbf{y}_1\right\| \\
                                                                           & >M^\prime\delta_1-\left\|\mathbf{y}-\mathbf{y}_1\right\|                                                        \\
                                                                           & >M^\prime\delta_1-\frac{M^\prime\delta_1}{2}                                                                    \\
                                                                           & =\frac{M^\prime\delta_1}{2}                                                                                     \\
                                                                           & >\left\|\mathbf{y}-\mathbf{y}_2\right\|                                                                         \\
                                                                           & =\left\|\mathbf{f}\left(\mathbf{x}_1\right)-\mathbf{y}\right\|
    \end{align*}
    但这与“$\mathbf{y}^\prime$到$\mathbf{y}$的距离最短”相矛盾,故$\mathbf{x}^\prime$在$B$的内部。

    \ref{thm:inv_func_sub2_sub2}的证明:设到$\mathbf{y}$的距离平方函数
    \[
        g\left(\mathbf{x}\right)=\left\|\mathbf{f}\left(\mathbf{x}\right)-\mathbf{y}\right\|^2=\left(\mathbf{f}\left(\mathbf{x}\right)-\mathbf{y}\right)\cdot\left(\mathbf{f}\left(\mathbf{x}\right)-\mathbf{y}\right)\]
    则$\mathbf{x}^\prime$应使得该函数的一阶导数等于零,即$\left.\frac{dg\left(\mathbf{x}\right)}{d\mathbf{x}}\right|_{\mathbf{x}=\mathbf{x}^\prime}=\mathbf{0}$(零变换)。由零变换性质和链式法则,对任一$\mathbf{z}\in\mathbb{R}^n$,
    \[0=\left.\frac{dg\left(\mathbf{x}\right)}{d\mathbf{x}}\right|_{\mathbf{x}=\mathbf{x}^\prime}\mathbf{z}=2\left(\mathbf{f}\left(\mathbf{x}^\prime\right)-\mathbf{y}\right)\cdot\left(\mathbf{L}\left(\mathbf{x}^\prime\right)\mathbf{z}\right)\]
    由于$\mathbf{L}\left(\mathbf{x}\right)$在$N$上的每个值都是双射(同构)线性变换,故有且只有一个向量$\mathbf{z}\in\mathbb{R}^n$满足$\mathbf{L}\left(\mathbf{x}^\prime\right)\mathbf{z}=\mathbf{f}\left(\mathbf{x}^\prime\right)-\mathbf{y}$。故上式$\Leftrightarrow$
    \[0=2\left(\mathbf{f}\left(\mathbf{x}^\prime\right)-\mathbf{y}\right)\cdot\left(\mathbf{f}\left(\mathbf{x}^\prime\right)-\mathbf{y}\right)\Leftrightarrow\mathbf{f}\left(\mathbf{x}^\prime\right)-\mathbf{y}=\mathbf{0}\]
    即,只要$\mathbf{x}^\prime\in N$是使$\mathbf{y}^\prime=\mathbf{f}\left(\mathbf{x}^\prime\right)$到任一$\mathbf{y}\in \mathbf{f}\left(N\right)$的距离最短的点,则$\mathbf{f}\left(\mathbf{x}^\prime\right)=\mathbf{y}^\prime=\mathbf{y}$。\ref{thm:inv_func_sub2_sub2}证毕。\ref{thm:inv_func_sub2}证毕。

    \ref{thm:inv_func_sub3}的证明:按照导数的定义,相当于要证明对任一$\mathbf{x}\in N$,极限
    \[\lim_{\mathbf{f}\left(\mathbf{x}^\prime\right)\to\mathbf{f}\left(\mathbf{x}\right)}\frac{\mathbf{x}^\prime-\mathbf{x}-\mathbf{L}^{-1}\left(\mathbf{x}\right)\left(\mathbf{f}\left(\mathbf{x}^\prime\right)-\mathbf{f}\left(\mathbf{x}\right)\right)}{\left\|\mathbf{f}\left(\mathbf{x}^\prime\right)-\mathbf{f}\left(\mathbf{x}\right)\right\|}=\mathbf{0}\]
    令未求极限前的比增量为$\mathbf{s}$,即
    \[\mathbf{s}=\frac{\mathbf{x}^\prime-\mathbf{x}-\mathbf{L}^{-1}\left(\mathbf{x}\right)\left(\mathbf{f}\left(\mathbf{x}^\prime\right)-\mathbf{f}\left(\mathbf{x}\right)\right)}{\left\|\mathbf{f}\left(\mathbf{x}^\prime\right)-\mathbf{f}\left(\mathbf{x}\right)\right\|}\]

    由于$\mathbf{L}\left(\mathbf{x}\right)$在$\mathbf{x}\in N$内都有定义,故极限
    \[\lim_{\mathbf{x}^\prime\to\mathbf{x}}\frac{\mathbf{f}\left(\mathbf{x}^\prime\right)-\mathbf{f}\left(\mathbf{x}\right)-\mathbf{L}\left(\mathbf{x}\right)\left(\mathbf{x}^\prime-\mathbf{x}\right)}{\left\|\mathbf{x}^\prime-\mathbf{x}\right\|}=\mathbf{0}\]
    令
    \[\mathbf{r}=\frac{\mathbf{f}\left(\mathbf{x}^\prime\right)-\mathbf{f}\left(\mathbf{x}\right)-\mathbf{L}\left(\mathbf{x}\right)\left(\mathbf{x}^\prime-\mathbf{x}\right)}{\left\|\mathbf{x}^\prime-\mathbf{x}\right\|}\]
    则$\lim_{\mathbf{x}^\prime\to\mathbf{x}}\mathbf{r}=\mathbf{0}$。
    对$\mathbf{x}^\prime\in N,\mathbf{x}^\prime\neq\mathbf{x}$,$\mathbf{s}$可由$\mathbf{r}$表示为
    \[
        \mathbf{s}=-\frac{\left\|\mathbf{x}^\prime-\mathbf{x}\right\|}{\left\|\mathbf{f}\left(\mathbf{x}^\prime\right)-\mathbf{f}\left(\mathbf{x}\right)\right\|}\mathbf{L}^{-1}\left(\mathbf{x}\right)\mathbf{r}
    \]

    由引理\ref{thm:inv_func_l3},存在足够小正实数$\mathbf{M}$使得$\left\|\mathbf{f}\left(\mathbf{x}^\prime\right)-\mathbf{f}\left(\mathbf{x}\right)\right\|\geq M\left\|\mathbf{x}^\prime-\mathbf{x}\right\|$,故有
    \[0\geq-\frac{\left\|\mathbf{x}^\prime-\mathbf{x}\right\|}{\left\|\mathbf{f}\left(\mathbf{x}^\prime\right)-\mathbf{f}\left(\mathbf{x}\right)\right\|}\geq-\frac{1}{M}\]
    即$-\frac{\left\|\mathbf{x}^\prime-\mathbf{x}\right\|}{\left\|\mathbf{f}\left(\mathbf{x}^\prime\right)-\mathbf{f}\left(\mathbf{x}\right)\right\|}$是有界的。

    又由定理\ref{thm:II.4.3}的推论,线性变换都是连续函数,故由复合函数的连续性,极限
    \[\lim_{\mathbf{x}^\prime\to\mathbf{x}}\mathbf{L}^{-1}\left(\mathbf{x}\right)\mathbf{r}=\mathbf{L}^{-1}\left(\mathbf{x}\right)\lim_{\mathbf{x}^\prime\to\mathbf{x}}\mathbf{r}=\mathbf{0}\]

    由于一个有界函数与一个有极限的函数的积的极限等于那个有极限的函数的极限\footnote{这个基本定理可由极限的$\delta-\epsilon$语言证明,很多地方有,此略。},故$\lim_{\mathbf{x}^\prime\to\mathbf{x}}\mathbf{s}=\mathbf{0}$。又由于当$\mathbf{x}^\prime\to\mathbf{x}$时$\mathbf{f}\left(\mathbf{x}^\prime\right)\to\mathbf{f}\left(\mathbf{x}\right)$,由$\mathbf{s}$的形式有$\lim_{\mathbf{x}^\prime\to\mathbf{x}}\mathbf{s}=\mathbf{0}\Leftrightarrow\lim_{\mathbf{f}\left(\mathbf{x}^\prime\right)\to\mathbf{f}\left(\mathbf{x}\right)}\mathbf{s}=\mathbf{0}$。\ref{thm:inv_func_sub3}证毕。

    \ref{thm:inv_func_sub4}的证明:要证$\mathbf{f}^{-1}$的导函数连续,即对任一$\mathbf{x}_1\in N, \mathbf{y}_1=\mathbf{f}\left(\mathbf{x}_1\right)$有
    \[\lim_{\mathbf{y}\to\mathbf{y}_1}\left.\frac{d\mathbf{f}^{-1}\left(\mathbf{y}\right)}{d\mathbf{y}}\right|_{\mathbf{y}=\mathbf{y}}=\left.\frac{d\mathbf{f}^{-1}\left(\mathbf{y}\right)}{d\mathbf{y}}\right|_{\mathbf{y}=\mathbf{y}_1}\]
    由\ref{thm:inv_func_sub3}的证明我们已经有
    \[
        \left.\frac{d\mathbf{f}^{-1}\left(\mathbf{y}\right)}{d\mathbf{y}}\right|_{\mathbf{y}=\mathbf{y}}=\mathbf{L}^{-1}\left(\mathbf{x}\right),\mathbf{y}=\mathbf{f}\left(\mathbf{x}\right),\forall\mathbf{x}\in N\]
    故只需证
    \[\lim_{\mathbf{x}\to\mathbf{x}_1}\mathbf{L}^{-1}\left(\mathbf{x}\right)=\mathbf{L}^{-1}\left(\mathbf{x}_1\right)\]

    由引理\ref{thm:inv_func_l3},总存在足够小正实数$M$满足$\left\|\mathbf{L}\left(\mathbf{x}\right)\mathbf{y}\right\|\geq M\left\|\mathbf{y}\right\|\forall\mathbf{y}\in\mathbb{R}^n$,故令$\mathbf{z}=\mathbf{L}\left(x\right)\mathbf{y}$,则$\left\|\mathbf{z}\right\|\geq M\left\|\mathbf{L}^{-1}\left(\mathbf{x}\right)\mathbf{z}\right\|$。

    由于$\mathbf{f}$是连续可微函数,设$\mathbf{x}_1\in N$,对任一$\epsilon^\prime>0$,总存在$\delta>0$,使得只要$\left\|\mathbf{x}-\mathbf{x}_1\right\|<\delta$就有$\left\|\mathbf{L}\left(\mathbf{x}\right)-\mathbf{L}\left(\mathbf{x}_1\right)\right\|<\epsilon^\prime$。具体的,设由$\delta$定义的$\mathbf{x}_1$的邻域$N_1=\left\{\mathbf{x}|\left\|\mathbf{x}-\mathbf{x}_1\right\|\right\}<\delta$在$N$的内部,则对任一$\mathbf{x}\in N_1$,以下不等式成立
    \begin{align*}
        \left\|\left(\mathbf{L}^{-1}\left(\mathbf{x}\right)-\mathbf{L}^{-1}\left(\mathbf{x}_1\right)\right)\mathbf{z}\right\| & =\left\|\mathbf{L}^{-1}\left(\mathbf{x}\right)\left(\mathbf{L}\left(\mathbf{x}\right)-\mathbf{L}\left(\mathbf{x}_1\right)\right)\mathbf{L}^{-1}\left(\mathbf{x}_1\right)\mathbf{z}\right\| \\
                                                                                                                              & \leq\frac{1}{M}\left\|\left(\mathbf{L}\left(\mathbf{x}\right)-\mathbf{L}\left(\mathbf{x}_1\right)\right)\mathbf{L}^{-1}\left(\mathbf{x}_1\right)\mathbf{z}\right\|                         \\
                                                                                                                              & \leq\frac{1}{M}\left\|\mathbf{L}\left(\mathbf{x}\right)-\mathbf{L}\left(\mathbf{x}_1\right)\right\|\left\|\mathbf{L}^{-1}\left(\mathbf{x}_1\right)\mathbf{z}\right\|                       \\
                                                                                                                              & \leq\frac{1}{M^2}\left\|\mathbf{L}\left(\mathbf{x}\right)-\mathbf{L}\left(\mathbf{x}_1\right)\right\|\left\|\mathbf{z}\right\|
    \end{align*}
    由线性变换的范的定义(最大下界),上述不等式$\Leftrightarrow$
    \[\left\|\mathbf{L}^{-1}\left(\mathbf{x}\right)-\mathbf{L}^{-1}\left(\mathbf{x}_1\right)\right\|\leq\frac{1}{M^2}\left\|\mathbf{L}\left(\mathbf{x}\right)-\mathbf{L}\left(\mathbf{x}_1\right)\right\|\leq\frac{\epsilon^\prime}{M^2}\]
    令$\epsilon=\frac{\epsilon^\prime}{M^2}$,我们就有对于任一$\mathbf{x}_1\in N$和任一$\epsilon>0$,总有$\delta>0$使得只要$\left\|\mathbf{x}-\mathbf{x}_1\right\|<\delta$就有$\left\|\mathbf{L}^{-1}\left(\mathbf{x}\right)-\mathbf{L}^{-1}\left(\mathbf{x}_1\right)\right\|<\epsilon$。具体地,这个$\delta$总存在是由于$M$总存在。这相当于说$\lim_{\mathbf{x}\to\mathbf{x}_1}\mathbf{L}^{-1}\left(\mathbf{x}\right)=\mathbf{L}^{-1}\left(\mathbf{x}_1\right)$,\ref{thm:inv_func_sub4}证毕。
\end{proof}
\end{document}