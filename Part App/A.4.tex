\documentclass[main.tex]{subfiles}
\begin{document}
我们先引入关于子群的共轭子群的概念。

\begin{definition}\label{def:A.7}
    设$G$是一个群,对任意$a,b\in G$,记其群操作为$a\circ b$,单位元为$e$,$a$的逆元为$a^{-1}$。若$G$中存在一个$i\in G$使$G$中的两个元素$a,b\in G$满足$b=i\circ a\circ i^{-1}$,则称$b$为$a$的一个\emph{共轭(conjugate)}。
\end{definition}

给定$i\in G$,设关系$\sim\subset G\times G$满足$b=i\circ a\circ i^{-1},\forall\left(a,b\right)\in\sim$,则易证$\sim$是一个等价关系。给定一个$G$的元素$a$,由关系$\sim$可形成一个等价类$\left\llbracket a\right\rrbracket_\sim$,称$a$的\emph{共轭类(conjugacy class)}。

\begin{lemma}\label{lem:A.3}
    设$G$是一个群,对任意$a,b\in G$,记其群操作为$a\circ b$,单位元为$e$,$a$的逆元为$a^{-1}$。若$V$是$G$的一个子群。给定$i\in G$,定义映射$Q:\mathcal{V}\rightarrow G,$
    \[Q\left(v\right)=i\circ v\circ i^{-1},\quad\forall v\in V\]
    则集合$V^*\equiv\mathrm{ran}Q$是$G$的一个子群,且映射$Q$是由$V$到$V^*$的同构映射。
\end{lemma}

该引理的证明是直接的,留作练习。

\begin{definition}[共轭子群]\label{def:A.8}
    设$G$是一个群,对任意$a,b\in G$,记其群操作记为$a\circ b$,单位元为$e$,$a$的逆元为$a^{-1}$。若$V$是$G$的一个子群。给定$i\in G$,定义映射$Q:\mathcal{V}\rightarrow G,$
    \[Q\left(v\right)=i\circ v\circ i^{-1}\]
    则称$\mathcal{V}^*$是$\mathcal{V}$的一个共轭子群。
\end{definition}

\begin{lemma}\label{lem:A.4}
    设$\left(\mathcal{E},d\right)$是一个度量空间,若其上的等距群存在一个满足条件G1至G6、S1至S4和N1的子群$\mathcal{V}$,则这样的子群只有一个。
\end{lemma}
\begin{proof}
    设$\mathcal{V}^\prime$也是满足上述条件的子群,则给定任意两点$X,Y\in\mathcal{E}$,$\mathcal{V}$中必存在唯一一个等距变换$\mathbf{u}$、$\mathcal{V}^\prime$中必存在唯一一个等距变换$\mathbf{u}^\prime$,分别满足$\mathbf{u}=Y-X,\mathbf{u}^\prime=Y-^\prime X$,其中$-$和$-^\prime$分别是由$\mathcal{V}$和$\mathcal{V}^\prime$的传递性所定义的记号。相应地有$Y=X+\mathbf{u}=X+^\prime\mathbf{u}^\prime$。

    定义由$\mathcal{V}$到$\mathcal{V}^\prime$的映射$f:\mathcal{V}\rightarrow\mathcal{V}^\prime,$
    \[X+\mathbf{u}=X+^\prime f\left(\mathbf{u}\right), \quad\forall X\in\mathcal{E},\forall\mathbf{u}\in\mathcal{V}\]
    易证$f$是双射且$f\left(\mathbf{0}\right)=\mathbf{0}$,其中$\mathbf{0}$是$\mathcal{E}$上的恒等变换,故它同时是$\mathcal{V}$和$\mathcal{V}^\prime$的单位元。

    $f$是由$\mathcal{V}$到$\mathcal{V}^\prime$的同构映射:对任意$\mathbf{u},\mathbf{v}\in\mathcal{V}$和$X_0\in\mathcal{V}$必存在$X,Y\in\mathcal{V}$满足
    \[X=X_0+\mathbf{u},Y=X_0+\mathbf{v}=X+\mathbf{v}-\mathbf{u}\]
    若记$\mathbf{u}^\prime=f\left(\mathbf{u}\right),\mathbf{v}^\prime=f\left(\mathbf{v}\right)$,则亦有
    \[X=X_0+^\prime\mathbf{u}^\prime,Y=X+^\prime\mathbf{u}^\prime-\mathbf{v}^\prime\]
    由$\mathcal{V}$和$\mathcal{V}^\prime$上的范的定义方式都来自同一度量$d$,故有
    \[\left\|\mathbf{v}-\mathbf{u}\right\|=\left\|\mathbf{v}^\prime-\mathbf{u}^\prime\right\|\]
    特别地,当$\mathbf{u}=\mathbf{0}$时,$\mathbf{u}^\prime=\mathbf{0}$,上式说明$\left\|\mathbf{v}\right\|=\left\|\mathbf{v}^\prime\right\|$对任意$\mathbf{v}\in\mathcal{V}$成立。故
    \begin{align*}
        \left\|\mathbf{v}-\mathbf{u}\right\|                                                                       & =\left\|\mathbf{v}^\prime-\mathbf{u}^\prime\right\|                                                                      \\
        \Rightarrow\left\|\mathbf{v}-\mathbf{u}\right\|^2                                                          & =\left\|\mathbf{v}^\prime-\mathbf{u}^\prime\right\|^2                                                                    \\
        \Leftrightarrow\left\|\mathbf{u}\right\|^2-2\left(\mathbf{u}|\mathbf{v}\right)+\left\|\mathbf{v}\right\|^2 & =\left\|\mathbf{u}^\prime\right\|^2-2\left(\mathbf{u}^\prime|\mathbf{v}^\prime\right)+\left\|\mathbf{v}^\prime\right\|^2 \\
        \Leftrightarrow\left(\mathbf{u}|\mathbf{v}\right)                                                          & =\left(\mathbf{u}^\prime|\mathbf{v}^\prime\right)
    \end{align*}
    即$f$保持内积。由类似引理\ref{lem:II.2.2}的证明过程可知$f$是由$\mathcal{V}$到$\mathcal{V}^\prime$的同构线性变换。

    由外延公理,如果两个集合$A$和$B$之间存在一个双映射$f$满足$f\left(x\right)=x,\forall x\in A$则$A=B$。对任意$X\in\mathcal{E}$和$\mathbf{v}\in\mathcal{V}$,设$\mathbf{u}=X-X_0,X_0\in\mathcal{E}$,则
    \begin{align*}
        X+\mathbf{v} & =X_0+\mathbf{u}+\mathbf{v}                                     \\
                     & =X_0+^\prime f\left(\mathbf{u}+\mathbf{v}\right)               \\
                     & =X_0+^\prime f\left(\mathbf{u}\right)+f\left(\mathbf{v}\right) \\
                     & =X_0+\mathbf{u}+^\prime f\left(\mathbf{v}\right)               \\
                     & =X+f\left(\mathbf{v}\right)
    \end{align*}
    均成立,故$f\left(\mathbf{v}\right)=\mathbf{v},\forall\mathbf{v}\in\mathcal{V}$,即$\mathcal{V}^\prime$与$\mathcal{V}$作为内积空间是同一的。
\end{proof}

\begin{theorem*}[等距变换的表示定理\ref{thm:II.3.2}]
    设$\mathcal{E}$是一个欧几里得空间,$\mathcal{V}$是其平移空间,选定任一点$X_0\in\mathcal{E}$,则$\mathcal{E}$上的任一等距变换$i:\mathcal{E}\rightarrow\mathcal{E},i\in\mathcal{V}$都可表示为
    \[
        i\left(X\right)=i\left(X_0\right)+\mathbf{Q}_i\left(X-X_0\right)
    \]
    其中$\mathbf{Q}_i$是一个正交算符,由$i$唯一确定。
\end{theorem*}
\begin{proof}
    由引理\ref{lem:A.3}、\ref{lem:A.4}和\ref{lem:II.2.2},给定任一$\mathcal{E}$上的等距变换$i$,$\mathcal{V}$的共轭子群都是它自己,且每个$i$引出的共轭映射$\mathbf{Q}_i\mathbf{v}=i\circ\mathbf{v}\circ i^{-1}$就是$\mathcal{V}$上的自同构映射,故$\mathbf{Q}$是$\mathcal{V}$上的正交算符。

    注意到$i\circ\mathbf{v}=\mathbf{Q}_i\mathbf{v}\circ i$,则对任一$X_0\in\mathcal{E}$,有
    \[i\circ \mathbf{v}\left(X_0\right)=i\left(X+0+\mathbf{v}\right),\quad \mathbf{Q}_i\mathbf{v}\circ i\left(X_0\right)=i\left(X_0\right)+\mathbf{Q}_i\mathbf{v}\]
    令$X=X_0+\mathbf{v}$,则有
    \[
        i\left(X\right)=i\left(X_0\right)+\mathbf{Q}_i\left(X-X_0\right)
    \]

    $\mathbf{Q}_i$由$i$唯一确定:设另有一正交算符$\mathbf{P}:\mathcal{V}\rightarrow\mathcal{V}$满足$\mathbf{Pu}=i\left(X+\mathbf{u}\right)-i\left(X\right),\forall\mathbf{u}\in\mathcal{V},\forall X\in\mathcal{E}$,则$\left(\mathbf{P}-\mathbf{Q}_i\right)\mathbf{u}=\mathbf{Pu}-\mathbf{Q}_i\mathbf{u}=\mathbf{0},\forall\mathbf{u}\in\mathcal{V}$。
\end{proof}
\end{document}