\documentclass[main.tex]{subfiles}
% 线性算符的行列式、迹、特征值分解
\begin{document}
在\S\ref{sec:II.4.1}中我们介绍了线性算符(定义\ref{def:II.4.3})。本节我们以引入线性算符的几种取值不依赖基的选择的标量参数。

%========================================================
\subsection{线性算符的行列式}
\begin{definition}[$n\times n$矩阵的行列式]
设$A$是数域$\mathbb{F}$上的$n\times n$矩阵(记为$A\in\mathbb{F}^{n\times n}$),$a_i=\left(A_{i1},\cdots,A_{in}\right),i=1,\cdots,n$是$A$的第$i$行的有序数组,如果函数$D:\mathbb{F}^{n\times n}\rightarrow\mathbb{F}$满足:
\begin{enumerate}
\item $D$是关于$\left(a_1,\cdots,a_n\right)$的$n$重线线性函数,即
\[
D\left(a_1,\cdots,\gamma a_i+a^\prime_i,\cdots,a_n\right)=\gamma D\left(a_1,\cdots,a_i,\cdots,a_n\right)+D\left(a_1,\cdots,a^\prime_i,\cdots,a_n\right),\forall\gamma\in\mathbb{F}
\]
\item 当$A$的任意两行相等则$D\left(A\right)=0$
\item 设$A^\prime$是$A$的任意两行调换后的矩阵,则$D\left(A^\prime\right)=-D\left(A\right)$
\item $D\left(I\right)=1$,其中$I$是$n\times n$单位矩阵
\end{enumerate}
则称$D$是$A$的行列式(determinant),记为$\mathrm{det}A$。
\end{definition}
上述定义中的条件1是本身是$n$重线性函数($n$-linear function)的定义,条件1和条件2一起则是交替(alternating)$n$重线性函数的定义。条件3是可由条件1和2独立证明得到的(此略\cite[\S 5.2,pp.~144\~146]{Hoffman1971}),加到了定义中只是为了便于理解。我们可以说,行列式就是一个关于$n\times n$矩阵的交替$n$重线性标量值函数且其还满足条件4。

上面的定义隐含默认了任意一个$n\times n$矩阵的行列式是唯一存在(因函数的定义)的要求,但这未被证明。本讲义暂不列出详细证明的证明过程,只简述证明的思路。证明存在性往往等于找到这一存在。回顾本科的线性代数课本,我们发现那里的定义方式\cite[\S 1.3“定义3.1”,p.~7]{周胜林2012线性代数}是直接列出公式
\[
\mathrm{det}A=\sum_{\sigma}\left(\mathrm{sgn}\sigma\right)A\left(1,\sigma_1\right)\cdots A\left(n,\sigma_n\right)
\]
其中$A\left(i,j\right)\equiv A_{ij}$;$\sigma_i,i=1,\cdots,n$表示$n$阶排列$\sigma$的一种;$\mathrm{sgn}\sigma$是根据$\sigma$的奇偶性取值1或$-1$\footnote{关于$n$阶排列的知识可参见\cite[\S1.2,p.~4]{周胜林2012线性代数}。}。作为存在性的证明,提出这一公式,然后验证其满足行列式定义中的条件,就完成了。而唯一性的证明则需要如下引理:任意$\mathbb{F}^{n\times n}$上的交替$n$重线性函数$D$均满足$D\left(A\right)=\mathrm{det}AD\left(I\right)\forall A\in\mathbb{F}^{n\times n}$,其中$\mathrm{det}A$在此只需存在即可。该引理的证明也可以从上述的行列式公式直接得出,使得行列式定义中的条件4实际成为行列式唯一性的必要条件。以下行列式的性质在本讲义中也不作证明而直接承认其成立。

\begin{theorem}
\quad
\begin{itemize}
    \item $\mathrm{det}\left(AB\right)=\mathrm{det}A\mathrm{det}B,\forall A,B\in\mathbb{F}^{n\times n}$
    \item $\mathrm{det}\left(A^\intercal\right)=\mathrm{det}A,\forall A,B\in\mathbb{F}^{n\times n}$
    \item $\mathrm{det}\left(A^{-1}\right)=\mathrm{det}A,\forall A,B\in\mathbb{F}^{n\times n}$
    \item 如果$B$是$A$的第$i$行加上第$j$行的倍数得到的矩阵,则$\mathrm{det}B=\mathrm{det}A$
    \item 克拉默法则\cite[\S1.5,p.~15]{周胜林2012线性代数}
\end{itemize}
\end{theorem}

定义在数域$\mathbb{F}$上的有限维向量空间$\mathcal{V}$上的线性算符$\mathbf{T}\in\mathcal{L}\left(\mathcal{V}\right)$在不同基下的坐标矩阵之间有变换公式(见\S\ref{sec:II.4.2})。设$B,B^\prime$是$\mathcal{V}$的两组基,$S$是从$B$到$B^\prime$的过渡矩阵,则由定理\ref{thm:II.5.2}有$\left(\mathbf{T}\right)=S\left(\mathbf{T}\right)^\prime S^{-1}$,其中$\left(\mathbf{T}\right),\left(\mathbf{T}\right)^\prime$分别是$\mathbf{T}$在$B,B^\prime$下的坐标矩阵。由行列式的性质有$\mathrm{det}\left(\mathbf{T}\right)=\mathrm{det}\left[S\left(\mathbf{T}\right)S^{-1}\right]=\mathrm{det}\left(\mathbf{T}\right)^\prime$,因此一个线性算符在任意基下的坐标矩阵的行列式都相等,我们因此可以定义“线性算符的行列式”,记为$\mathrm{det}\mathbf{T}$,为其在任一基下的坐标矩阵的行列式。正式定义如下。

\begin{definition}[线性算符的行列式]
设$\mathbf{T}$是$n$维向量空间$\mathcal{V}$上的线性算符,$B$是$\mathcal{V}$的一组有序基,则$\mathbf{T}$的行列式$\mathrm{det}\mathbf{T}\equiv\mathrm{det}\left(\mathbf{T}\right)$,其中$\left(\mathbf{T}\right)$是$\mathbf{T}$在$B$下的坐标矩阵。
\end{definition}

%===========================================================
\subsection{线性算符的迹}
从线性算符的行列式定义过程我们发现,与本科线性代数课上直接定义成一个运算公式不同,我们总是先定义代数规则,再去证明这样的代数规则只唯一对应一个具体的运算公式。线性算符的迹也可按类似的方式重新定义。不过,既然已经通过行列式的例子来了解这种思想,此处关于迹的定义仍采用简化的方式。值得注意的是,迹是定义在内积空间上的线性算符上的。

\begin{definition}[线性算符的迹]
设$\mathbf{A}$是数域$\mathbb{F}$上的内积空间$\mathbf{V}$上的线性算符,$\mathbf{A}$的迹(trace)$\mathrm{tr}\mathbf{A}=\sum_k\left(\mathbf{A}\mathbf{\hat{e}}_k|\mathbf{\hat{e}}_k\right)$,其中$\left\{\mathbf{\hat{e}}_k\right\}$是$\mathcal{V}$的一组规范正交基。
\end{definition}

易验,上述定义的迹的值不依赖基的选择而改变,从而任一线性算符唯一对应一个迹。我们还能进一步获得如下性质。

\begin{theorem}
设$\mathbf{A}$是数域$\mathbb{F}$上的内积空间$\mathbf{V}$上的线性算符,
\begin{enumerate}
    \item $\mathrm{tr}\left(\alpha\mathbf{A}+\mathbf{B}\right)=\alpha\mathrm{tr}\mathbf{A}+\mathrm{tr}\mathbf{B}$
    \item $\mathrm{tr}\left(\mathbf{AB}\right)=\mathrm{tr}\left(\mathbf{BA}\right)$
\end{enumerate}
\end{theorem}

我们在此还可通过迹来定义$\mathcal{L}\left(\mathcal{V}\right)$空间上的一种内积。

\begin{definition}[线性算符的标准内积]
设$\mathbf{A},\mathbf{B}$是数域$\mathbb{F}$上的内积空间$\mathbf{V}$上的任意两个线性算符,令$\left(\mathbf{A}|\mathbf{B}\right)\equiv\mathrm{tr}\left(\mathbf{A}^*\mathbf{B}\right)$,可验证,该定义满足内积规定,称为线性算符的标准内积,以经常记作$\mathbf{A}:\mathbf{B}$
\end{definition}

%======================================================================
\subsection{线性算符的特征值}
如果一个数域$\mathbb{F}$上的有限维向量空间$\mathcal{V}$上的线性算符$\mathbf{T}$在基$\left\{\mathbf{a}_i\right\}$下的坐标矩阵是对角矩阵,
\[\left(\mathbf{T}\right)=\left(\begin{array}{ccc}\lambda_1&&\\&\ddots&\\&&\lambda_n\end{array}\right),\lambda_i\in\mathbb{F},i=1,\cdots,n\]
那么就有$\mathbf{Ta}_i=\lambda_i \mathbf{a}_i,i=1,\cdots,n$,$\mathbf{T}$的值域就是由$\lambda_i\neq 0$对应的$\mathbf{a}_i$线性生成的,$\lambda_i\neq 0$的个数就是$\mathrm{rank}\mathbf{T}$,$\mathbf{T}$的很多性质就能确定了。为了这一目标,我们先讨论一般的形如$\mathbf{Ta}=\lambda\mathbf{a}$的情况。

\begin{definition}[特征值、特征向量、特征空间]
设$\mathcal{V}$是数域$\mathbb{F}$上的有限维向量空间,$\mathbf{T}\in\mathcal{L}\left(\mathcal{V}\right)$是一个线性算符。若存在$\mathbf{a}\neq\mathbf{0},\mathbf{a}\in\mathcal{V},\gamma\in\mathbb{F}$满足$\mathbf{Ta}=\gamma\mathbf{a}$,则称$\gamma$是$\mathbf{T}$的一个特征值(characteristic value),$\mathbf{a}$是$\mathbf{T}$的一个对应特征值$\gamma$的特征向量(characteristic vector)。$\mathbf{T}$的所有不同的特征向量的集合称为$\mathbf{T}$的特征空间(characteristic space)。
\end{definition}

我们马上且迅速地解决一条定理。

\begin{theorem}
设$\mathcal{V}$是数域$\mathbb{F}$上的有限维向量空间,$\mathbf{T}\in\mathcal{L}\left(\mathcal{V}\right)$是一个线性算符,则$\mathbf{T}$的特征空间是$\mathcal{V}$的子空间。
\end{theorem}
\begin{proof}
设$\mathbf{a},\mathbf{b}\in\mathcal{V}$是关于$\mathbf{T}$的特征值$\gamma$的其中两个特征向量,则对任意$\alpha\in\mathbb{F}$有$\mathbf{T}\left(\alpha\mathbf{a}+\mathbf{b}\right)=\alpha\mathbf{Ta}+\mathbf{Tb}=\alpha\gamma\mathbf{a}+\gamma\mathbf{b}=\gamma\left(\alpha\mathbf{a}+\mathbf{b}\right)$。
\end{proof}

由特征值和特征向量的定义,一个线性算符可能有多个特征值,对应同一个特征值也可以有多个特征向量。上述特征值的定义无法告诉我们回答“一个线性算符到底有多少个特征值”的方法,它甚至没有告诉我们除了碰运气之外“如何找到线性算符的任一个特征值”。为此我们进一步考虑一个重要的线性算符式。设$\mathcal{V}$是数域$\mathbb{F}$上的有限维向量空间,$\mathbf{T}\in\mathcal{L}\left(\mathcal{V}\right)$是一个线性算符,$\gamma$是$\mathbf{T}$的一个特征值,$\mathbf{a}$是$\mathbf{T}$关于$\gamma$的任一特征向量,则有
\[\left(\mathbf{T}-\gamma\mathbf{I}\right)\mathbf{a}=\mathbf{Ta}-\gamma\mathbf{Ia}=\mathbf{0}\]
即$\mathbf{T}$关于$\gamma$的特征空间是线性算符$\mathbf{T}-\gamma\mathbf{I}$的零空间。根据线性变换的维数定理,$\mathbf{T}$的特征空间的维数就是$\mathbf{T}-\gamma\mathbf{I}$的零化度,进而有如下定理。

\begin{theorem}
设$\mathcal{V}$是数域$\mathbb{F}$上的有限维向量空间,$\mathbf{T}\in\mathcal{L}\left(\mathcal{V}\right)$是一个线性算符,则以下命题相互等价:
\begin{enumerate}
    \item $\gamma$是$\mathbf{T}$的特征值
    \item $\mathbf{T}-\gamma\mathbf{I}$是奇异的
    \item $\mathrm{det}\left(\mathbf{T}-\gamma\mathbf{I}\right)=0$
\end{enumerate}
\begin{proof}
我们在本科的线性代数课中已经了解,矩阵行列式为零,就是该矩阵相对就的一个线性方程组无非全零解,亦即该矩阵是不可逆的。这些结论的详细证明就是对$2\Leftrightarrow 3$的证明。此略。$1\Leftrightarrow 3$可利用行列式的性质简单证得,此略。
\end{proof}
\end{theorem}

这一定理的第3个命题实际提供了一个求给定线性变换的特征值的方法。因为$\mathrm{det}\left(\mathbf{T}-\gamma\mathbf{I}\right)$必是一个$\mathrm{dim}\mathcal{V}$阶首一多项式\footnote{这一命题的结论需要依靠多项式代数的知识来证明。后文还要用到代数基本定理,这个定理也是通过多项式代数证明的重要结论,此略。}。因此可将此列式$\mathrm{det}\left(\mathbf{T}-\gamma\mathbf{I}\right)$定义为$\mathbf{T}$的特征多项式(characteristic polynomial)。于是$\mathbf{T}$的任一特征值均为其特征多项式的一个根。在复数域上,由代数基本定理,$n$阶首一多项式有且必有$n$个复根(可能有重根)\footnote{这样的数域又叫代数闭域。例如实数域就不是代数闭域。}。因此我们可以说复数域上任一$n$维线性空间上的线性算符必有$n$个特征值;解其特征多项式就可以得到它们全部。

本节开头提出的,希望线性变换在某组基下的坐标矩阵是对角矩阵,这相当于要求线性变换的特征向量就是一组基。我们可以从这一要求出发,找出与其等价的一系列命题,如以下定理所示。

\begin{theorem}
设$\mathcal{V}$是数域$\mathbb{F}$上的$n$维向量空间,$\mathbf{T}\in\mathcal{L}\left(\mathcal{V}\right)$是一个可逆线性算符,则以下命题相互等价:
\begin{enumerate}
    \item $\mathcal{V}$中存在一组基恰好就是$\mathbf{T}$的特征向量
    \item $\mathbf{T}$的特征向量中有$n$个线性无关
    \item $\mathbf{T}$的特征向量线性生成$\mathcal{V}$
    \item $\mathbf{T}$的特征空间维数是$n$
    \item $\mathbf{T}$是可对角化的(diagonalizable)
\end{enumerate}
\end{theorem}

以上定理中的命题之间的等价性几乎是直接的\cite[\S5.2“矩阵可对角化的条件”,p.123]{周胜林2012线性代数},第5条命题是为了定义“可对角化的线性变换”而设。如果$\mathbf{T}$可对角化,我们就有$n$个特征值$\left\{\gamma_i\right\}_{i=1}^n$,但它们之间有可能有重复的(即$\mathbf{T}$的特征多项式有重根)。设$\theta_1,\cdots,\theta_k$为$\mathbf{T}$的两两不同特征值,第$j$个特征值重复$d_j$次,则$\mathbf{T}$的对角化坐标矩阵可以写成
\[\left(\mathbf{T}\right)=\left(\begin{array}{ccc}\theta_1I_1&&\\&\ddots&\\&&\theta_kI_k\end{array}\right)\]
其中$I_j$是$d_j\times d_j$单位矩阵。由之前关于特征值的知识可知,每个两两不同特征值的重复次数$d_j$就是它对应的特征空间的维数,进而有$d_1+\cdots+d_k=n$。易验在相同的基下,算符$\mathbf{T}-\theta_j\mathbf{I},j=1,\cdots,k$的坐标矩阵也都是对角矩阵,它们各将含有$d_j$个零在对角线上。$\mathbf{T}-\theta_i\mathbf{I}$的零化度就是$d_i$。

以下介绍两个与特征值相关的定理\cite[“性质1、性质2、p.~118]{周胜林2012线性代数}。

\begin{theorem}
\begin{enumerate}
    \item 线性算符的的转置的特征值与原线性算符的特征值相同
    \item 设$\left\{\lambda_i\right\}$是线性算符$\mathbf{T}\in\mathcal{L}\left(\mathcal{V}\right)$的$n$个特征值,则$\mathrm{tr}\mathbf{T}=\sum_{i=1}^n\lambda_i$,$\mathrm{det}\mathbf{T}=\prod_{i=1}^n\lambda_i$,$\mathrm{det}\left(\mathbf{T}-\lambda\mathbf{I}\right)=\lambda^n-\mathrm{tr}\mathbf{T}\lambda^{n-1}+\cdots+\left(-1\right)^n\mathrm{det}\mathbf{T}$
\end{enumerate}
\end{theorem}
\begin{proof}
利用$n$阶行列式的计算公式证明,略。
\end{proof}

特别地,当$n=3$时,$\mathrm{det}\left(\mathbf{T}-\lambda\mathbf{I}\right)=-\lambda^3+\mathrm{tr}\mathbf{T}\lambda^2-\frac{1}{2}\left(\mathrm{tr}^2\mathbf{T}-\mathrm{tr}\mathbf{T}^2\right)\lambda+\mathrm{det}\mathbf{T}$。我们令$I_\mathbf{T}=\mathrm{tr}\mathbf{T},II_\mathbf{T}=\frac{1}{2}\left(\mathrm{tr}^2\mathbf{T}-\mathrm{tr}\mathbf{T}^2\right),III_\mathbf{T}=\mathrm{det}\mathbf{T}$,称为$\mathbf{T}$的第一、第二和第三主不变量(principal invariants)。易见这些主不变量都是标量,且它的值不依赖基的选择,与$\mathbf{T}$唯一对应。我们还常用$J_1=I_\mathbf{T},J_2=I_\mathbf{T}^2-2II_\mathbf{T},J_3=I_\mathbf{T}^3-3I_\mathbf{T}II_\mathbf{T}+3III_\mathbf{T}$,中文也常译为主不变量(main invariants)。
\end{document}