\documentclass[main.tex]{subfiles}
% 线性代数部分定理证明
\begin{document}
%===============================================
\subsection{范的等价性}
本节将证明有限维向量空间上的不同范的定义是等价的。

\begin{lemma}
设$\mathcal{V}$是数域$\mathbb{F}$上的有限维向量空间,$n=\mathrm{dim}\mathcal{V}$,$\left\|\cdot\right\|$是定义在$\mathcal{V}$的一个范,$\left\|\cdot\right\|_\mathrm{E}\equiv\left(\mathbf{a}|\mathbf{a}\right)^\frac{1}{2}\forall\mathbf{a}\in\mathcal{V}$是定义在$\mathcal{V}$上的欧几里得范,则总存在正实数$k>0,K>0$使得$k\left\|\mathbf{x}\right\|_\mathrm{E}\leq\left\|\mathbf{x}\right\|\leq K\left\|\mathbf{x}\right\|_\mathrm{E},\forall\mathbf{x}\in\mathcal{V}$。
\end{lemma}
\begin{proof}
选择$\mathcal{V}$的任意一组基$\left\{\mathbf{e}_i\right\}$,任一向量$\mathbf{x}\in\mathcal{V}$可表示成$\mathbf{x}=\sum_{i=1}^nx_i\mathbf{e}_i$。由范的一般定义有
\begin{align*}
    \left\|\mathbf{x}\right\|&=\left\|\sum_{i=1}^nx_i\mathbf{e}_i\right\|\\
    &\leq\sum_{i=1}^n\left|x_i\right|\left\|\mathbf{e}_i\right\|\quad\text{(三角不等式,当且仅当}x_1=\cdots=x_n\text{时取等号。)}\\
    &\leq\mathrm{max}\left\{\left|x_i\right|\right\}\sum_{i=1}^n\left\|\mathbf{e}_i\right|\quad\text{(非负实数求和的简易性质,当且仅当}\left|x_1\right|=\cdots\left|x_n\right|\text{时取等号。)}\\
    &\leq\left(\sum_{i=1}^nx_i^2\right)^\frac{1}{2}\sum_{i=1}^n\left\|\mathbf{e}_i\right\|\quad\text{(当且仅当}n=1\text{时取等号。)}\\
    &=K\left\|\mathbf{x}\right\|_\mathrm{E}
\end{align*}
其中$K=\sum_{i=1}^n\left\|\mathbf{e}_i\right\|>0$(基向量$\mathbf{e}_i\neq\mathbf{0}\forall i=1,\cdots,n$)。故总存在$K>0$使得$\left\|\mathbf{x}\right\|\leq K\left\|\mathbf{x}\right\|_\mathrm{E}\forall\mathbf{x}\in\mathcal{V}$。

考虑一般范关于欧几里得范的函数:$f:\mathbb{R}\supset\left(0,+\infty\right]\rightarrow\mathbb{R},f\left(\left\|\mathbf{x}\right\|_\mathrm{E}\right)=\left\|\mathbf{x}\right\|,\forall\mathbf{x}\in\mathcal{V}$。则对任意$\mathbf{x},\mathbf{x}_0\in\mathcal{V}$,有$\left|\left\|\mathbf{x}\right\|_\mathrm{E}-\left\|\mathbf{x}_0\right\|_\mathrm{E}\right|\leq\left\|\mathbf{x}-\mathbf{x}_0\right\|_\mathrm{E}$,故总存在$k^\prime\leq1$使得$\left|\left\|\mathbf{x}\right\|_\mathrm{E}-\left\|\mathbf{x}_0\right\|_\mathrm{E}\right|=k^\prime\left\|\mathbf{x}-\mathbf{x}_0\right\|_\mathrm{E}$。前面又证明了,总存在$K^\prime>0$使得$\left\|\mathbf{x}-\mathbf{x}_0\right\|\leq K^\prime\left\|\mathbf{x}-\mathbf{x}_0\right\|_\mathrm{E}$。故对任意$\epsilon>0$,总可取$\delta=\frac{\epsilon k^\prime}{K^\prime}$,使得只要$\left|\left\|\mathbf{x}\right\|_\mathrm{E}-\left\|\mathbf{x}_0\right\|_\mathrm{E}\right|<\delta$,就有$\left\|\left\|\mathbf{x}\right\|-\left\|\mathbf{x}_0\right\|\right|\leq\left\|\mathbf{x}-\mathbf{x}_0\right\|\leq K^\prime\left\|\mathbf{x}-\mathbf{x}_0\right\|_\mathrm{E}=\frac{K^\prime}{k^\prime}\left|\left\|\mathbf{x}\right\|_\mathrm{E}-\left\|\mathbf{x}_0\right\|_\mathrm{E}\right|<\frac{K^\prime\delta}{k^\prime}=\epsilon$,即函数$f$是连续函数。

由函数的极值定理,闭区间上的连续函数必存在最小值。故在闭区间$\left\|\mathbf{x}\right\|_\mathrm{E}\leq 1$范围内,$f$必存在最小值,即必存在$0\leq a\leq 1$使得$0<f\left(a\right)\leq\left\|\mathbf{x}\right\|\forall\mathbf{x}\in\left\{\mathbf{x}|\mathbf{x}\in\mathcal{V},\left\|\mathbf{x}\right\|_\mathrm{E}\leq1\right\}$。又因为对任一$\mathbf{x}\in\mathcal{V}$总有$\mathbf{\hat{x}}\equiv\frac{\mathbf{x}}{\left\|\mathbf{x}\right\|_\mathrm{E}}\in\left\{\mathbf{x}|\mathbf{x}\in\mathcal{V},\left\|\mathbf{x}\right\|_\mathrm{E}\leq1\right\}$。故总存在$k>0$使得
\begin{align*}
    &k\leq\left\|\mathbf{\hat{x}}\right\|\leq K\left\|\mathbf{\hat{x}}\right\|_\mathrm{E}\\
    \Leftrightarrow&k\leq\frac{\left\|\mathbf{x}\right\|}{\left\|\mathbf{x}\right\|_\mathrm{E}}\leq K\\
    \Leftrightarrow&k\left\|\mathbf{x}\right\|_\mathrm{E}\leq\left\|\mathbf{x}\right\|\leq K\left\|\mathbf{x}\right\|_\mathrm{E}
\end{align*}
\end{proof}

\begin{lemma}
如果存在$k>0,K>0$使得数域$\mathbb{F}$上的有限维向量空间$\mathcal{V}$上的任意两种范的定义$\left\|\cdot\right\|_1,\left\|\cdot\right\|_2$满足$k\left\|\mathbf{x}\right\|_1\leq\left\|\mathbf{x}\right\|_2\leq K\left\|\mathbf{x}\right\|_1,\forall\mathbf{x}\in\mathcal{V}$,则该不等式成立条件定义了两种范的定义之间的等价关系。
\end{lemma}
\begin{proof}
自反性:显然满足。

对称性:只需令$k^\prime=\frac{1}{K},K^\prime=\frac{1}{k}$。

传递性:设$\left\|\cdot\right\|_1,\left\|\cdot\right\|_2,\left\|\cdot\right\|_3$是$\mathcal{V}$上的三种范的定义,正实数$k,k^\prime,K,K^\prime$满足$k\left\|\mathbf{x}\right\|_1\leq\left\|\mathbf{x}\right\|_2\leq K\left\|\mathbf{x}\right\|_1,k^\prime\left\|\mathbf{x}\right\|_2\leq\left\|\mathbf{x}\right\|_3\leq K^\prime\left\|\mathbf{x}\right\|_2,\forall\mathbf{x}\in\mathcal{V}$,则总可取$0<k^{\prime\prime}=kk^\prime\leq KK^\prime=K^{\prime\prime}$使得$k^{\prime\prime}\left\|\mathbf{x}\right\|_1=kk^\prime\left\|\mathbf{x}\right|_1\leq k^\prime\left\|\mathbf{x}\right\|_2\leq\left\|\mathbf{x}\right\|_3\leq K^\prime\left\|\mathbf{x}\right\|_2\leq KK^\prime\left\|\mathbf{x}\right\|_1=K^{\prime\prime}\left\|\mathbf{x}\right\|_1$。
\end{proof}

\begin{theorem}\label{thm:VI.1.1}
有限维向量空间上的任意两种范的定义等价。
\end{theorem}
\begin{proof}
由以上两引理可知有限维向量空间上的任意一种范的定义均与欧几里得范等价。由等价关系性质本命题成立。
\end{proof}

%====================================================
\subsection{伴随算符的唯一存在性}
\begin{lemma}
设$\mathcal{V}$是数域$\mathbb{F}$上的有限维内积空间,给定$\mathcal{V}$上的一个线性泛函$f\in\mathcal{V}^*$,则存在唯一一个向量$\mathbf{b}\in\mathcal{V}$满足$f\left(\mathbf{a}\right)=\left(\mathbf{a}|\mathbf{b}\right),\forall\mathbf{a}\in\mathcal{V}$。
\end{lemma}
\begin{proof}
设$\left\{\mathbf{\hat{e}}_i\right\}$是$\mathcal{V}$的一个规范正交基,$\mathbf{a}=\sum_i\alpha_i\mathbf{\hat{e}}_i$是$\mathcal{V}$中的任一向量,$f$是$\mathcal{V}^*$中的任一线性变换,则$f\left(\mathbf{a}\right)=\sum_i\alpha_if\left(\mathbf{\hat{e}}_i\right)$。若要找到一个向量$\mathbf{b}\in\mathcal{V}$满足$\left(\mathbf{a}|\mathbf{b}\right)=f\left(\mathbf{a}\right)$,则要求等号左边
$\left(\mathbf{a}|\mathbf{b}\right)=\sum_i\alpha_i\overline{\beta_i}$等于等号右边$f\left(\mathbf{a}\right)=\sum_i\alpha_if\left(\mathbf{\hat{e}}_i\right)$,且需对任意$\mathbf{a}\in\mathbf{V}$成立。这相当于要求$\overline{\beta_j}=f\left(\mathbf{\hat{e}}_j\right)$,即$\mathbf{b}=\sum_j\overline{f\left(\mathbf{\hat{e}}_j\right)}\mathbf{\hat{e}}_j$。这就证明了$\mathbf{b}$的存在性。

设$\mathbf{c}\in\mathcal{V}$也满足$\left(\mathbf{a}|\mathbf{b}\right)=\left(\mathbf{a}|\mathbf{c}\right),\forall\mathbf{a}\in\mathcal{V}$,则$\left(\mathbf{a}|\mathbf{b}-\mathbf{c}\right)=0,\forall\mathbf{a}\in\mathcal{V}\Leftrightarrow\mathbf{b}=\mathbf{c}$,故$\mathbf{b}$是唯一的。
\end{proof}

\begin{theorem}
设$\mathcal{V}$是数域$\mathbb{F}$上的有限维内积空间,对$\mathcal{V}$上的任一线性算符$\mathbf{T}\in\mathcal{L}\left(\mathcal{V}\right)$有且只有一个伴随算符$\mathbf{T}^*\in\mathcal{L}\left(\mathcal{V}\right)$。
\end{theorem}
\begin{proof}
给定任意向量$\mathbf{b}\in\mathcal{V}$,均可定义一个线性泛函$f\in\mathcal{V}^*,f\left(\mathbf{a}\right)\equiv\left(\mathbf{Ta}|\mathbf{b}\right),\forall\mathbf{a}\in\mathcal{V}$。由前一条引理,每一个这样的线性泛函都唯一对应一个$\mathbf{b}^\prime\in\mathcal{V}$满足$f\left(\mathbf{a}\right)=\left(\mathbf{a}|\mathbf{b}^\prime\right)$。因此,对每一个线性变换$\mathbf{T}\in\mathcal{L}\left(\mathcal{V}\right)$,都可以定义一个映射$\mathbf{T}^*:\mathcal{V}\rightarrow\mathcal{V}$来将$\mathcal{V}$中的每一个向量$\mathcal{b}$如上所述地对应到$\mathcal{V}$中的另一个向量$\mathbf{b}^\prime$,且这个映射$\mathbf{T}^*$是一个线性算符,因为对任意$\gamma\in\mathbb{F},\mathbf{b},\mathbf{c}\in\mathcal{V}$,
\begin{align*}
    \left(\mathbf{a}|\mathbf{T}^*\left(\gamma\mathbf{b}+\mathbf{c}\right)\right)&=\left(\mathbf{Ta}|\gamma\mathbf{b}+\mathbf{c}\right)\\
    &=\overline{\gamma}\left(\mathbf{Ta}|\mathbf{b}\right)+\left(\mathbf{Ta}|\mathbf{c}\right)\\
    &=\overline{\gamma}\left(\mathbf{a}|\mathbf{T}^*\mathbf{b}\right)+\left(\mathbf{a}|\mathbf{T}^*\mathbf{c}\right)\\
    &=\left(\mathbf{a}|\gamma\mathbf{T}^*\mathbf{b}+\mathbf{T}^*\mathbf{c}\right),\forall\mathbf{a}\in\mathcal{V}\\
    \Leftrightarrow\mathbf{T}^*\left(\gamma\mathbf{b}+\mathbf{c}\right)&=\gamma\mathbf{T}^*\mathbf{b}+\mathbf{T}^*\mathbf{c}
\end{align*}

现证明$\mathbf{T}^*$是唯一的。设另一线性算符$\mathbf{U}\in\mathcal{L}\left(\mathcal{V}\right)$满足命题条件即$\left(\mathbf{Ta}|\mathbf{b}\right)=\left(\mathbf{a}|\mathbf{T}^*\mathbf{b}\right)=\left(\mathbf{a}|\mathbf{Ub}\right)$,则$0=\left(\mathbf{a}|\mathbf{T}^*\mathbf{b}\right)-\left(\mathbf{a}|\mathbf{Ub}\right)=\left(\mathbf{a}|\left(\mathbf{T}^*-\mathbf{U}\right)\mathbf{b}\right),\forall\mathbf{a},\mathbf{b}\in\mathcal{V}\Leftrightarrow\mathbf{T}^*=\mathbf{U}$,故$\mathbf{T}^*$是唯一的。
\end{proof}
\end{document}