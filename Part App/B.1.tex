\documentclass[main.tex]{subfiles}
% {$\mathbb{R}^n$空间上的一些拓扑概念。
\begin{document}
本附录的内容并不新。在高等数学中我们已经学过邻域、去心邻域、内点、外点、边界点、聚点、孤立点、闭集、开集、连通集、区域、有界区域等概念\cite[\S 7.1]{华工高数2009下},但当时学习的定义仅限于$\mathbb{R}^2$的情况。

\begin{definition}[开集]\label{def:B.1}
    如果对于$\mathbb{R}^n$的子集$S$中一个元素$\mathbf{x}_0\in S$,存在正实数$\delta>0$使得只要$\left\|\mathbf{x}-\mathbf{x}_0\right\|<\delta$则$\mathbf{x}\in S$,就称\emph{$\mathbf{x}_0$在$S$内}。所有这样的点$\mathbf{x}_0$的集合称为集合$S$的\emph{内部(interior)},记为$\mathrm{int}S$。如果$S$的所有元素都在$S$内($S=\mathrm{int}S$),就称$S$是\emph{开集(open set)}。一个含有某元素$\mathbf{x}_0$的开集$S$又可称为该点$\mathbf{x}_0$的一个\emph{邻域(neighborhood)}。
\end{definition}

由定义,整个$\mathbb{R}^n$是一个开集。$\mathbb{R}^n$中的开集的交集也是开集。$\mathbb{R}^n$中的有限个开集的并集也是开集。空集不含任何元素,故命题“空集是一个开集”\emph{虚真(vacuously true)}。这些结论都需要证明但此略。

\begin{definition}[闭集]\label{def:B.2}
    如果集合$S\subset\mathbb{R}^n$中的点$\mathbf{x}_0$的每个邻域都含有至少一个$S$中的点(可以就是点$\mathbf{x}_0$本身),则称点$\mathbf{x}_0$是\emph{$S$的闭包中的点},所有这样的点$\mathbf{x}_0$的集合称$S$的\emph{闭包(closure)},记为$\mathrm{cl}S$。如果点$\mathbf{x}_0$的每个邻域都含有至少一个$S$中的与$\mathbf{x}_0$不同的点$\mathbf{x}$,则称$\mathbf{x}_0$是$S$的一个\emph{极限点(limit point)}。$S$的所有极限点的集合称$S$的\emph{导集(derived set)}。如果集合$S$包含它的所有极限点,则称集合$S$是\emph{闭集(closed set)}。
\end{definition}

闭集有若干个等价定义。由定义\ref{def:B.2},可证——

\begin{theorem}\label{thm:B.1}
    集合$\mathbf{S}\in\mathbb{R}^n$是闭集当且仅当:
    \begin{enumerate}
        \item 集合$S$等于其闭包($S=\mathrm{cl}S$);
        \item 集合$S$相对于$\mathbb{R}^n$的补集$\mathbb{R}^n\setminus S$是开集;
        \item 集合$S$包含它的所有边界上的点(见定义\ref{def:B.3})($\partial S\subset S$)。
    \end{enumerate}
\end{theorem}
\begin{corollary}
    一个集合是开集当且仅当它不包含其任何边界上的点。
\end{corollary}

由定义\ref{def:B.2},整个$\mathbb{R}^n$是一个闭集。$\mathbb{R}^n$中的闭集的交集也是闭集。$\mathbb{R}^n$中的有限个闭集的并集也是闭集。命题“空集是一个闭集”虚真。这些结论都需要证明但此略。

\begin{definition}[边界]\label{def:B.3}
    如果对于$\mathbb{R}^n$的子集$S$中的一个元素$\mathbf{x}_0\in S$和任意正实数$\delta>0$,都存在至少一个$\mathbf{x}\in S$满足$\left\|\mathbf{x}-\mathbf{x}\right\|=\delta$(显然$\mathbf{x}\neq\mathbf{x}_0$)和至少一个$\mathbf{y}\notin S$满足$\left\|\mathbf{y}-\mathbf{x}\right\|=\delta$,则称$\mathbf{x}_0$是在$S$的\emph{边界上的点(boundary point)}。所有这样的点$\mathbf{x}_0$的集合称为集合$S$的\emph{边界(boundary)},常记为$\partial S$。
\end{definition}

\begin{definition}[孤立点]\label{def:B.4}
    如果对于点$\mathbf{x}_0\in S\subset\mathbf{R}^n$,存在正实数$\delta>0$使得$\left\{\mathbf{x}\in\mathbb{R}^n|\left\|\mathbf{x}-\mathbf{x}_0\right\|=\delta\right\}\cap S=\left\{\mathbf{x}_0\right\}$,则称$\mathbf{x}_0$是$S$的一个孤立点。
\end{definition}

\begin{example}
    设$S=\left(0,1\right]\cup\left\{2\right\}$,则$\mathrm{int}S=\left(0,1\right)$,$\partial S=\left\{0,1,2\right\}$,$S$的所有极限点是$\left[0,1\right]$。$2$是$S$的一个孤立点。
\end{example}
\end{document}