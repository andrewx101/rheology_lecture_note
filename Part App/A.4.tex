\documentclass[../main.tex]{subfiles}
\begin{document}
我们先引入关于子群的共轭子群的概念。

\begin{definition}\label{def:A.8}
    设$G$是一个群,对任意$a,b\in G$,记其群操作为$a\circ b$,单位元为$e$,$a$的逆元为$a^{-1}$。若$G$中存在一个$i\in G$使$G$中的两个元素$a,b\in G$满足$b=i\circ a\circ i^{-1}$,则称$b$为$a$的一个\emph{共轭(conjugate)}。
\end{definition}

其实,对于$a\in G$,选不同的元素$i\in G$都可以为$a$找到一个共轭元素。我们可以更仔细地称$b=i\circ a\circ i^{-1}$为元素$a$通过元素$i$获得的共轭。但是,任意给定$G$的两个元素$a,b$,并不一定总能找到一个$i\in G$使$b=i\circ a\circ i^{-1}$。事实上,两个群元素的共轭作为一个关系,是一个等价关系。与给定元素共轭的所有元素将构成一个等价类。不同元素的共轭等价类分割了整个群$G$。

我们简短地证明为什么共轭是一个等价类,就不单独列为引理了。设关系$\sim\in G\times G$满足:$a\sim b\Leftrightarrow\exists i\in G,b=i\circ a\circ i^{-1}$。我们需要证明$\sim$满足等价关系的三个性质:自反性、对称性和传递性。首先,自反性要求$a\sim a$,即存在$i\in G$使$a=i\circ a\circ i^{-1}$。取$i=e$即可。其次,对称性要求$a\sim b\Rightarrow b\sim a$。若存在$i\in G$使$b=i\circ a\circ i^{-1}$,则取$i^{-1}$即可。最后,传递性要求$a\sim b,b\sim c\Rightarrow a\sim c$。若存在$i,j\in G$使$b=i\circ a\circ i^{-1},c=j\circ b\circ j^{-1}$,则有$c=j\circ i\circ a\circ i^{-1}\circ j^{-1}$,取$i^\prime=j\circ i$即可。故共轭是一个等价关系。给定任一$G$的元素$a$,由共轭关系$\sim$可形成一个等价类$\left\llbracket a\right\rrbracket_\sim$,称$a$的\emph{共轭类(conjugacy class)}。

另一个更加平凡的事实是,对每一$G$的元素$a$,通过给定元素$i\in G$,有且只有一个共轭$b=i\circ a\circ i^{-1}$。这是因为,给定的$a, i\in G$是确定的,而且对每一$i\in G$有且只有一个$i^{-1}\in G$,故整个表达式$b=i\circ a\circ i^{-1}$是唯一的。这是下面的引理\ref{lem:A.3}中我们之所以能够定义里面的映射$Q$的基础。

\begin{lemma}\label{lem:A.3}
    设$G$是一个群,对任意$a,b\in G$,记其群操作为$a\circ b$,单位元为$e$,$a$的逆元为$a^{-1}$。若$V$是$G$的一个子群。给定$i\in G$,定义映射$Q:V\rightarrow G,$
    \[Q\left(v\right)=i\circ v\circ i^{-1},\quad\forall v\in V\]
    则集合$V^*\equiv\mathrm{ran}Q$是$G$的一个子群,且映射$Q$是由$V$到$V^*$的同构映射。
\end{lemma}
\begin{proof}
    首先,$V^*$是$G$的子集。对任意$v_1,v_2\in V$,有$Q\left(v_1\right)=i\circ v_1\circ i^{-1},Q\left(v_2\right)=i\circ v_2\circ i^{-1}$,则有
    \[Q\left(v_1\right)\circ Q\left(v_2\right)=i\circ v_1\circ i^{-1}\circ i\circ v_2\circ i^{-1}=i\circ v_1\circ v_2\circ i^{-1}=Q\left(v_1\circ v_2\right)\in V^*\]
    其中由$V$的封闭性,$v_1\circ v_2$也是$V$的元素,故也能被$Q$映射到$V^*$。这证明了,任意两个$V^*$中的元素的群运算结果也在$V^*$中,即$V^*$满足封闭性。

    其次,我们将看到,$e\in V^*$。由$V$是$G$的子群,故$e\in V$。由$Q$的定义,有$Q\left(e\right)=i\circ e\circ i^{-1}=e\in V^*$。故$V^*$包含单位元。

    再次,我们将看到,$V^*$包含逆元。对任意$v\in V$,有$Q\left(v\right)=i\circ v\circ i^{-1}$,同时也有
    \[Q\left(v\right)^{-1}=i\circ v^{-1}\circ i^{-1}=Q\left(v^{-1}\right)\in V^*\]
    其中由$V$的封闭性,$v^{-1}$也是$V$的元素,故也能被$Q$映射到$V^*$。故$V^*$包含逆元。

    综上,$V^*$是一个群。结合$V^*$是$G$的子集的事实,就有$V^*$是$G$的子群。

    以下证明$Q$是由$V$到$V^*$的同构映射。之前已经证明了$Q\left(v_1\right)\circ Q\left(v_1\right)=Q\left(v_1\circ v_2\right)$即$Q$是$V$上的同态映射,只需证明$Q$是由$V$到$V^*$的双射。首先,由于$V^*=\mathrm{ran}Q$,故$Q$是由$V$到$V^*$的满射。其次,$Q$是单射:$\forall v_1,v_2\in V$,若$Q\left(v_1\right)=Q\left(v_2\right)$,则有
    \[\begin{aligned}
                            & i\circ v_1\circ i^{-1}  =i\circ v_2\circ i^{-1}                                      \\
            \Leftrightarrow & i^{-1}\circ i\circ v_1\circ i^{-1}=i^{-1}\circ i\circ v_2\circ i^{-1}                \\
            \Leftrightarrow & i^{-1}\circ i\circ v_1\circ i^{-1}\circ i= i^{-1}\circ i\circ v_2\circ i^{-1}\circ i \\
            \Leftrightarrow & v_1=v_2
        \end{aligned}\]
    因此$Q$是由$V$到$V^*$的双射,即是同构映射。
\end{proof}

\begin{definition}[共轭子群]\label{def:A.9}
    设$G$是一个群,对任意$a,b\in G$,记其群操作记为$a\circ b$,单位元为$e$,$a$的逆元为$a^{-1}$。若$V$是$G$的一个子群。给定$i\in G$,定义映射$Q:V\rightarrow G,$
    \[Q\left(v\right)=i\circ v\circ i^{-1},\quad\forall v\in V\]
    则称$V^*\equiv\mathrm{ran}Q$是子群$V$通过元素$i$形成的共轭子群。
\end{definition}

我们从引理\ref{lem:A.3}和定义\ref{def:A.9}可见,现在我们讨论的焦点放在给定元素$i\in G$,它能为任一$G$的子群中的元素指定其共轭(即通过$i$构建的映射$Q$)而生成一个共轭子群$V^*$。这是对$G$的任一元素$i$,针对$G$的任一子群$V$,都能享有的事实。

\begin{lemma}\label{lem:A.4}
    设$\left(\mathcal{E},d\right)$是一个度量空间,若其上的等距群$\mathcal{I}$中存在一个满足条件G1至G6、S1至S4和N1的子群$\mathcal{V}$,则这样的子群只有一个。
\end{lemma}
\begin{proof}
    设$\mathcal{V}^\prime$也是满足上述条件的子群,则给定任意两点$X,Y\in\mathcal{E}$,$\mathcal{V}$中必存在唯一一个等距变换$\mathbf{u}$、$\mathcal{V}^\prime$中必存在唯一一个等距变换$\mathbf{u}^\prime$,分别满足$\mathbf{u}=Y-X,\mathbf{u}^\prime=Y-^\prime X$,其中$-$和$-^\prime$分别是由$\mathcal{V}$和$\mathcal{V}^\prime$的传递性所定义的记号。相应地有$Y=X+\mathbf{u}=X+^\prime\mathbf{u}^\prime$。

    定义由$\mathcal{V}$到$\mathcal{V}^\prime$的映射$f:\mathcal{V}\rightarrow\mathcal{V}^\prime,$
    \[X+\mathbf{u}=X+^\prime f\left(\mathbf{u}\right), \quad\forall X\in\mathcal{E},\forall\mathbf{u}\in\mathcal{V}\]
    易证$f$是双射且$f\left(\mathbf{0}\right)=\mathbf{0}$,其中$\mathbf{0}$是$\mathcal{E}$上的恒等变换,故它同时是$\mathcal{V}$和$\mathcal{V}^\prime$的单位元。

    $f$是由$\mathcal{V}$到$\mathcal{V}^\prime$的同构映射:对任意$\mathbf{u},\mathbf{v}\in\mathcal{V}$和$X_0\in\mathcal{V}$必存在$X,Y\in\mathcal{V}$满足
    \[X=X_0+\mathbf{u},\quad Y=X_0+\mathbf{v}=X+\mathbf{v}-\mathbf{u}\]
    若记$\mathbf{u}^\prime=f\left(\mathbf{u}\right),\mathbf{v}^\prime=f\left(\mathbf{v}\right)$,则亦有
    \[X=X_0+^\prime\mathbf{u}^\prime,\quad Y=X+^\prime\mathbf{u}^\prime-\mathbf{v}^\prime\]
    由于$\mathcal{V}$和$\mathcal{V}^\prime$上的范的定义方式都来自同一度量$d$,故有
    \[\left\|\mathbf{v}-\mathbf{u}\right\|=\left\|\mathbf{v}^\prime-\mathbf{u}^\prime\right\|\]
    特别地,当$\mathbf{u}=\mathbf{0}$时,$\mathbf{u}^\prime=\mathbf{0}$,上式说明$\left\|\mathbf{v}\right\|=\left\|\mathbf{v}^\prime\right\|$对任意$\mathbf{v}\in\mathcal{V}$成立。故
    \begin{align*}
        \left\|\mathbf{v}-\mathbf{u}\right\|                                                                       & =\left\|\mathbf{v}^\prime-\mathbf{u}^\prime\right\|                                                                      \\
        \Rightarrow\left\|\mathbf{v}-\mathbf{u}\right\|^2                                                          & =\left\|\mathbf{v}^\prime-\mathbf{u}^\prime\right\|^2                                                                    \\
        \Leftrightarrow\left\|\mathbf{u}\right\|^2-2\left(\mathbf{u}|\mathbf{v}\right)+\left\|\mathbf{v}\right\|^2 & =\left\|\mathbf{u}^\prime\right\|^2-2\left(\mathbf{u}^\prime|\mathbf{v}^\prime\right)+\left\|\mathbf{v}^\prime\right\|^2 \\
        \Leftrightarrow\left(\mathbf{u}|\mathbf{v}\right)                                                          & =\left(\mathbf{u}^\prime|\mathbf{v}^\prime\right)
    \end{align*}
    即$f$保持内积。由类似引理\ref{lem:II.2.2}的证明过程可知$f$是由$\mathcal{V}$到$\mathcal{V}^\prime$的同构线性变换。

    由外延公理,如果两个集合$A$和$B$之间存在一个双射$f$满足$f\left(x\right)=x,\forall x\in A$则$A=B$。现在对任意$X\in\mathcal{E}$和$\mathbf{v}\in\mathcal{V}$,设$\mathbf{u}=X-X_0,X_0\in\mathcal{E}$,则
    \begin{align*}
        X+\mathbf{v} & =X_0+\mathbf{u}+\mathbf{v}                                     \\
                     & =X_0+^\prime f\left(\mathbf{u}+\mathbf{v}\right)               \\
                     & =X_0+^\prime f\left(\mathbf{u}\right)+f\left(\mathbf{v}\right) \\
                     & =X_0+\mathbf{u}+^\prime f\left(\mathbf{v}\right)               \\
                     & =X+f\left(\mathbf{v}\right)
    \end{align*}
    均成立,故$f\left(\mathbf{v}\right)=\mathbf{v},\forall\mathbf{v}\in\mathcal{V}$,即$\mathcal{V}^\prime$与$\mathcal{V}$作为内积空间是同一的。
\end{proof}

\begin{theorem*}[等距变换的表示定理\ref{thm:II.3.2}]
    设$\mathcal{E}$是一个欧几里得空间,$\mathcal{V}$是其平移空间,选定任一点$X_0\in\mathcal{E}$,则$\mathcal{E}$上的任一等距变换$i:\mathcal{E}\rightarrow\mathcal{E},i\in\mathcal{I}$都可表示为
    \[
        i\left(X\right)=i\left(X_0\right)+\mathbf{Q}_i\left(X-X_0\right)
    \]
    其中$\mathbf{Q}_i$是$\mathcal{V}$上的一个正交算符,由$i$唯一确定。
\end{theorem*}
\begin{proof}
    由引理\ref{lem:A.3}、\ref{lem:A.4}和\ref{lem:II.2.2},给定任一$\mathcal{E}$上的等距变换$i\in\mathcal{I}$,$\mathcal{V}$在$\mathcal{I}$中的共轭子群都是它自己,且每个由$i$引出的共轭映射$\mathbf{Q}_i\mathbf{v}=i\circ\mathbf{v}\circ i^{-1}$就是$\mathcal{V}$上的自同构映射,故$\mathbf{Q}$是$\mathcal{V}$上的正交算符(由定义\ref{def:II.2.23},幺正算符就是内积空间上的自同构映射)。

    注意到$i\circ\mathbf{v}=\mathbf{Q}_i\mathbf{v}\circ i$,则对任一$X_0\in\mathcal{E}$,有
    \[i\circ \mathbf{v}\left(X_0\right)=i\left(X_0+\mathbf{v}\right),\quad \mathbf{Q}_i\mathbf{v}\circ i\left(X_0\right)=i\left(X_0\right)+\mathbf{Q}_i\mathbf{v}\]
    令$X=X_0+\mathbf{v}$,就有
    \[
        i\left(X\right)=i\left(X_0\right)+\mathbf{Q}_i\left(X-X_0\right)
    \]

    $\mathbf{Q}_i$由$i$唯一确定:设另有一正交算符$\mathbf{P}:\mathcal{V}\rightarrow\mathcal{V}$满足$\mathbf{Pu}=i\left(X+\mathbf{u}\right)-i\left(X\right),\forall\mathbf{u}\in\mathcal{V},\forall X\in\mathcal{E}$,则$\left(\mathbf{P}-\mathbf{Q}_i\right)\mathbf{u}=\mathbf{Pu}-\mathbf{Q}_i\mathbf{u}=\mathbf{0},\forall\mathbf{u}\in\mathcal{V}$。
\end{proof}
\end{document}